\documentclass[9pt]{article}
\usepackage{amsmath}
\usepackage{tikz}
\usepackage{pgfplots}
\RequirePackage{amsfonts}
\usepackage{enumitem}
\usepackage{epsfig}  
\usepackage[T2A,T1]{fontenc}
\usepackage[utf8]{inputenc}
\usepackage[russian,english]{babel}

\begin{document}
\title{\foreignlanguage{russian}{Эконометрика, CheatSheet}}
\author{\foreignlanguage{russian}{Перепелкин Владимир}}
\maketitle
\begin{otherlanguage*}{russian}
\subsection*{Полезные формулы}
\subsubsection*{Матстат}
\begin{align*}
&\operatorname{E} (X) = \begin{cases} 
\sum_{i=1}^n x_i \cdot p_i \\
\int_{-\infty}^{+\infty} f_x(x) \cdot x dx 
\end{cases} 
&\mathbb{D} (X) = \begin{cases} 
\operatorname{E} \big( (x - \operatorname{E}(x))^2\big) \\
\operatorname{E}(X^2) - (E(X))^2
\end{cases} \\
&\text{corr} = \dfrac{\text{cov} (X, Y)}{\sigma_X \cdot \sigma_Y}
& \text{Var} (X) = \mathbb{E} (X^2) - \mathbb{E} (X)^ 2 \\
& \text{cov} (X, Y) = \mathbb{E} \Big( (X - \mathbb{E} (X)) \cdot  (Y - \mathbb{E}(Y)\Big) = \mathbb{E}(XY) - \mathbb{E}(X) \mathbb{E}(Y) \\
&\text{MSE} = \mathbb{E} ( \hat \theta - \theta) ^ 2 \\
&\mathbb{D} (X + Y) = \mathbb{D} (X) + 2 \cdot \text{cov} (X, Y) + \mathbb{D} (Y)   \\
&\mathbb{D} (X \cdot \alpha) = \alpha^2 \cdot \mathbb{D} (X) 
&\mathbb{E} (X \cdot \alpha) = \alpha \cdot \mathbb{E} (X) \\
& \text{cov} (X + \alpha, Y + \beta) = \text{cov} (X, Y) 
& \mathbb{E} (\alpha \cdot X + \beta \cdot Y) = \alpha \mathbb{E} (X) + \beta \mathbb{E}(Y)
\end{align*} 
\subsubsection*{Оценки параметров}
\begin{align*}
&\hat \mu = \bar{X} = \dfrac{\sum_{i=1}^n X_i}{n}
&\widehat{\sigma}^2 = \dfrac{1}{n-1} \sum_{i=1}^n (X_i - \bar{X}) ^ 2 
\\
&\widehat{\text{cov}} (X, Y) = \dfrac{1}{n-1} \sum_{i=1}^n (X_i - \bar{X}) (Y_i - \bar{Y}) 
\end{align*}
\subsubsection*{Для распределений}
\begin{align*}
&Z \sim N(0,1 ) 
&t_k = \dfrac{Z}{\sqrt{\chi^2_k / k}} \\
&\chi^2_k = Z_1^2 + \ldots + Z_k^2 
&F(m, n) = \dfrac{\chi^2_m / m}{\chi^2_n / n} \\
&F(m, n) = \dfrac{1}{F(n, m)}
\end{align*}
\subsubsection*{Для оценок}
\begin{itemize}
\item Несмещенность: $ \mathbb{E} (\hat \theta) = \theta $
\item Эффективность: $ \hat \theta_1 $ эффективна в классе $ K $, если 

 $ \text{MSE}_{\theta }({\widehat {\theta }}_{1}-\theta )^{2}\le  \text{MSE}_{\theta }({ \widehat {\theta }}_{i}-\theta )^{2} \,\,\,\,\forall \theta_i \in K$
\item Состоятельность: $ \hat \theta \rightarrow \theta \,\,\,\, \forall \theta \in \Theta $ при $ n \rightarrow \infty $  
\end{itemize}
\subsubsection*{Парная регрессия}
\begin{itemize}
\item Общие формулы 
\begin{align*}
&\text{TSS} = \sum_{i=1}^n (Y_i - \bar{Y}) ^2 
&\text{ESS} = \sum_{i=1}^n (\hat{Y_i} - \bar{Y})^2  \\
&e_i = Y_i - \hat Y_i 
&\text{RSS} (\hat \beta_0, \hat \beta_1) = \sum_{i=1}^n e_i^2 \\
& R^2 = \dfrac{ESS}{TSS} = 1 - \dfrac{RSS}{TSS} 
& \text{MSE} = \frac{\sum_{i=1}^n (\beta_0 + \beta_1 \cdot X_i - Y_i)^2}{n} \rightarrow \min \\
&\hat y = \hat \beta_0 + \hat \beta_1 \cdot X \\
&\hat \beta_0 = \bar{Y} - \hat{\beta_1} \cdot X 
& \hat \beta_0 = \bar{Y} - \dfrac{\widehat{\text{cov}}(x, y)}{\widehat{\text{Var}}(X)} \cdot X \\
&\hat \beta_1 =\dfrac{\sum_{i=1}^n (X - \bar{X})(Y - \bar{Y})}{\sum_{i=1}^n (X - \bar{X})} &
\hat \beta_1 = \dfrac{\widehat{\text{cov}}(x, y)}{\widehat{\text{Var}}(X)} = \dfrac{\bar{X  Y} - \bar{X} \cdot \bar{Y}}{\bar{X^2} - (\bar{X})^2}
\end{align*}
\item Частные случаи

Если $ \hat y = \beta \cdot X $, то $ \hat \beta = \dfrac{\sum_{i=1}^n X_i \cdot Y_i}{\sum_{i=1}^n X_i^2} $
\item Оценки 
\begin{align*}
&\text{Var}( \hat \beta_1 ) = \dfrac{\sigma_{\varepsilon}^2}{\sum_{i=1}^n (X_i - \bar{X})^2} 
&\widehat{\text{cov}} ( \hat \beta_0, \hat \beta_1) = \dfrac{\sum X_i}{n \sum (X_i - \bar{X})^2} \cdot \hat \sigma_{\varepsilon}^2 \\
& \widehat{\text{var} (e)} = \dfrac{\sum_{i=1}^n e_i^2}{n-2}
\end{align*}
\item Доверительные интервалы для парной регрессии
\begin{align*}
\hat \beta - t_{n-2, \alpha / 2} \cdot \widehat{s.e.} (\hat \beta) \le \beta \le \hat \beta + t_{n-2, \alpha / 2} \cdot \widehat{s.e.} (\hat \beta)
\end{align*}
\item Гипотеза $ \beta = \beta^* $
\begin{align*}
t_{\text{stat}} = \dfrac{\hat \beta - \beta^*}{\widehat{s.e.}(\hat \beta)} \sim t_{n-2}
\end{align*}
\end{itemize}
\subsubsection*{Множественная регрессия}
\begin{align*}
MSE = \dfrac{1}{n} \cdot (X \cdot \beta - y)^T \cdot (X \cdot \beta - y) \\
\beta =  (X^T \cdot X)^{-1} \cdot X^T \cdot y 
\end{align*}
\subsubsection*{Теорема Гаусса-Маркова}
Оценки $\hat \beta_1, \hat \beta_2 $ оптимальны в классе линейных несмещенных оценок, если:
\begin{itemize}
\item Модель данных правильно специфицирована
\item Все $ X_i$ детерминированы и не все равны между собой
\item $\mathbb{E} (\varepsilon_i) = 0 \,\,\, \forall i $
\item Дисперсия ошибок одинакова
\item $\text{cov} (\varepsilon_i, \varepsilon_j) = 0 \,\,\, \forall i \ne j $
\end{itemize}
\end{otherlanguage*} 
\end{document}
