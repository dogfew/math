\documentclass{article}
\usepackage{amsmath}
\usepackage{epsfig}  
\usepackage[T2A,T1]{fontenc}
\usepackage[utf8]{inputenc}
\usepackage[russian,english]{babel}

\begin{document}
\title{\foreignlanguage{russian}{Динамическая оптимизация в экономике и финансах}}
\maketitle

\begin{otherlanguage*}{russian}
\section{\foreignlanguage{russian}{Вариацонное исчисление}}
Матанализ ; Варисч
переменная $x$ ; $y(t)$ - траекториия

функция $f(x)$ ; функционал $V[y(t)]$ 

производная $f'(x)$ ; Вариация $\delta V[y(t)]$ - функционал 

$\Delta f(x) $- приращение ; $ \Delta V[y(t)] $

\begin{enumerate}

\item Задача Коши

Условия:
\begin{align}
y^{''}(t) = \frac{1}{2} t \\
y(0) = 0 \\
y(1) = 1 
\end{align}
Решение:
\begin{align}
y^{'} (t) = \frac{t^2}{4} + C_1 \\
y(t) = \frac{t^3}{12} + C_1 t + C_2 \\
\begin{cases}
y(0) = \frac{0^3}{12} + C_1 \cdot 0 + C_2 = 0 \\
y(1) = \frac{1^3}{12} + C_1 \cdot 1 + C-2 = 1 
\end{cases} \Rightarrow
\begin{cases}
C_2 = 0 \\
C_1 = \frac{11}{12}
\end{cases}
\end{align}
Ответ: $y(t) = \frac{t^3}{12} + \frac{11}{12} t + 0 $

\item Уравнение с разделяющимися переменными

Условия:
\begin{align}
y^{'} - y = 2 t - 3 
\end{align}

Идея: все что с $y$ в левую часть. всё что с $t$ - в правую часть

Факты:
\begin{align}
y^{'} (t) = \frac{dy(t)}{dt}
\end{align}

Решение: 

$\frac{dy}{dt} = y + 2t - 3$

Введем обозначение:
\begin{align}
\frac{dy}{dt}= y^{'} = z \\
z = y + 2 t - 3 \\
\frac{dz}{dt} = \frac{d(y + 2 t - 3)}{dt} = z + 2 \\
\frac{1}{z + 2} dz = dt \\
\int \frac{1}{z + 2 } dz = \int dt \\
\ln |z + 2 | = C_1 + t \\
\ln | z + 2 | = \ln C_2 + \ln e^t \\
z + 2 = C_2 \cdot e^t \\
y + 2t - 3 + 2 = C_2 \cdot e ^ t 
\end{align}

Ответ: $y(t) = C_2 \cdot e^t - 2 t + 1 $

\item Однородные линейные уравнения с некратными вещественными корнями

Условия:
$ y^ {'''} - y^{'} = 0 $

Решение:

1. Характеристическое уравнение: $ \lambda ^ 3 - \lambda^1 = 0 $. То есть лямбды, где степень лямбды соответствует порядку производной. 

2.Корни этого характеристического уравнения: $ \lambda_1 = 0; \lambda_2 = 1; \lambda_3 = - 1$. 

3. Ответ: $y(t) = C_1 \cdot e^{0t} + C_2 \cdot e^{1t} + C_3 \cdot e^{-1t} $. 

4. В общем виде: $y(t) = C_1 \cdot e ^ {\lambda_1 t} + \cdots + C_n \cdot e ^ {\lambda_n t} $О
\item Однородные линейные уравнения с кратными вещественными корнями

Условия: 
$y^{'''} - 3 y^{''} + 3 y ^{'} - y = 0 $

1. Характеристическое уравнение: $\lambda^3 - 3 \lambda ^2 + 3 \lambda - 1 = 0 = (\lambda - 1)^3 $

2. Корни этого уравнения: $ \lambda_1 = \lambda_2 = \lambda_3 = 1$

3. Ответ: $y(t) = C_1 \cdot e ^{1 \cdot t} + t \cdot C_2 \cdot e^{1\cdot t} + t ^ 2 \cdot C_2 \cdot e^{1 \cdot t} $.  

4. В общем виде: $ y(t) = C_1 e ^ {\lambda t} + t \cdot C_2 \cdot e^{\lambda t} + t^2 \cdot C_3 \cdot e^{\lambda t} + \cdots + t^{n-1} \cdot C_n \cdot e^{\lambda t} $. Мнемоническое правило: кратность добавляет умножение на t. 

\item Однородные линейные уравнения с некратными комплексными корнями

Условия: 

$ y ^{''} + a^2 y = 0$ 

1. Характеристическое уравнение: $ \lambda^ 2 + a ^ 2 = 0 $ 

2. Корни этого уравнения: $ \lambda_{1,2} = \pm a \cdot i$. Комплексное число задаётся как $ \lambda = \alpha + \beta \cdot i $ 

3. Ответ: $ y(t) = C_1 \cdot \cos(at) + C_2 \cdot \sin(at) $.  

4. В общем виде $ y(t) = C_1 \cdot e ^ {\alpha t} \cdot \cos (\beta t) + C_2 \cdot e^{\alpha t} \cdot \sin(\beta t) + \cdots  C_n \cdot e ^ {\alpha_n t} \cdot \cos (\beta_n t) + C_{n+1} \cdot e^{\alpha_n t} \cdot \sin(\beta_n t)$

Мнемоническое правило: для каждой пары комплексных чисел пишется пара из cos и sin. 
 
\item Однородные линейные уравнения с кратными комплексными корнями

1.Условия: $ \cdots $

2. Корни этого уравнения: $ \lambda_{1, 2} = \pm a \cdot i; \lambda_{3,4} = \pm a \cdot i$

3. Ответ: $ y(t) = C_1 \cos (at) + C_2 \sin (at) + t \cdot \Big( C_3 \cos (at) + C_4 \cdot \sin (at) \Big)$

\item Однородные линейные уравнения со специальной правой частью 

1. Условия: $y^{''} + y = t ^ 2 + t$

2. Решение однородного уравнения: $ y^{''} + y = 0 $

2.1 Характеристическое уравнение: $ \lambda ^ 2 + 1 = 0 $ 

2.2 Корни этого х.у.: $ \lambda_{1, 2} = \pm i $ 

2.3 Ответ: $ y(t) = C_1 \cos t + C_2 \sin t $ 

3. Подбор частного решения. Введём обозначение: $ t^2 + t = f(t) $. Если мы можем представить правую часть в общем виде как $ f(t) = Pm(t) \cdot e^{\lambda t}$. Тогда $ y_2 (t) = Q_m (t) \cdot e ^{\lambda t} \cdot t ^k $

3.1 Попробуем представить правую часть в желаемом виде: $t^2 + t = (t^2 + t) \cdot e ^ {0t} $

3.2 $ y_2(t) = (A t ^ 2 + Bt + C) \cdot e ^{0t} $, поскольку корень х.у. не совпал с 0, то $t^k$ игнорируется. $ k $ - количество корней, которые соответствуют числу над $e$.  

3.3 $ y ^ {'}_2 (t) = 2 At + t$

3.4 $ y^{''}_2 (t) = 2 A  $

3.5 Подставим частное решение в исходное уравнение. 
\begin{align}
2 A + A t ^ 2 + Bt + C = t ^ 2 + t \\
\begin{cases}
2 A + C = 0 & 2 = - 2 \\
A = 1 & B = 1 
\end{cases} \Rightarrow y_2 (t) = t ^ 2 + t - 2 
\end{align}
4. Ответ : $ y(t) = y_{homo} (t) + y_{part} (t) = C_1 \cos t + C_2 \sin t + t ^ 2 + t - 2 $

\item Однородные линейные уравнения со специальной правой частью 

\begin{enumerate}
\item Условие: $y^{(4)} + 2 y ^{''} + y = \sin t $ 

\item Найти y однородное 

Характеритическое уравнение: $ \lambda ^ 4 + 2 \lambda ^ 2 + 1 = (\lambda ^ 2 + 1) ^ 2 = 0 $
Его корни: $\lambda_{1, 2} = \pm i$, $\lambda_{3, 4} = \pm i$ 

Решение однородного уравнения: $ C_1 \cdot \cos t + C_2 \cdot \sin t + t \big( C_3 \cos t + C_4 \sin t \big)$  

\item Как находится частное решение в этом случае
Представим правую часть и т.д.

$ f(t) = \big( P_m(t) \cos \beta t + Q_n(t) \sin \beta t \big) \cdot e ^{\alpha t} $

$ y_2(t) = \big( S_m (t) \cos \beta t + R_n (t) \sin \beta t \cdot e^{\alpha t} + t^k \big)$

\item Найти y частное

$ \sin t = ( 0 \cdot \cos \beta t + 1 \cdot \sin \beta t) \cdot e ^{0 t}$

$ \lambda = 0 \pm 1 \cdot i $

$ y_2 (t) = (A \cos t + B \sin t) \cdot e ^ {0 t} \cdot t ^ 2 $

\item Посчитать ответ как сумму однородного решения + частного. 

Ответ: $ y(t) = y_{homo} (t) + y_{part} (t) = $
\end{enumerate}
\end{enumerate}
\end{otherlanguage*}
\end{document}