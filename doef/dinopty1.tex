\documentclass{article}
\usepackage{amsmath}
\usepackage{epsfig}  
\usepackage[T2A,T1]{fontenc}
\usepackage[utf8]{inputenc}
\usepackage[russian,english]{babel}

\begin{document}
\title{\foreignlanguage{russian}{Динамическая оптимизация в экономике и финансах}}
\maketitle

\begin{otherlanguage*}{russian}
\section{\foreignlanguage{russian}{Вариацонное исчисление}}

\begin{enumerate}

\item Функционал

Переменная величина V называется функционалом, зависящим от функции y(t) 	
\begin{equation}
V = V(y(t)), \,\,\, \operatorname{if} \,\, \forall y(t) \in \operatorname{M} \rightarrow V 
\end{equation}
где $M$ - класс функций, а $V$ - число 

Примеры функционалов
\begin{align}
V = \max_{t \in [0, T]} y(t) \\
V = \int_{0}^T y(t) dt \\
V = y(a) 
\end{align}
Пример задачи (микроэкономика). Есть фирма, которая имеет функцию издержек и т.д. Причем спрос зависит не только от цен, но и от их динамики. 
\begin{align}
Q(t) = D(P(t), P^{'}(t)) \\
C(Q) - \operatorname{costs} \\
\pi = P(t) \cdot D(P(t), P^{'} (t)) - C(D(P(t)), P^{'}(t)) \\
\int_{0}^T \pi (P, P^{'}) dt \rightarrow \max_{P(t)} 
\end{align}
где $\int_0^T \pi(P, P^{'}) dt$ - прибыль фирмы с момента 0 по момент $T$ 


Пример задачи (макроэкономика). Типовая задача домохозяйств в современных макроэкономических моделях, заключающаяся в максимизации дисконтировонной полезности с момента 0 по момент T. 
\begin{align}
\int_0^T U(C(T)) \cdot e^{-\delta t} dt \rightarrow \max \\
Y = C +  I \\
K^{'} = I \\
Y = F(K, L) \\
C = Y - K^{'} \\
\int_{0}^T U(F(K, L) - K^{'}) e^{-\delta t} dt \rightarrow \max 
\end{align}
\item Каноническая задача вариационного исчисления 

Как правило, такие задачи сводятся к следующему: найти оптимальную траекторию, чтобы попасть из точки 0 в точку T. (Видимо, это заключается в нахождении оптимальных параметров, функций) 
\begin{align}
V(y) = \int_0^T F(t, y, y^{'}) dt \\
\operatorname{restrictions}: \begin{cases} 
y(0) = y_0 \\
y(T) = y_T 
\end{cases} 
\end{align}
Предположения: интеграл сходится, а $F$ и $y$ непрерывные и дважды дифференцируемые

\item Расстояние между функциями

Расстояние между функциями n-го порядка 
\begin{equation}
\rho_n (y_1 (t), y_2(t)) = \max_{0 \le k \le n} \{ \max_{0 \le t \le T} | y_1^{(k)} (t) - y_2^{(k)} (t) | \}
\end{equation}
Пример. Даны две функции, найти между ними расстояние 0-го порядка. 
\begin{align}
y_1 = t ^ 3 \\
y_2 = t ^ 2 \\ 
t \in [0, 1 ] \\
\rho_0 = \max_{t \in [0,1]} |t ^ 3 - t ^ 2| 
\end{align}
Чтобы найти максимум, будем дифференцировать и проверять значения в крайних точках. 
\begin{align}
2t - 3 t ^ 2 = 0 \\
t = \frac{2}{3} \\
\Big(\frac{2}{3}\Big)^2 - \Big(\frac{2}{3}\Big)^3 = \frac{4}{9} - \frac{8}{27} = \frac{4}{27} \\
\rho_0 = \max \begin{cases} \frac{4}{27}& t = \frac{2}{3}\\ 0 & t = 1 \\ 0 & t = 0 \end{cases}
\end{align}
Теперь найдём расстояние 1-го порядка 
\begin{align} 
\rho_1 = \max \{ \rho_0, \max_{t \in [0,1]} |y_1^{'} - y_2^{'}|\} \\
 \max_{t \in [0,1]} |2t-3t^2| \\
 6 t - 2 = 0 \Rightarrow t = \frac{1}{3} \\
\rho_1 = \max
\begin{cases}
\frac{1}{3} & t = \frac{1}{3} \\
0 & t = 0 \\
1 & t = 1  \\
\rho_0 = \frac{4}{27} & \,\,
\end{cases}
\Rightarrow
\rho_1 = 1
\end{align}
$\varepsilon$-окрестность порядка n для функции $y(t)$ - множество всех функций $y_1(t)$ для которых 
\begin{equation}
\rho_n(y, y_1)<\varepsilon
\end{equation}
n = 0 сильная окрестность $\varepsilon_0$. Задаёт сильный максимум.

n = 1 слабая окрестность $\varepsilon_1 \le \varepsilon_0 $. Задаёт слабый максимум.

$y(t)$ - точка локального максимума если $\forall y_1 \in \varepsilon_0 \\
V(y_1) \le V(y) \Rightarrow$ $y_0$ сильный max и $y_1$ слабый max   

\item Непрерывный функционал

Функционал $V$ называется непрерывным в смысле близости n-го порядка на $y_0(t)$ если 
\begin{align}
\forall \varepsilon > 0:\exists \delta (\varepsilon) > 0: 
\begin{cases}\rho(y, y_0) < \delta \\
|V(y) - V(y_0) | < \varepsilon \end{cases}
\end{align}
\item Приращение функционала 

Пусть $\delta y(t)$ - приращение аргумента. 
\begin{align}
y(t) \rightarrow V(y(t)) \\
y(t) + \delta y(t) \rightarrow V(y + \delta y) \\
\Delta V = V(y + \delta y) - V(y) 
\end{align}
Если приращение функционала $\Delta V$ представимо в виде 
\begin{equation}
\Delta V = L\Big( y(t), \delta y(t) \Big) + \beta (y, \delta y) \max | \delta y) 
\end{equation}
где $L(\cdots)$ линейна по $\delta y$, а $\beta(\cdots) \rightarrow 0$ при $\max | \delta y| \rightarrow 0 $, тогда: 

1. L называется вариацией функционала $ V (\delta V) $

2. $\delta V = \frac{\partial}{\partial \alpha} \big( V(y + \alpha \cdot \delta y) \big)$

\item Задача 1

1 способ решения: 

\begin{align}
V = \int_0^T y^2 dt \\
\Delta V = V (y + \delta y) - V(y) =  \\
= \int_0^T (y + \delta y) ^2 dt - \int_0^T y^2 dt = \\
= \int_0^T y ^ 2 + 2 \delta y y + \delta y^2 - y ^ 2 dt = \\
= \int_0^T 2 \delta y y + \delta y^2 \\
\end{align}
Оставляем только часть, линейную по $\delta y(t)$
\begin{align}
\delta V = \int_0^T 2 y \delta y dt 
\end{align}

2 способ решения:

\begin{align}
V = \int_0^T y^2 dt \\
\delta V = \frac{\partial }{\partial \alpha} \big( \int_0^T (y + \alpha \delta y)^ 2 dt\big)|_{\alpha = 0} = \\
= \int_0^T 2 (y + \alpha \delta y) \delta y dt |_{\alpha = 0} \Rightarrow \\
\Rightarrow \delta V = \int_0^T 2y\delta y dt 
\end{align}
\end{enumerate}
\end{otherlanguage*}
\end{document}
