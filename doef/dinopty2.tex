\documentclass{article}
\usepackage{amsmath}
\usepackage{tikz}
\usepackage{pgfplots}
\usepackage{epsfig}  
\usepackage[T2A,T1]{fontenc}
\usepackage[utf8]{inputenc}
\usepackage[russian,english]{babel}

\begin{document}
\title{\foreignlanguage{russian}{Динамическая оптимизация в экономике и финансах}}
\maketitle

\begin{otherlanguage*}{russian}
\section{\foreignlanguage{russian}{Уравнение Эйлера}}
\begin{enumerate}
\item Теорема

\begin{enumerate}

\item Формулировка: 

Если функционал $ V(y(t)) $, имеющий вариацию достигает $ \max (\min)$ на некоторой кривой $y_0$, то $ \delta V = 0 $ 

\item Доказательство: 

зафиксируем $y_0(t), \delta y(t)$. Тогда пусть $ y(t) = y_0(t) + \delta y \cdot \alpha $ 

$ V(y(t)) = V(y_0 + \delta y \cdot \alpha) = V(\alpha) $ 

Т.е. экстремум достигается при $\alpha = 0 \Rightarrow V^{'} (\alpha)|_{\alpha = 0} = 0 \Rightarrow \delta V = 0 $ 
\end{enumerate}

\item Каноническая задача  
\begin{align} 
\frac{\partial }{\partial \alpha} \Big( \int_{0}^T F(t, y_0 + \alpha \cdot \delta y, y^{'}_0 + \alpha \cdot \delta y^{'})dt\Big)_{\alpha = 0} = 0 \\
\int_0^T F^{'}_y \cdot \delta y + F^{'}_{y^{'}} \cdot \delta y^{'} dt 
\end{align} 
Для решения задачи стоит вспомнить интегрирование по частям: 
\begin{equation}
\int_a^b u dv = uv|^b_a - \int_a^b v du 
\end{equation}
Пусть $ F^{'}_y  = u$, $\delta y^{'} = dv $ 

\begin{align}
\int_0^T \Big( F^{'}_y \cdot \delta y + F^{'}_{y^{'}} \cdot \delta y^{'} \Big) dt = \\
= F^{'}_{y^{'}} \delta y|^T_0 + \int_0^T \Big(F^{'}_y \cdot \delta y - \frac{d}{d t} F^{'}_{y^{'}} \cdot \delta y \Big) dt = \\
\begin{pmatrix} \delta y(0) = 0 & \delta y (T) = 0 \end{pmatrix} = \\
= F^{'}_{y^{'}} |_{t = T} \,\,\delta y(T) - F^{'}_{y^{'}} |_{t = 0} \,\,\delta y(0) + \int_0^T \Big( F^{'}_y - \frac{d}{d t} F^{'}_{y{'}} \Big) \delta y dt = 0 
\end{align}

\item Само уравнение Эйлера

\begin{equation}
F^{'}_y - \frac{d }{d t} F^{'}_{y^{'}} = 0
\end{equation}

Кривые, удовлетворяющиее уравнению Эйлера, называются экстремалями. 

\item Задача 

Найдите экстремали
\begin{enumerate}
\item Условия
\begin{align}
\int_0^2 12 ty + y^{'^{2}} dt \\
\begin{cases}
y(0) = 0 \\ 
y(2) = 8 
\end{cases}
\end{align}
\item Решение

\begin{align} 
\begin{cases}
F^{'}_y = 12 t \\
F^{'}_{y^{'}} = 2 y^{'} \\ 
12 t - 2 y^{''} = 0 \\ 
\end{cases}
\Rightarrow
\begin{cases}
y^{''} = 6t \\
y^{'} = 3t^2 + C_1 \\
y = t^3 + C_1 t + C_2 
\end{cases} \\
y(0) = 0^3 + C_1 \cdot 0 + C_2 = 0 \Rightarrow  C_2 = 0 \\ 
y(2) = 2^3 + 2 \cdot C_1 = 8 \Rightarrow C_1 =0 
\end{align} 
\end{enumerate}
\item Частные случаи уравнения Эйлера 

\begin{enumerate}
\item $ 
F(y, y^{'}) \Rightarrow \begin{cases}\frac{\partial}{\partial} F_{y^{'}} = 0 \\
F_{y^{'}} = 0 \end{cases} $

\item $ F (y, y^{'}) \Rightarrow F_y - F_{y^{'} t} - F_{y^{'} y} - F_{y^{'} y^{'}} = 0 $

\begin{equation}
\frac{d}{dt} (y^{'} F_{y^{'}} - F) = 0 \\ 
y^{'} F_{y^{'}} - F = 0 
\end{equation}
Мораль такая: сначала попробуйте всё-таки вот в такой постановке решить через отдельное уравнение $ F_{y^{'}} (t, y, y^{'}) $, если вдруг вылазиют слоны, то попробовать. 

\item $ F(y^{'}) $ 

\begin{align}
F_{y^{'}} = C \Rightarrow y^{'} = C_1 \Rightarrow y = C_1 \cdot t + C_2 
\end{align}

Пример задачи 
\begin{equation}
\int_0^1 y^{'^{5}} + e^{sin(y^{'})} dt \Rightarrow y = C_1 t + C_2 
\end{equation}

\item $ F(t, y) $

\begin{equation}
F_y = 0 
\end{equation}
Не дифференциальное уравнение $ \Rightarrow $ нет $ \operatorname{const} $ 

Решение существует, только если удовлетворяет граничным условиям $ 
\begin{cases} 
y(0) = y_0 \\ 
y(T) = y_T \\
\end{cases}$ 
\end{enumerate}
\item Сюжет: фракталы и пространства дробной размерности 

Введем размерность по аналогии: = $ \log_{\operatorname{incr}} \operatorname{amt}$ 

Мораль: существуют странные объекты непонятных размерностей помимо гладких и приятных функций. 

\item Обобщения уравнения Эйлера 
\begin{enumerate}
\item Случай нескольких переменных 

Предположим, что задача теперь выглядит вот таким вот образом: 
\begin{equation}
\int_0^T F(t, y_1, y_2, \ldots, y_n, y^{'}_1, \ldots, y^{'}_n) dt \rightarrow (\max | \min) 
\end{equation}
\begin{align}
\begin{cases}
y_1(0) = y_{10} \\
y_1(T) = y_{1T} 
\end{cases} 
\cdots
\begin{cases}
y_n(0) = y_{n0} \\
y_n (T) = y_{nT} 
\end{cases}
\end{align}
То есть решение задачи описывается наобором из n уравнений второго порядка: 
\begin{equation}
i = 1, \ldots, n \,\,\,\, F_{y_i} - \frac{d}{dt} F_{y^{'}_i} = 0 
\end{equation}
Констант будет $ 2 n$. Все нормально. 
Пример задачи: 
\begin{align}
\int_0^T (y + z + y^{'^{2}} + z^{'^{2}}) dt  \\
F_y = 1 \\ 
F_{y^{'}} = 2y^{'} \\ 
1 - \frac{d}{dt} 2y^{'} = 0 \\ 
y^{''} = \frac{1}{2} \\
\begin{cases}
F_z = 1 \\ 
F_{z^{'}} = 2 z^{'} 
\end{cases} \Rightarrow z^{''} = \frac{1}{2}
\end{align}
\item Случай производных более высокого порядка 

Вид задачи:
\begin{equation}
\int_0^T F(t, y, y^{'}, \cdots, y^{(k)}) dt \rightarrow (\max | \min) 
\end{equation}
В этом случае надо пользоваться уравнением Эйлера-Пуассона. 
\begin{equation}
F_y - \frac{d}{dt} F_{y^{'}} + \frac{d^2}{dt^2} F_{y^{''}} - \frac{d^3}{dt^3} F_{y^{'''}} + \ldots + (-1)^k \frac{d^k}{dt^k} F_{y^{(k)}} = 0 
\end{equation}	
Тогда будет дифференциальное уравнение порядка $ 2k \Rightarrow $ нужно $ 2k $ ограничений. 
\begin{align}
y(0) = y_0 \,\,\,\,\,\,\,\,
y(T) = y_T \\ 
y^{'}(0) = y_{1, 0} \,\,\,\,\,\,\,\,
y^{'}(T) = y_{1, T} \\ 
y^{''}(0) = y_{2, 0} \,\,\,\,\,\,\,\,
y^{''}(T) = y_{2, T} \\ 
\ldots \ldots \ldots \ldots \ldots \ldots\ldots \ldots \ldots \\ 
y^{(k-1)} (0) = y_{k-1, 0} \,\,\,\,\,\,\,\, y^{(k-1)} (T) = y_{k-1, T} 
\end{align} 
\item Случай условной задачи. Как в пункте (а), но добавить ограничения. 

Целевая функция:
\begin{equation}
\int_0^T F(t, y_1, y_2, \ldots, y_n, y^{'}_1, \ldots, y^{'}_n) dt \rightarrow (\max | \min) 
\end{equation}
Ограничения (которые выполняются в каждой точке):
\begin{align} 
g_i (t, y_1, y_2, \ldots, y_n, y^{'}_1, \ldots, y^{'}_n) = 0 \\
\end{align} 
Кол-во ограничений: $i = 1, \ldots, m < n $

Вводим двойственную переменную (функцию) к i-му ограничению $ \lambda_i (t) $

Составляем функционал Лагранжа:

\begin{align} 
\mathbf{L} = \int_0^T F dt + \sum_{i=1}^m \int_0^T \lambda_i g_i dt = \\
= \int_0^T \Big( F + \sum_{i=1}^m \lambda_i \cdot g_i \Big) \rightarrow
\end{align} 
Решение обычной задачи с $ n + m $ переменными

Также бывают ограничения, которые должны выполняться целиком за период.
\begin{equation}
\int_0^T G_j (t, y_1, \ldots, y^{'}_n) dt = 0
\end{equation}
Вводим двойственную скалярную переменную $ \Lambda_j$, которая от t не зависит. 
\end{enumerate}
\item Сюжет 2: Динамический хаос 

Там были красивые картинки. 

\item Вариационные задачи с подвижными концами 
\begin{enumerate}
\item Задача с фиксированным конечным временем (задача с вертикальной конечной линией) 
\begin{align}
\int_0^T F(t, y, y^{'}) dt \\ 
y(0) = y_0 \\ 
y(T) = y_T 
\end{align} 
\item Задача с фиксированным состоянием. Задача с горизонтальной линией. Надо найти траектории до горизонтальной линии. 

\item Задача с конечной кривой. Надо найти траектории до кривой линии. 

\item Условие трансверсальности (условие после жизни) 

\begin{equation}
y = y_0 + \alpha \cdot \delta y 
\end{equation}

Пусть есть оптимальный момент остановки

$ T = T^* + \alpha \cdot \Delta T $

$ y_T = y_T^* + \alpha \cdot \Delta y_T$   
\begin{equation} 
V(\alpha) = \int_0^{T^* + \alpha \cdot \Delta T} F(t, y_0 + \alpha \cdot \delta y, y^{'}_0 + \alpha \cdot \delta y^{'}_0 ) dt    
\end{equation}
\begin{align}
\frac{d V(\alpha)}{d(\alpha)} = 0 \\ 
\frac{dv }{d\alpha} = \int_0^{T^* + \alpha \cdot \Delta T } F_y \delta y + F_{y^{'}} \delta y^{'} dt  + \\ +  F (T, y(T), y^{'}(T)) \cdot \Big(T^* + \alpha \cdot \Delta T \Big)^{'}_{\alpha} = \\
\int_0^{T^* + \alpha \cdot \Delta T } F_y \delta y + F_{y^{'}} \delta y^{'} dt  + F (T, y(T), y^{'}(T)) \cdot \delta T = \\  =  \int_0^T (F_y - \frac{d}{dt} F_{y^{'}}) dt  + F_{y^{'}}|_{t = T} \delta y (T) + F (\cdots ) \Delta T = 0 
\end{align}

\item (Картинка) 

Мощнейшая формула:
 
\begin{align}
\Delta y_T = \delta y (T) + y^{'} (T) \cdot \Delta T \\ y = y_0 + \alpha \delta y \\
\delta y (T) = \Delta y_T  - y^{'} (T) \Delta T \\
F_{y^{'}}|_{t=T} \Big( \Delta y_T - y^{'} (T) \Delta T \Big) + F|_{t = T} \Delta T = 0 
\end{align}
В итоге, условие трансверсальности выглядит так: 
\begin{equation}
\Big( F_{y^{'}} \Big)_{t = T}  \Delta y_T + \Big( F - y^{'} F_{y^{'}} \Big)_{t = T} \Delta T = 0  
\end{equation}
\item Случай 1 
$ T = \operatorname{const} $, $ y_t $ - любой 

$ \Delta T = 0 $, $ \Delta y_T $ - любой 

$ \Rightarrow \Big( F_{y^{'}} \Big)_{t = T} = 0 $

\item Случай 2 

$ T $ - любое $\,\,\,\,\,  y_T $ фиксирвано

$\Delta T \,\, \forall \,\,\,\,\, \,\,\,\,\,\Delta y_T = 0 $ 

$ \Big( F - y^{'} F_{y^{'}} \Big) = 0 $ 
\end{enumerate}
\end{enumerate}
\end{otherlanguage*} 
\end{document}