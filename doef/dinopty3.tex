\documentclass{article}
\usepackage{amsmath}
\usepackage{epsfig}  
\usepackage[T2A,T1]{fontenc}
\usepackage[utf8]{inputenc}
\usepackage[russian,english]{babel}

\begin{document}
\title{\foreignlanguage{russian}{Динамическая оптимизация в экономике и финансах}}
\maketitle

\begin{otherlanguage*}{russian}
\section{\foreignlanguage{russian}{Краткое содержание предыдущих серий}}
Каноническая задача 
\begin{align*}
\int_0^T F(t, y, y^{'}) dt \rightarrow \{ \min | \max \} \\
y(0) = y_0 \\
y(T) = y_T \\ 
F_y - \frac{d}{dt} F_{y^{'}} = 0 \\
\ldots \ldots \ldots \\
[F_{y^{'}}]_{t=T} \Delta y_T + [F - y^{'} F_{y^{'}}] \cdot \Delta T = 0 
\end{align*}

Конечное условие: $ y_T = g(T) $

Если выполнено оно, то $ \Delta y_t = g^{'} (T) \cdot \Delta T$

\begin{align*}
[F_{y^{'}}]_{t=T} g^{'} (T) \cdot \Delta T + [F - y^{'} F_{y^{'}}]_{t=T} \cdot \Delta T = 0 \\
[F - y^{'} F_{y^{'}} + g^{'} F_{y^{'}}]_{t=T} = 0
\end{align*}
\section{\foreignlanguage{russian}{Обобщенное условие трансверсальности}}
\begin{enumerate}
\item $ \int_0^T F(t, y, y^{'}, y^{''}, y^{'''} ) dt$

Всегда можно свести к задаче с несколькими переменными

$ z = y^{'} $ 

$ w = y^{''} $ 

Раньше у нас было $ [F - y_1^{'} F_{y_1^{'}} - \cdots - y_n^{'} \cdot F_{y_n^{'}} ] + [F_{y_1^{'}}] \Delta y_{1, T} + \cdots + [F_{y_n^{'}}] \Delta y_{n, T}  $

Как это можно обобщить? Вот у нас есть задача с тремя переменными. 

\begin{align*}
\int_0^T F(t, y_1, y_2, y_3, \ldots) dt \\
y_1 (T) = y_0 \\
\end{align*}

$ y_2 (T), y_3 (T), T$ - свободны

$ y_{1, T} = 0 $ мы зануляем те дельты, которые привязаны к переменным, имеющим фиксировнное значение. 

$ \Delta y_{2, T}, \Delta y_{3, T} \Delta T $ -- любые

\item Задача Больца. 

\begin{align*}
\int_0^T F(t, y, y^{'} ) dt + G(y(0), y(T)) \rightarrow \{ \max | \min \}
\end{align*}
Нет ограничений на $ y(0), \,\,\, y(T) $ 

$ F $ должна быть дважны непрерывно дифференцируема, $ G $ должна быть непрерывно дифференцируема 

Решается в два шага
\begin{enumerate}
\item $ F_y - \frac{d}{dt} F_{y^{'}} = 0 $ 

\item $ F_{y^{'}}|_{t=0} = G_{y(0)} \,\,\,\, F_{y^{'}}|_{t=T} = - G_{y(T)} $ 
\end{enumerate}
\item Как работает задача Больца? 
\begin{align*}
\int_0^1 (y^{'^{2}} - y) dt + y^2 (1)
\end{align*}
Используем уравнение Эйлера:
\begin{align*}
\begin{cases}
F_y = -1\\
F_{y^{'}} = 2 y^{'} \\ 
\end{cases}
-1 - 2 y ^{''} = 0 \\
y^{''} = - \dfrac{1}{2} \\
y^{'} = - \dfrac{1}{2}t + C_1 \\
y= - \dfrac{1}{4} t^2 + C_1 t + C_2 \\
F_{y^{'}} = 2 y^{'} = - t + 2 C_1 \\
y(0): 2 C_1 = 0 \Rightarrow C_1 = 0 \\
y(1): -1 + 2 C_1 = - (2 y(1)) \\
-1 = - 2 (- \dfrac{1}{4} + C_2) \\
C_2 = \frac{3}{4}
\end{align*}
\item Достаточное условие экстремума 

Необходимое условие: $ \dfrac{dv}{d \alpha} = 0 $ 

Достаточное условие: 
\begin{equation*}
\begin{cases}\dfrac{d^2 v}{d \alpha ^ 2 } < 0 \Rightarrow \max \\
\dfrac{d^2 v}{d \alpha^2} > 0 \Rightarrow \min 
\end{cases}
\end{equation*}

Мы говорили, что $ y(t) = y^{*} (t) + \alpha \cdot \delta y $

\begin{align*}
F(t, y, y^{'}) = F(t, y^{*} + \alpha \delta y, y^{*^{'}} + \alpha \cdot \delta y ^{'} ) \\
V = \int_0^T F(t, y, y^{'} ) dt \\
\dfrac{dv}{d \alpha} = \int_0^T F_y \delta y + F_{y^{'}} \delta y ^{'} dt  \\
\dfrac{d^2 v}{d \alpha ^ 2} = \int_0^T F_{y,y} \delta y ^ 2 + F_{y,y^{'}} \delta y \delta y ^{'} + F_{y^{'}, y} \delta y ^{'} \delta y + F_{y^{'}, y^{'}} (\delta y ^{'})^2 ) dt = \\
= \int_0^T F_{y, y} \delta ^ 2 y + 2 F_{y^{'}, y} \delta y \delta y^{'} + F_{y^{'}, y^{'}} \delta y^{'^{2}} dt 
\end{align*}
Т.е. мы получаем бесконечно квадратичных форм 

Если $ \forall t $ сохраняется вывод критерия Сильвестра, то этого достаточно для определения типа экстремума. 

\item Задача на дост. усл. экстремума

\begin{align*}
\int_0^T 4 y ^ 2 + 4 y y^{'} + 3 y^{'^{2}} \rightarrow \operatorname{extr} \\
\begin{pmatrix}
F_{y^{'}, y^{'}} & F_{y^{'}, y} \\
F_{y, y^{'}} & F_{y, y}
\end{pmatrix} \\
F_y = 8 y + 4 y^{'} \\
F_{y^{'}} = 4 y + 6 y ^{'} \\
\begin{pmatrix}
6 & 4 \\
4 & 8 
\end{pmatrix} \\
\begin{cases}
\Delta 6 > 0 \\
\Delta_2 = 6 \cdot 8 - 4 \cdot 4 > 0 \\
\end{cases} \rightarrow \min 
\end{align*}
\item Необходимое условие Лежандра 

Математическая статистика полезна тем, что даёт получить две новые клички для котов: "Муавр" и "Лежандр" 

Если на $ y ^ * $ достигается максимум, то $ F_{y^{'}, y^{'}} \le 0 \,\,\,\forall t \in [0, T] $ 

А если достигается минимум, то $ F_{y^{'}, y^{'}} \ge 0 \,\,\, \forall t \in [0, T] $

\item Введение в теорию геометрических полей 

Рассмотрим плоскость $ (t, y) $ и некоторую область $ D $ в этой плоскости 

Определение: семейство кривых образует собственное поле в области $ D $, если через каждую точку $ (t, y) $ проходит одна и только одна кривая семейства  

$ D: t^2 + y^2 \le 1 $ 

Семейство: $ y = f(t, c) $ 

$ y = t + c $ 

\item Пример

\begin{align*}
y = (t - c) ^ 2 - 1  \\ 
\end{align*}
Здесь через одну точку могут проходить сразу несколько кривых, хотя по идее должна проходить только одна. Такое семейство полем являться не будет. 

\item Если все кривые семейства проходят через некоторую точку $ (t_0, y_0 ) $, то есть образуют пучок кривых покрывают всю область $ D $ и больше нигде ( в этой области)  не пересекаются, то это семейство образует центральное поле

\begin{align*}
y = c \cdot t \\
D_1 : t ^ 2 + y^2 \le 1 \\
\end{align*}
В этой области это семейство не образует поле, потому что есть точки, через котоые не проходит ни одна прямая

А вот здесь образуется собственное поле 
\begin{align*}
D_2 = [1, 2] \cdot [1, 2]
\end{align*}

А если взять $ D_3 = (0, 2) \cdot (0, 2) + (0,0) $, то образует центральное поле. 

\item Задача

\begin{align*}
D: \Big( (0, a] \cdot R \Big) + (0, 0) \\
\begin{cases}
0 < a < \pi \Rightarrow \,\,\, \operatorname{central\,\, field} \\
0 \ge \pi \Rightarrow \,\,\, \operatorname{not \,\, field} \\
\end{cases}
\end{align*}

Если поле образовано семейством экстремалей некоторой задачи $ \Rightarrow $ поле экстремалей, экстремаль $ y(t) $ включена в поле экстремалей, если найдено семейство экстремалей образующих поле $ y = y(t, c) $ содержащее при некотором $ C = C_0 $ рассматриваемую экстремаль, причем она (рассматриваемая экстремаль) не лежит на границе области $ D $ в которой образовано поле. 

\item Пример

\begin{align*}
\int_0^1 y^{'^{2}} - y \,\,\, dt \\
y(0) = 0 \\
y(1) = 0 \\
C_2 = 0; C_1 = \frac{3}{4}
\end{align*}
У нас есть экстремаль.  $ y = \frac{-t^2}{4} + \frac{t}{4} $ 

Рассмотрим следующее семейство: 

\begin{align*}
y = - \frac{t^2}{4} + \frac{t}{4} + C_2 
\end{align*}

$ \forall C_2 $ решение уравнения Эйлера 

Образует собственное поле. 

При $ C_2 = 0 $ это поле включает кривую $ y = - \dfrac{t^2}{4} + \dfrac{t}{4} $ 

\begin{align*}
D: \Big( (0,1 ) \cdot R \Big) + (0,0) + (1, 0) 
\end{align*}
\item Финишный рывок 

Аналитический способ, чтобы найти угрожаемые точки. 

Было: $ y = y(t, c) $ 

Стало: $ F(t, y, c) = 0 $ 

Поработаем с более общим обозначением для функции  

Две бесконечно близкие кривые $ F(t, y, c) = 0 $ пересекаются в точках $ C $-дискриминантной кривой, определяемой уравнениями: 
\begin{align*}
\begin{cases}
F(t, y, c) = 0 \\
\frac{\partial F}{\partial C} = 0 
\end{cases}
\end{align*}
\item Случай номер 1

Рассмотрим семейство вида $ (t - c) ^ 2 + y ^ 2 = 1 $

Что эе такое $C$-дискриминантная кривая для этой окружности 

\begin{align*}
\dfrac{\partial F}{\partial C} = -2 (t - c) = 0 \Rightarrow y^2 = 1 \Rightarrow \begin{cases} y=1 \\ y= - 1 \end{cases}
\end{align*}

\item Случай 2 
\begin{align*}
y = - c t - c ^ 2 \\
\dfrac{\partial F}{\partial C} = -t -2c = 0 \Rightarrow c = - \dfrac{t}{2} \Rightarrow y = \dfrac{t^2}{4} \\
\end{align*}
\item Случай 3

\begin{align*}
y = c(t-1) (t-2) \Rightarrow \begin{cases} t=1 & y = 0 \\ t=2 & y = 0 \end{cases}
\end{align*}
Мораль: $C$-дискриминантная кривая представляет собой либо огибающую, либо набор центральных точек. Она даёт границу, когда с нашими функциями что-то не то происходит 

\item Условие Якоби

Для построения поля экстремалей достаточно, чтобы решение (вариационной задачи) не содержало точку $ A ^ * $ (точку пересечения с $ C$-дискр кривой) 

\item Достаточные условия экстремума функционала

Для того, чтобы $ y(t) $ была решением задачи на слабый минимум (максимум), достаточно: 
\begin{enumerate}
\item Кривая являлась экстремалью, удовлетворяющей граничным условиям 
\item Экстремаль может быть включена в поле экстремалей (предъявить поле или условие Якоби) 

*(если сильно оттягивать, всё становится плохо) 
\item Кривая удовлетворяет усиленному условию Лежандра на $ y(t) $  

\begin{align*}
F_{y^{'}, y^{'}} > 0 \rightarrow \min \\ 
F_{y^{'}, y^{'}} < 0 \rightarrow \max 
\end{align*}
\end{enumerate} 
Для того, чтобы $ y(t) $ была решением задачи на сильный минимум (максимум), достаточно: 
\begin{enumerate}
\item Кривая являлась экстремалью, удовлетворяющей граничным условиям 
\item Экстремаль может быть включена в поле экстремалей (предъявить поле или условие Якоби) 

*(если сильно оттягивать, всё становится плохо) 
\item Кривая удовлетворяет условию Лежандра 

\begin{align*}
F_{y^{'}, y^{'}} \ge 0 \rightarrow \min \\ 
F_{y^{'}, y^{'}} \le 0 \rightarrow \max 
\end{align*}
для точек $ (t, y) $, близких к точкам на исследуемой экстремали и для любых $ y ^{'} $ 	
\end{enumerate}
\item 

\begin{align*}
F_{y^{'}, y^{'}} = y + y^{'} \\
y = t + 1 \,\,\,\, t \in [1, 2] \\
y^{'} = 1 \\
F_{y^{'}, y^{'}} = t +1 + 1 > 0 
\end{align*}
$ \Rightarrow $ слабый минимум есть. А что насчёт сильного? 

$ F_{y^{'}, y^{'}} = y + y ^{'} $ 

Знак выражения зависит от $ y ^{'} \Rightarrow $ условие не выполнено. У нас не хватает доказательств, чтобы обосновать, что экстремум сильный. 

\item $ F_{y^{'}, y^{'}} = y + (y^{'} + 1) ^ 2 > 0 $  

\end{enumerate}
\end{otherlanguage*}
\end{document}
