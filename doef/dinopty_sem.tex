\documentclass{article}
\usepackage{amsmath}
\usepackage{epsfig}  
\usepackage[T2A,T1]{fontenc}
\usepackage[utf8]{inputenc}
\usepackage[russian,english]{babel}

\begin{document}
\title{\foreignlanguage{russian}{Динамическая оптимизация в экономике и финансах}}
\maketitle

\begin{otherlanguage*}{russian}
\section{\foreignlanguage{russian}{Вариацонное исчисление}}
Матанализ ; Варисч

переменная $x$ ; $y(t)$ - траекториия

функция $f(x)$ ; функционал $V[y(t)]$ 

производная $f'(x)$ ; Вариация $\delta V[y(t)]$ - функционал 

$\Delta f(x) $- приращение ; $ \Delta V[y(t)] $

\begin{enumerate}

\item Задача Коши

Условия:
\begin{align}
y^{''}(t) = \frac{1}{2} t \\
y(0) = 0 \\
y(1) = 1 
\end{align}
Решение:
\begin{align}
y^{'} (t) = \frac{t^2}{4} + C_1 \\
y(t) = \frac{t^3}{12} + C_1 t + C_2 \\
\begin{cases}
y(0) = \frac{0^3}{12} + C_1 \cdot 0 + C_2 = 0 \\
y(1) = \frac{1^3}{12} + C_1 \cdot 1 + C-2 = 1 
\end{cases} \Rightarrow
\begin{cases}
C_2 = 0 \\
C_1 = \frac{11}{12}
\end{cases}
\end{align}
Ответ: $y(t) = \frac{t^3}{12} + \frac{11}{12} t + 0 $

\item Уравнение с разделяющимися переменными

Условия:
\begin{align}
y^{'} - y = 2 t - 3 
\end{align}

Идея: все что с $y$ в левую часть. всё что с $t$ - в правую часть

Факты:
\begin{align}
y^{'} (t) = \frac{dy(t)}{dt}
\end{align}

Решение: 

$\frac{dy}{dt} = y + 2t - 3$

Введем обозначение:
\begin{align}
\frac{dy}{dt}= y^{'} = z \\
z = y + 2 t - 3 \\
\frac{dz}{dt} = \frac{d(y + 2 t - 3)}{dt} = z + 2 \\
\frac{1}{z + 2} dz = dt \\
\int \frac{1}{z + 2 } dz = \int dt \\
\ln |z + 2 | = C_1 + t \\
\ln | z + 2 | = \ln C_2 + \ln e^t \\
z + 2 = C_2 \cdot e^t \\
y + 2t - 3 + 2 = C_2 \cdot e ^ t 
\end{align}

Ответ: $y(t) = C_2 \cdot e^t - 2 t + 1 $

\item Однородные линейные уравнения с некратными вещественными корнями

Условия:
$ y^ {'''} - y^{'} = 0 $

Решение:

1. Характеристическое уравнение: $ \lambda ^ 3 - \lambda^1 = 0 $. То есть лямбды, где степень лямбды соответствует порядку производной. 

2.Корни этого характеристического уравнения: $ \lambda_1 = 0; \lambda_2 = 1; \lambda_3 = - 1$. 

3. Ответ: $y(t) = C_1 \cdot e^{0t} + C_2 \cdot e^{1t} + C_3 \cdot e^{-1t} $. 

4. В общем виде: $y(t) = C_1 \cdot e ^ {\lambda_1 t} + \cdots + C_n \cdot e ^ {\lambda_n t} $О
\item Однородные линейные уравнения с кратными вещественными корнями

Условия: 
$y^{'''} - 3 y^{''} + 3 y ^{'} - y = 0 $

1. Характеристическое уравнение: $\lambda^3 - 3 \lambda ^2 + 3 \lambda - 1 = 0 = (\lambda - 1)^3 $

2. Корни этого уравнения: $ \lambda_1 = \lambda_2 = \lambda_3 = 1$

3. Ответ: $y(t) = C_1 \cdot e ^{1 \cdot t} + t \cdot C_2 \cdot e^{1\cdot t} + t ^ 2 \cdot C_2 \cdot e^{1 \cdot t} $.  

4. В общем виде: $ y(t) = C_1 e ^ {\lambda t} + t \cdot C_2 \cdot e^{\lambda t} + t^2 \cdot C_3 \cdot e^{\lambda t} + \cdots + t^{n-1} \cdot C_n \cdot e^{\lambda t} $. Мнемоническое правило: кратность добавляет умножение на t. 

\item Однородные линейные уравнения с некратными комплексными корнями

Условия: 

$ y ^{''} + a^2 y = 0$ 

1. Характеристическое уравнение: $ \lambda^ 2 + a ^ 2 = 0 $ 

2. Корни этого уравнения: $ \lambda_{1,2} = \pm a \cdot i$. Комплексное число задаётся как $ \lambda = \alpha + \beta \cdot i $ 

3. Ответ: $ y(t) = C_1 \cdot \cos(at) + C_2 \cdot \sin(at) $.  

4. В общем виде $ y(t) = C_1 \cdot e ^ {\alpha t} \cdot \cos (\beta t) + C_2 \cdot e^{\alpha t} \cdot \sin(\beta t) + \cdots  C_n \cdot e ^ {\alpha_n t} \cdot \cos (\beta_n t) + C_{n+1} \cdot e^{\alpha_n t} \cdot \sin(\beta_n t)$

Мнемоническое правило: для каждой пары комплексных чисел пишется пара из cos и sin. 
 
\item Однородные линейные уравнения с кратными комплексными корнями

1.Условия: $ \cdots $

2. Корни этого уравнения: $ \lambda_{1, 2} = \pm a \cdot i; \lambda_{3,4} = \pm a \cdot i$

3. Ответ: $ y(t) = C_1 \cos (at) + C_2 \sin (at) + t \cdot \Big( C_3 \cos (at) + C_4 \cdot \sin (at) \Big)$

\item Однородные линейные уравнения со специальной правой частью 

1. Условия: $y^{''} + y = t ^ 2 + t$

2. Решение однородного уравнения: $ y^{''} + y = 0 $

2.1 Характеристическое уравнение: $ \lambda ^ 2 + 1 = 0 $ 

2.2 Корни этого х.у.: $ \lambda_{1, 2} = \pm i $ 

2.3 Ответ: $ y(t) = C_1 \cos t + C_2 \sin t $ 

3. Подбор частного решения. Введём обозначение: $ t^2 + t = f(t) $. Если мы можем представить правую часть в общем виде как $ f(t) = Pm(t) \cdot e^{\lambda t}$. Тогда $ y_2 (t) = Q_m (t) \cdot e ^{\lambda t} \cdot t ^k $

3.1 Попробуем представить правую часть в желаемом виде: $t^2 + t = (t^2 + t) \cdot e ^ {0t} $

3.2 $ y_2(t) = (A t ^ 2 + Bt + C) \cdot e ^{0t} $, поскольку корень х.у. не совпал с 0, то $t^k$ игнорируется. $ k $ - количество корней, которые соответствуют числу над $e$.  

3.3 $ y ^ {'}_2 (t) = 2 At + t$

3.4 $ y^{''}_2 (t) = 2 A  $

3.5 Подставим частное решение в исходное уравнение. 
\begin{align}
2 A + A t ^ 2 + Bt + C = t ^ 2 + t \\
\begin{cases}
2 A + C = 0 & 2 = - 2 \\
A = 1 & B = 1 
\end{cases} \Rightarrow y_2 (t) = t ^ 2 + t - 2 
\end{align}
4. Ответ : $ y(t) = y_{homo} (t) + y_{part} (t) = C_1 \cos t + C_2 \sin t + t ^ 2 + t - 2 $

\item Однородные линейные уравнения со специальной правой частью 

\begin{enumerate}
\item Условие: $y^{(4)} + 2 y ^{''} + y = \sin t $ 

\item Найти y однородное 

Характеритическое уравнение: $ \lambda ^ 4 + 2 \lambda ^ 2 + 1 = (\lambda ^ 2 + 1) ^ 2 = 0 $
Его корни: $\lambda_{1, 2} = \pm i$, $\lambda_{3, 4} = \pm i$ 

Решение однородного уравнения: $ C_1 \cdot \cos t + C_2 \cdot \sin t + t \big( C_3 \cos t + C_4 \sin t \big)$  

\item Как находится частное решение в этом случае

Представим правую часть и т.д.

$ f(t) = \big( P_m(t) \cos \beta t + Q_n(t) \sin \beta t \big) \cdot e ^{\alpha t} $

$ y_2(t) = \big( S_m (t) \cos \beta t + R_n (t) \sin \beta t \cdot e^{\alpha t} + t^k \big)$

\item Найти y частное

$ \sin t = ( 0 \cdot \cos \beta t + 1 \cdot \sin \beta t) \cdot e ^{0 t}$

$ \lambda = 0 \pm 1 \cdot i $

$ y_2 (t) = (A \cos t + B \sin t) \cdot e ^ {0 t} \cdot t ^ 2 $

\item Посчитать ответ как сумму однородного решения + частного. 

Мораль: считайте сами. 

Ответ: $ y(t) = y_{homo} (t) + y_{part} (t) = $
\end{enumerate}
\end{enumerate}
\section{\foreignlanguage{russian}{Вариация функционала}}
Обозначение: $ \delta V [y(t)] $ 

Способы подсчёта:
\begin{enumerate}
\item Способ

\begin{equation}
\delta V [y(t)] = \frac{\partial }{\partial \alpha} V[ y(t) + \alpha \cdot \delta y (t)]|_{\alpha = 0}
\end{equation} 

\begin{equation}
V [y(t)] = \int_a^b y - y^{'^{2}} dt 
\end{equation}
\begin{align}
y(t) \rightarrow y(t) + \alpha \delta y (t) \\
\delta V [ y(t)] = \frac{\partial}{\partial \alpha} \int_a^b y(t) + \alpha \delta y (t) - (y^{'} (t) + \alpha \delta y^{'} (t) ) ^ 2 dt |_{\alpha = 0} = \\
= \int_a^b \delta y (t) - 2 ( y^{'} (t) + \alpha \delta y^{'} (t) ) \cdot \delta y^{'} (t) dt |_{\alpha = 0} =  \\
= \int_a^b \delta y(t) - 2 y^{'}(t) \delta y^{'} (t) dt  
\end{align}

\item Способ 

\begin{equation}
\Delta V [y(t) ] = V [ y(t) + \delta y (t) ] - V [ y(t) ]
\end{equation}

\begin{align}
y(t) \rightarrow y(t) + \delta y(t) \\
\Delta V[y(t)] = \int_a^b y + \delta y - (y^{'} + \delta y ^{'}) ^ 2 dt - \int_a^b y - y^{'^{2}} dt = \\
\int_a^b y + \delta y - (y^{'} + \delta y ^{'}) ^ 2- y + y^{'^{2}} dt = \\
= \int_a^b \delta y - 2 y^{'} \delta y - \delta y^{'^2} dt 
\end{align}

На кончиках пальцев: вариация это когда мы даём линейное приращение и что-то у нас меняется. Нас интересует линейная часть. Вариация линейна по $ \delta y $ и по $ \delta y^{'}$. Фильтруем только линейные штуки. 

Итоговый результат: 

\begin{equation}
\delta V[y(t)] = \int_a^b \delta y - 2 y^{'} \delta y^{'} dt
\end{equation}
\item Расстояние между 2-мя функциями 

Расстояние между функциями - максимум между всеми порядками производных этих функций

\begin{equation}
\rho_n (y_1 (t) , y_2(t)) = \max \{ \max_{0 \le k \le n} | y_1^{(k)}(t) - y_2^{(k)} (t) | \} 
\end{equation}
Пример: 
\begin{align}
y_1(t) = e^t  \\
y_2(t) = t ^ 2 \\
t \in [0, 1]  
\end{align}

Порядок 0: 
\begin{equation}
\max_{0 \le t \le 1} | e^ t - t ^ 2 |
\end{equation}
\begin{align}
e^t - 2t = 0 
\end{align}
На $[0,1 ] $ они не пересекутся. 

\begin{align}
\delta y (0 ) = e^0 - 0 ^ 2 = 1 \\
\delta y (1) = e - 1 = 1.71 \\ 
\rho_0 = \max_{k = 0} \{  1.71 \} = 1.71
\end{align}

Порядок 1: 

\begin{align}
\max_{0 \le t \le 1} |e^t - 2t| \\
e^t - 2 = 0 \\
t^* = \ln 2 = 0.69 \\
\begin{cases}
\delta y(0) = e^ 0 - 2 \cdot 0 = 1 \\
\delta y(1) = e - 2 = 0.71 \\
\delta y(\ln 2) = e^{ln 2} - 2 \ln 2 = 0.69 \\
\end{cases}
\rho_1 = \max_{0 \le k \le 1} \{ 1.71;1 \} = 1.71 
\end{align}

Порядок 2: 

\begin{align}
\max_{0 \le t \le 1} | e ^ t - 2 | \\
e^t \ne 0  \\
\delta y(0) = |e^0 - 2| = 1 \\ 
\delta y(1) = |e^1 - 2| = 0.71 \\
\rho_2 = \max \{ 1.71; 1; 1 \} = 1.71
\end{align}

Порядок 3 и более: 

\begin{align}
\max_{0 \le t \le 1} | e ^ t | \\
\delta y(1) = e^ 1 = e \\
\rho_{n \ge 3}  = \max \{ 1.71; 1; 1; e \} = e
\end{align}

Ответ: 
\begin{equation}
\begin{cases}
\rho_0 = 1.71 \\
\rho_1 = 1.71 \\
\rho_2 = 1.71 \\
\rho_{n \ge 3} = 2.71  
\end{cases}
\end{equation}

Вас 300, а мы не психотерапевты и не психологи. Мы всю эмпатию проявить не сможем, мы где-то сойдём с ума. 

ДЗ Задачки: 
\begin{enumerate}

\item Найти вариацию функционала двумя способами 

\begin{equation}
V [ y(t) ] = y^ 2 (0) + \int_0^1 (ty + y^{'^2} ) dt  \\
\end{equation}
\begin{align}
y(0) + \alpha \delta y (0) \\
y(0) + \delta y (0) 
\end{align}

\item Найти расстояние между функциями 

\begin{align}
y_1 (t) = t ^ 2 - 2 t + 1 \\
y_2(t) = -t ^ 2 + 4 \\
t \in [0;2] 
\end{align}
\end{enumerate}
\end{enumerate}
\section{\foreignlanguage{russian}{Уравнение Эйлера}}
\begin{enumerate}
\item .. 
\begin{align}
\begin{cases}
V [y(t)] \cdot \int_0^T F(t, y(t), y^{'} (t) ) dt \\
y(0) = y_0 \\
y(T) = y_T  
\end{cases} \\
F_y - \frac{d}{dt} F_{y^{'}} = 0 
\end{align}
Как найти допустимую экстремаль 
\begin{align}
\begin{cases}
\int_0^1 t^2 + y^{'^{2}} dt \\
y(0) =0 \\
y(1) = 2 \\
\end{cases} \\
F_y - \frac{d}{dt} F_{y^{'}} = 0 \\
\end{align}
Note: $ F $ -  подынтегральное выражение, $ F_y = F^{'}_y$, $ F_{y^{'}} =  F^{'}_{y^{'}}  $. Находится и подставляется в уравнение Эйлера.   
\begin{align}
F_y = 0 \,\,\, F_{y^{'}} = 2 y^{'} \\
0 = - \frac{d}{dt} (2 y ^{'}) = 0 \Rightarrow y^{''} = 0 \\
y^{'} = C_1 \\
y = t C_1 + C_2 \\ 
y(0) = C_2 = 0; \,\,\, y(1) = C_1 + 0 - 2 \Rightarrow \begin{cases}
C_2 = 0 \\
C_1 = 2 
\end{cases} \Rightarrow y(t) = 2 t \\
\end{align}
\item Случай 1 

\begin{equation}
F(y^{'}, t) \Rightarrow F_y - \frac{d}{dt} F_{y^{'}} = 0 \Rightarrow F_{y^{'}} = C_1 
\end{equation}
Пример: 
\begin{align}
\int_0^1 (t \cdot y ^{'} + y ^{'^{2}} dt = y(0) = y(1) = 1 \\
F_{y^{'}} = C_1 \\
t + 2y^{'} = C_1  \\
y^{'} = \frac{C_1}{2} - \frac{t}{2} \Rightarrow y(t) = \frac{C_1}{2} t - \frac{t^2}{4} + C_2 \\
\begin{cases}
y(0) = C_2 = 1 \\
y(1) = \frac{C_1}{2} - \frac{1}{4} + 1 = 1
\end{cases} \Rightarrow C-1 = \frac{1}{2} \\
y(t) = \frac{t}{4} - \frac{t^2}{4} + 1
\end{align}

\item Случай 2 

\begin{equation}
F(y, y^{'}) 
\end{equation}

Первая история
\begin{align}
y^{'} F_{y^{'}} - F = C (*) 
\end{align}
Вторая история: 
\begin{equation}
F_y - \frac{d}{dt} F_{y^{'}} = 0
\end{equation}
Пример с неберущимся интегралом
\begin{align}
\begin{cases}
\int_0^1 y^2 + y^{'^{2}} dt  \\
y(0) = y(1) = 0 
\end{cases}
y^{'} (2y^{'}) - y ^ 2 - y^{'^{2}} = C \\
y^{'^{2}} - y ^ 2 = C 
\end{align}
В задаче стоит пойти по уравнению Эйлера.
\begin{align}
2y - \frac{d}{dt} 2 y^{'} = 0 \\
y - y^{''} = 0 \\
\end{align}
Это типовое дифференциальное уравнение: однородное ду. 
\begin{align}
\lambda_{1, 2} = \pm 1 \\
y(t) = C_1 \cdot e ^t + C_2 e ^{-t} \\
C_1 = -C_2 = 0 \\
y(t) = 0 
\end{align}

\item Случай 3

\begin{equation}
F(y^{'}) 
\end{equation}

Пример: 
\begin{align}
\int_0^1 y^{'^{5}} + y^{'^{4}} + 8y^{'^{2}} dt \\
y(0) = y(1) = 5 
\end{align}
В этой ситуации можно сразу выписывать решение уравнения Эйлера: 
\begin{align}
y^{'} = C_1 
\end{align}
Почему? Потому что: с
\begin{align}
F_y = \frac{d}{dt} F_{y^{'}} = 0 \\
\frac{d}{dt} F_{y^{'}} = 0 \\
F_{y^{'}} = C_0 \\
5 \cdot y^{'^{4}} + 4 y^{'^{3}} + 16 y ^{'} = C_0 
\end{align}
Проверка, соответствует ли экстремаль, которую мы выведем, нашим ограничениям
\begin{align}
y(t) = C_1 t + C_2 \\
\begin{cases}
y(0) = C_2 = 5 \\
y(1) = C_1 + 5 = 5 \Rightarrow C_1 = 0 
\end{cases} \Rightarrow y(t) = 5 
\end{align}

\item Случай 4 


\begin{equation}
F(y, t) \Rightarrow F_y - \frac{d}{dt} F_{y^{'}} = 0 \Rightarrow F_y = 0 
\end{equation}
Пример. Нам надо подобрать константы, где ограничение выполнено. В примере определенная траектория противоречит ограничению 
\begin{align}
\int_0^1 t ^3 + \sin t + y ^ 2 + 2 y dt \\
y(0) = y(1) = 1 \\
2 y + 2 = 0 \\
y(t) = - 1 
\end{align}

Ответ: нет решений / нет допустимых экстремалей

\item Обобщение уравнения Эйлера (1) Случай нескольких функций

\begin{equation}
F \big( t, y(t), y^{'} (t), z(t) , z^{'} (t) \big)
\end{equation}
Пример. Условия. 
\begin{align}
\begin{cases}
F_y - \frac{d}{dt} F_{y^{'}} = 0 \\
F_z - \frac{d}{dt} F_{z^{'}} = 0 
\end{cases} \\
\int_1^2 y^{'^{2}} + z ^ 2 + z ^{'^{2}} dt \\
y(1) = 1\,\,\, y(2) = 2 \,\,\, z(1) = 0 \,\,\, z(2) = 1 
\end{align}
Пример. Решение
\begin{align}
\begin{cases}
2y^{'} = C_1 \\
2z - \frac{d}{dt} 2 z^{'} = 0 
\end{cases} \Leftrightarrow \begin{cases}
y(t) = C_1 t + C_2 \\
z - z^{''} = 0 
\end{cases} \Rightarrow 
\begin{cases}
y(t) = t \\
z(t) = C_3 \cdot e^t + C_2 \cdot e^{-t} 
\end{cases}
\end{align}
При решении этих уравнений и подстановках получаем экстремали: 
\begin{equation}
\begin{cases}
C_3 = \frac{1}{e^2 - 1} & C_4 = - \frac{e^2}{e^2 - 1} \\
y(t) = t & \, \\
z(t) = \frac{1}{e^2 - 1} e^t - \frac{e^2}{e^2 - 1} \cdot e ^ {-t} & \,
\end{cases}
\end{equation}
\item Обобщение уравнения Эйлера (2) Случай производных высоких порядков 

\begin{align}
F(t, y, y^{'}, y^{''}, \ldots, y^{(n)} ) \\
F_y - \frac{d}{dt} F_{y^{'}} + \frac{d^2}{dt^2} \cdot F_{y^{'''}} - \frac{d^3}{dt^3} + \ldots (-1) ^ n \frac{d^n}{dt^n} F_{y^{(n)}} = 0 
\end{align}
Пример: 
\begin{align}
\int_0^1 360 t^2 y - y^{'''^2} dt \\
\begin{cases}
y(0) = 0 & y(1) = 0 \\ 
y^{'} (0) = 1 & y^{'} (1) = 2.5 
\end{cases}
\end{align}
\begin{align}
F_y - \frac{d}{dt} F_{y^{'}} + \frac{d^2}{dt^2} F_{y^{''}} = 0 \\
F_y = 360 t ^ 2 \\
F_{y^{'}} = 0 \,\,\,\,\, F_{y^{''}} = - 2 y^{''} \\
360 t ^ 2 + \frac{d^2}{dt^2} (-2 y^{''}) = 0 \\
360 t ^2 - 2 y^{(4)} = 0 \\
y^{(4)} = 180 t ^ 2 \\
y^{'''} = 60 t ^3 + C_1 \\
y^{''} = 15 t ^ 4 + C_1 t + C_2 \\
y^{'} = 3t^5 + \frac{C_1}{2} t^2 + C_2 t + C_3 \\
y = \frac{1}{2} \cdot t ^ 6 + \frac{C_1}{6} t^3 + \frac{C_2}{2} t^2 + C_3 t + C_4 
\end{align}
Мы не будем сейчас этим заниматься, а выпишем сразу же константы. 
\begin{align}
\begin{cases}
C_1 = \frac{3}{2} & C_2 = - 3\\
C_3 = 1 & C_4 = 0 
\end{cases}
\end{align}

\item Для тренировок 

(1) Найти допустимую экстремаль
\begin{align}
\int_0^2 7(y^{'^{3}}) dt \\
y(0) = 9\\
y(2) = 11
\end{align}

(2) Найти допустимую экстремаль 
\begin{align}
\int_0^1 \Big( y + y^{'} y + y^{'} + \frac{1}{2} \cdot y^{'^{2}} \Big) dt \\
y(0) = 2 \\
y(1) = 5  
\end{align}
История актуальна две недели с момента того, как мы её увидели. Как бы фидбек актуален тогда, когда он актуален 
\end{enumerate}
\section{\foreignlanguage{russian}{Простая портфельная задача}}
\begin{enumerate}
\item Предпосылки

$ S(t) $ - остаток на анковском счету 

$ r_s (t) $ - процент на остаток 

$ \dfrac{d}{dt} S(t) = r_s (t) \cdot S(t) - CF(t) $ 

$ CF(T) > 0 \Rightarrow $ снимаем денег 

$ CF(T) < 0 \Rightarrow $ вносим деньги  

$ S(t) > 0 $ - депозит 

$ S(t) < 0 $ - кредит 

Для простоты будем считать, что $ r_s(t) = r_L(t) $ 

$ NPV = \sum_{t=0}^\infty \frac{CF_t}{(1 + \delta)^t}$ 

$ \delta $ - дисконт-фактор $ t \in [0, T] $

Надо вспомнить второй замечательный предел:

\begin{equation*}
\lim_{\alpha \rightarrow 0 } (1 + \alpha) ^ {\dfrac{1}{\alpha}} = e 
\end{equation*}

\begin{align*}
NPV = \int_0^T \dfrac{CF(t)}{(1 + \delta)^t} dt = \int_0^T CF(t) (1 + \delta) ^{- \dfrac{\delta t}{\delta} } dt = \\ = \int_0^T CF(t) e^{-\delta t} dt 
\end{align*}

\item Формулировка задачи

\begin{align*}
\int_0^T CF(t) \cdot e ^{-\delta t} dt \rightarrow \max_{CF(t)} \\
\dfrac{d}{dt} S(t) = r_s (t) \cdot S(t) - CF(t) \\
CF(t) = r_s(t) \cdot S(t) - \dfrac{d}{dt} S(t) \\
\int_0^T \Big( r_s (t) \cdot S(t) - \dfrac{d}{dt} S(t) \Big) \cdot e^{-\delta t} dt \rightarrow \max_{S(t)} 
\end{align*}
\item Решение задачи

Уравнение Эйлера: $ F_s - \dfrac{d}{dt} F_{s^{'}} = 0$

\begin{align*}
F_s = r_s (t) \cdot e ^{-\delta t } \\
F_{S^{'}} = -e^{\delta t} \\
r_s (t) e^{-\delta t} - \dfrac{d}{dt} (- e ^ {-\delta t} ) = 0 \\
r_s (t) \cdot e ^ {-\delta t} - \delta \cdot e ^{-\delta t} = 0 \\
r_s(t) = \delta 
\end{align*} 
Тогда $ CF(t) $ и $ S(t) $ любые.

\begin{equation*}
\begin{cases}
r_s (t) > \delta & S(t) \rightarrow + \infty \\
r_s(t) < \delta & S(t) \rightarrow - \infty 
\end{cases}
\end{equation*}

\item Что будет, если изменить предположения? 

$ \delta $ - была задана как константа, но её можно задать как функцию от времени $ \delta (t) $ 

\item  Тогда задача поменяется с момента выписывания уравнения Эйлера 

\begin{align*}
r_s (t) e^{-\delta (t) t} - \dfrac{d}{dt} ( - e^{- \delta (t) t} ) = 0  \\
r_s (t) \cdot e ^{-\delta (t) t } - \delta (t) \cdot e ^{-\delta (t) t} - t \delta^{'} (t) e^{- \delta (t) t} = 0 \\
r_s (t) = \delta (t) + t \cdot \delta ^{'} (t) 
\end{align*}

\item Следующая модификация в модели: комиссия

$ \operatorname{Co} (t) = \alpha (S^{'} (t) ) ^ \beta \\ $

$ \alpha, \beta $ - параметры 

\begin{align*}
\dfrac{d}{dt} S(t) = r_s(t) \cdot S(t) - CF(t) - \operatorname{Co}(t) \\
\int_0^T (r_s (t) \cdot S(t) - S^{'} (t) - \alpha ( S^{'} (t) ) ^ \beta ) \cdot e ^{-\delta t} dt  \rightarrow \max_{S(t)} \\
F_s - \dfrac{d}{dt} F_{s^{'}} = 0 \\
r_s (t) \cdot e ^{-\delta t} - \dfrac{d}{dt} \Big( -e ^{-\delta t} - \alpha \beta (S^{'} (t) ) ^ {\beta - 1} \cdot e ^ {-\delta t} \Big) = 0 \\
r_s (t) \cdot e ^{ - \delta t} - \delta e ^ {-\delta t} + \alpha \beta ( \beta -1 ) \cdot (S^{'} (t) ) ^{\beta -2} \cdot S ^{''} (t) \cdot e ^{ - \delta t} - \alpha \beta (S^{'} (t) ) ^{\beta - 1} \cdot e ^{-\delta t} \delta = 0 \\
r_s(t) - \delta + \alpha \beta (\beta - 1) (S^{'} (t) ) ^{\beta - 2} \cdot S ^{''} (t) - \alpha \beta (S ^{'} (t) ) ^{\beta - 1} \cdot \delta = 0 
\end{align*}

\item Подставим какие-то параметры

\begin{equation*}
\begin{cases}
r_s (t) \delta = 0.1 \\ 
\alpha = \frac{1}{2} \\
\beta = 2 
\end{cases}
\end{equation*}
Если подставить, то будет следующая история

\begin{align*}
S^{''} (t) - 0.1 S^{'} (t) = 0 
\end{align*}

Решим дифур. 
\begin{align*}
\lambda_1 = 0, \,\,\,\, \lambda_2 = 0.1 \\
S(t) = C_1 \cdot e ^{0 \cdot t } + C_2 \cdot e ^{0.1 \cdot t} \\
S(t) = C_1 + C_2 e ^{0.1 \cdot } \\
S(0) = S_0 \\
S(t) = S_T  
\end{align*}
\item Подставим какие-то другие параметры 

\begin{equation*}
\begin{cases}
r_s= 0.125 \\
\delta = 0.1 \\
\alpha = \dfrac{1}{2} \\
\beta = 2 
\end{cases}
\end{equation*}

\begin{align*}
\ldots \ldots \\ 
0.025 + S^{''} (t) - 0.1 S^{'} (t) = 0 \\
S_0 (t) = C_1 + C_2 e^{0.1 \cdot t} \\ 
S_r (t) = A \cdot e ^{0 \cdot t} \cdot t = A \cdot t \\ 
S^{'}_r (t) = A \\ 
S^{''}_r (t) = 0 \\
0.025 + 0.1 \cdot A = 0 \Rightarrow A = 0.25 \\
S(t) = C_1 + 0.25 \cdot t + C_2 \cdot e ^{0.1 \cdot t} \\
S(0) = S_0 \,\,\,\,\, S(T) = S_T 
\end{align*}

\item Ещё одна модификация

Валютный счёт. Введем переменную $ w(t) $ -- валютный курс

\begin{align*}
\dfrac{d}{dt} S_{\$} = r_{\$} (t) \cdot S_{\$} (t) - CF_{\$} (t)  \\
\int_0^T w(t) \cdot (r_{\$} (t) \cdot S_{\$} (t) - S^{'} (t) ) \cdot e ^{- \delta t} dt \\
\ldots \ldots \ldots 
\end{align*}
Если решить 

$ r_{\$} = \delta + \dfrac{w^{'} (t) }{w(t) }$ 
\end{enumerate}
\section{\foreignlanguage{russian}{Условие трансверсальности}}
\begin{align*}
[ F - y^{'} \cdot F_{y^{'}} ]_{t = T} \cdot \Delta T + [ F_{y^{'}} ]_{t=T} \cdot \Delta y = 0 \\
\begin{cases}
y(0) = y_0 \,\,\,\, y(T) = y_T
\end{cases}
\end{align*}
и либо не знаем $ T $, но знаем $ y_T$ , либо наоброот. 

\begin{enumerate}
\item Задача 1 

\begin{align*}
\int_0^{\dfrac{3}{2} \pi} y^{'^{2}} + 4 y \cos t - 8 y \sin t dt \\
y(0) = 1 \\
y(\dfrac{3}{2} \pi) = y_T  \\
[\ldots ] \cdot \Delta T + [\ldots] \cdot \Delta y_t = 0 \\
\end{align*}
$ \Delta T = 0  $ -- так как он нам известен это константа

$ \Delta y_T $ -- это не константа, оно нам неизвестно.

В таком случае первое слагаемое у условия трансверсальности зануляется 

Перепишем условие трансверсальности: 

\begin{align*}
 [ F_{y^{'}} ]_{t=T} \cdot \Delta y = 0 \Rightarrow [ F_{y^{'}} ]_{t=T} = 0 
\end{align*}

Уравнение Эйлера:  $ 4 \cos t - 8 \sin t - 2 y ^ {''} = 0 $ 

\begin{align*}
y^{''} = 2 \cos t - 4 \sin t \\ 
y^{'} = 2 \sin t + 4 \cos t + C_1 \\ 
y = - 2 \cos t + 4 \sin t + C_1 t +  C_2  
\end{align*}
Проверка ограничений

$ y(0) = - 2 \cos 0 + 4 \sin 0 + C_1 \cdot 0 + C_2 = 1 \Rightarrow C_2 = 3 $

$ [ 2 y^{'} ]_{t =  3/2 \pi} = 0 $ 

$ 2 \cdot (2 \sin t + 4 \cos t + C_1 )_{t = 3 /2 \pi } = 0 $  

$ -4 + 2 C_1 = 0 \Rightarrow C_1 = 2 $ 

Ответ получается, когда мы подставим константы $ C_1, C_2 $ в формулу для $y$

$ y = -2 \cos t + 4 \sin t + 2 t + 3 $ 

\item Задача Больца 

\begin{align*}
\int_1^2 t^2 y^{'^{2}} dt - 2 y(1) + y^2(2) \rightarrow \operatorname{extr} 
\end{align*}
Условие трансверсальности:

\begin{align*}
\begin{cases}
[F_{y^{'}}]_{t=1} = G_{y(1)} \\
[F_{y^{'}} ]_{t=2} = - G_{y(2)} 
\end{cases}
\end{align*}

где $ F $ - это подынтегральное выражение, а $ G $ - терминант, добавка (в данном случае, $ -2y(1) + y^2(2) $ 

Уравнение Эйлера: 

$ 2y^{'} t^2 = C_1  \Rightarrow y^{'} = \dfrac{C_2}{t^2} \Rightarrow y = \dfrac{-C_2}{t} + C_3$ 

давайте теперь решать эту систему

\begin{align*}
\begin{cases} 
[2y^{'} t ^ 2]_{t=1} = G_{y(1)} = -2 \\
[2y^{'} t^2]_{t=2} = -2 y (2) 
\end{cases} 
\end{align*}

* $ G_{y(1)} = \dfrac{\partial G}{\partial y(1)}$ 

Подставляем, в $ t $, поскольку мы имеем уже $ y(t) = - \dfrac{C_2}{t} + C_2 $ и $ y^{'} = \dfrac{C_2}{t^2} $ 

\begin{align*}
\begin{cases}
[2 \cdot \dfrac{C_2}{t^2} \cdot t ^ 2 ]_{t=1} = -2  \\
[2 \cdot \dfrac{C_2}{t^2} \cdot t^2 ]_{t=2} = -2 \cdot \Big( \dfrac{-C_2}{2} + C_3 \Big)
\end{cases} \Rightarrow
\begin{cases}
C_2 = -1 \\
C_3 = \dfrac{1}{2}
\end{cases} 
\end{align*}

Ответ: $ y(t) = \dfrac{1}{t} + \dfrac{1}{2} $ 

\item Достаточные условия. Условия чего? Предварительная подготовка к сюжету. 

Вопрос заключается в следующем: является ли полем семейство кривых?

\begin{enumerate}

\item $ y = c \cdot e ^ t, \,\,\,\, t^ 2 + y ^ 2 \le 1 $ 

Эскпоненты, любую точку достигают, поскольку не пересекаются и т.д., тогда поле образуют.

\item $ y = (c + t ) ^ 2, \,\,\,\, t^ 2+ y^2 \le 1 $ 

Эти параболы пересекаются, поэтому не подходят нам, да ещё и не достигают некоторых точек на окружности, поле не образуют. 

\item $ y = c \cdot t \,\,\,\, t > 0 $ 

На области, которая нам интересна, они не пересекаются 

\end{enumerate}

Собственное поле 
\begin{enumerate}
\item Через любую точку области можно провести кривую семейства 

\item Кривые семейства на этой области не пересекаются 
\end{enumerate}

Центральное поле 
\begin{enumerate}
\item Через любую точку области можно провести кривую семейства 

\item  Все кривые семейства пересекаются в некотором центре и больше нигде. 
\end{enumerate}

Достаточные условия экстремали 

$ y(t) $ -- слабый $ \min ( \max ) $ если $ y(t) $:
\begin{enumerate}
\item Является решением уравнения Эйлера и удовлетворяет ограничению 

\item $ y(t) $ можно включить в поле экстремалей 

\item Удовлетворяет условию Лежандра (усиленному) 

$ \begin{cases} 
F_{y^{'}, y^{'}} > 0 \rightarrow \min \\
F_{y^{'}, y^{'}} < 0 \rightarrow \max \\
\end{cases}
$ на рассматриваемой $ y(t) $ 
\end{enumerate}

$ y(t) $ -- сильный $ \min ( \max ) $ если $ y(t) $:
\begin{enumerate}
\item Является решением уравнения Эйлера и удовлетворяет ограничению 

\item $ y(t) $ можно включить в поле экстремалей 

\item Удовлетворяет условию Лежандра (не усиленному) 

$ \begin{cases} 
F_{y^{'}, y^{'}} \le 0 \rightarrow \min \\
F_{y^{'}, y^{'}} \ge 0 \rightarrow \max \\
\end{cases}
$ $\forall y^{'} в некоторой окрестности (t, y) $ 
\end{enumerate}
\end{enumerate}

\begin{align*}
\int_0^2 t \cdot y ^{'} + y ^{'^{2}} dt \\ 
y(0) = 1 \\ 
y(2) = 0 
\end{align*}

Уравнение Эйлера: 

\begin{align*}
F_{y^{'}} = C_1 \\ 
t + 2 y^{'} = C_1 \\ 
y^{'} = \dfrac{C_1}{2} = C_1 \\ 
y^{'} = \dfrac{C_1}{2} - \dfrac{t}{2} \\
y(t) = \dfrac{C_1}{2} t - \dfrac{t^2}{4} + C_2 \\ 
C_1 = 0 \,\,\,\, C_2 = 1 \\
С_1 = 0; \,\,\,\, y(t) = - \dfrac{t^2}{4} + C_2 
\end{align*}

То, что мы предложили $ y(t) = - \dfrac{t^2}{4} + C_2 $ -- это собственное поле, но не центральное. Значит, допустимую экстермаль можно включить в поле экстремалей $y(t) = - \dfrac{t^2}{4} + C_2 $ 

Последнее, что нам остаётся, это проверить усиленное условие Лежандра. 

$ F_{y^{'}, y^{'}} = 2 > 0 \rightarrow \min $ (слабый и сильный тоже)  
\subsubsection*{Задачи из КР}
\begin{enumerate}
\item Задача 6. 
Кто-то за 24 месяца максимизирует приведенную стоимость своих активов
\begin{align*}
\text{Сюжет следующий: } \int_0^{24} CF(t) \cdot e ^{-\delta \cdot t} dt \rightarrow \max \\
\text{Дисконт фактор} : e^{- \delta t} \\
\text{Начальный момент времени} S_{dollar} (0) = 20 dollar \\
\text{Стоимость компании} Q(0) = 3 \cdot 10^6 \text{руб} \\
\text{Курс платины} P(0) = 1500 \text{грам} \\
S_{dollar} (24 ) = 0 \\ 
Q(24) = 5 \cdot 10^6 \text{руб} \\
P(24) = 0 \\
\text{курс доллара  } w_{doll} \\
\text{курс платины  } \rho 
\end{align*}
\begin{align*}
\text{Доходы/убытки от всяких штук  }  w_{doll} (t) \cdot \dfrac{d}{dt } S_{doll} (t) + \dfrac{d}{dt} Q(t) + \rho (t) \dfrac{d}{dt} P(t) =\\
=  w_{doll} \cdot r_{doll} (t) \cdot S_{doll} (t) + g(t) \cdot Q(t) - CF(t) - C_0 (t) 
\end{align*}
Агент несёт издержки в друг формах 
\begin{align*}
& f_1 (x) = \ln x \\
& f_2 (x) = x^{\beta} \\
& \text{Издержки   } Co(t) = w_{doll} (t) \ln S^{'}_{doll} (t) + Q^{'} (t)^\beta 
\end{align*}
Подставляем в интеграл 
\begin{align*}
\int_0^24 (w (t) \cdot r(t) \cdot S(t) + &\\ g(t) \cdot Q(t)  - w(t) \cdot \ln S^{'} (t) - Q^{'} (t) ^ \beta - & \\ w(t) \cdot S^{'} (t) - Q^{'} (t) - \rho (t) \cdot P^{'} (t) )\\ \cdot e ^ {- \delta \cdot t} dt \rightarrow \max _{S(t) , Q(t), P(t)} 
\end{align*}
Решаем Эйлера 
\begin{align*}
F_s = w(t) \cdot r(t) \cdot e^{-\delta t} \\
F_{s^{'}} = \Big( - \dfrac{w(t)}{s^{'}(t)} - w(t) \Big) \cdot e ^ {- \delta t} \\
F_Q = g(t) \cdot e^{- \delta t} \\
F_{Q^{'}} = ( - \beta \cdot Q^{'} (t) ^{\beta - 1} - 1 ) \cdot e ^ {- \delta t } \\
F_P = 0 \\
F_{P^{'}} = - \rho (t) \cdot e ^{- \delta t} 
\end{align*}
\end{enumerate}
\subsection*{Принцип максимума Понтрягина}
\begin{enumerate}
\item Формулировка задачи 
\begin{align*}
\int_0^T F(t, y, u ) dt \rightarrow \max 
\end{align*}
\item Максимум ищется так 
\begin{align*}
y^{'} = f(t, y, u) \\
u(t) \in \mathcal{U} \\
y(0) = y_0 
\end{align*}
\item Переменные управления и состояния 
\begin{align*}
&u(t) \rightarrow \text{переменная управления} \\
&y(t) \rightarrow \text{переменная состояния}
\end{align*}
\item Что нас интересует на выходе? При каждом из состояний переменной управления у нас как-то будет реагировать переменная состояния. 
\item Как определить максимум в такой задаче? 
\begin{enumerate}
\item Надо построить так называемый гамильтониан.. $ H = F(t, y, u ) + \lambda \cdot f(t, y, u) $. Где $\lambda(t) $ --- двойственная переменная, а $ f(t, y, u) = y^{'}  $
\item $ \max_{u \in \mathcal{U}} H $ --- найти максимум этого гамильтониана.
\item $ - \dfrac{\partial H}{\partial y} = \lambda^{'} $ --- первая часть канонической системы
\item $ \dfrac{\partial H}{\partial \lambda} = y^{'} $ --- вторая часть канонической системы 
\item  $ \lambda(T) = 0 $  --- пока что берём на веру, но в следующих сериях мы узнаем, откуда эта штука берется. 
\end{enumerate}
\item Задача 1. 
\begin{align*}
\int_0^2 (y - u^2) dt \rightarrow \max \\
y^{'} = u \\
u(t) \in \mathcal{R} \\
y(0) = 0 
\end{align*}

Выражение гамильтониана
\begin{align*}
H = y - u ^ 2 + \lambda \cdot u \rightarrow \max_{u \in \mathcal{R}} \\
\end{align*}

Поиск максимума 
\begin{align*}
\dfrac{\partial H}{\partial u}  = - 2 u + \lambda = 0 \Rightarrow u = \dfrac{\lambda}{2} 
\end{align*}

Каноническмие уравнения 
\begin{align*}
\begin{cases}
- \dfrac{\partial H}{\partial y}  = \lambda^{'} =  -1 \\
\dfrac{\partial H}{\partial \lambda} = y^{'} = u  
\end{cases} 
\end{align*}

Выразим лямбду. Из первого уравнения системы.
\begin{align*}
\lambda^{'} = -1 \Rightarrow \lambda = -t + C_1 \\ 
\lambda(T) = 0 \\
\lambda(2) = - 2 + C_1 = 0 \Rightarrow C_1 = 2 \\
\lambda(t) = -t  + 2 
\end{align*}

Выразим из второго уравнения системы $ y(t) $ 

\begin{align*}
&y^{'} = \lambda /2  = - \dfrac{t}{2}  + 1 \\
&y(t) = - \dfrac{t^2}{4} + t + C_2 \\
&y(0) = 0 \Rightarrow C_2 = 0 \\
&y(t) =  - \dfrac{t^2}{4} + t 
\end{align*}

Итоговый ответ такой :

\begin{align*}
&y(t) = - \dfrac{t^2}{4} + t \\
&u(t) = - \dfrac{t}{2} + 1 \\
&\lambda (t) = - t + 2 
\end{align*}

\item Задача 2. 

Условия:
\begin{align*}
\int_{-1}^2 (3 y - 9 u ) dt \rightarrow \max_{u} \\
y^{'} = 3 u + y \\
y(-1) = 0 \\
u(t) \in [0; 1] 
\end{align*}
Решение:
\begin{align*}
H = 3 y - 9 u + \lambda (3 u + y ) \rightarrow \max_{u \in [0;1] } \\
H = 3 y + \lambda y + u ( 3\lambda - 9 ) \\
u = \begin{cases} 
0 & \lambda < 3 \\
[0; 1] & \lambda = 3 \\
1 & \lambda > 3 
\end{cases} \\
\begin{cases} 
- \dfrac{ \partial H}{\partial y} = \lambda^{'} = - 3 - \lambda \\
\dfrac{\partial H}{\partial \lambda} = y ^{'} = 3 u + y \\
\end{cases} \\
\lambda^{'} + \lambda = - 3 \Rightarrow \lambda (t) = C_1 \cdot e ^{-t} - 3 \\
\lambda (T) = 0 \\
\lambda(2) = C_1 \cdot e ^{-2} - 3 = 0 \\
C_1 = 3 \cdot e ^ 2 \\
\lambda = 3 e ^{2 - t } - 3 \\
t^* \text{находится так: (это точка в котор. у u нюансы происходят) } 3 = 3 \cdot e ^ {2 - t ^* } - 3 \\
e ^{2 - t ^* } = 2 \Rightarrow t^* = 2 - \ln 2 \\
u^* = \begin{cases}
0 & t \in 2 - \ln 2; 2 ] \\
[0; 1]  & t = 2 - \ln 2 \\ 
1 & t \in [-1; 2 \ln 2) 
\end{cases} 
\end{align*}

Разберёмся со вторым уравнением нашей системы. 

$$ \dfrac{\partial H}{\partial \lambda} = y ^{'} = 3 u + y \\$$


\begin{align*}
\text{Первый случай }\\
u^* = 0, \,\,\,  t \in (2 - \ln 2 ; 2 ] \\
y^{'} = y = C_3 \cdot e ^ t 1 \\
\text{Второй случай}\\
u^*  = 1 \,\,\,\, t \in [-1; 2 - \ln 2 ] \\
y^{'} = 3 + y \\
y = C_4 - 3 \\
y = 3 e^{t+1} - 3 \\
y(-1) = 0 \\
y = C_4 \cdot e ^{-1} - 3 = 0 \Rightarrow C_4 = 3 e \\
\text{Третий случай} \\
C_3 e^{t^*} = 3 e^{t^* + 1} - 3 \\
C_3 = \dfrac{3 e ^{t^* + 1}}{e^{t^*}}
\end{align*}

В первом случае мы не можем найти констнату, нам надо приравнять $ y $ из второго и третьего случая.
\begin{align*}
\text{Ответ} 
\text{(1) } \\
y = \dfrac{3 e ^{t^* + 1} - 3 }{e^{t ^*}} \cdot e ^ t \,\,\, t^* = 2 - \ln 2 \\
u(t)  = 0 \\
\text{2} \\
y(t) = 3e ^{t+1} - 3 \\
u = 1 
\end{align*}
\item Задача 3. 
\begin{align*}
\int_0^4 3 y dt \rightarrow \max \\
y^{'} = y + u \\
y(0) = 5 \\
u \in [0;2 ] 
\end{align*}
\begin{align*}
H = 3 y + \lambda (t) \cdot (y + u ) \rightarrow \max_{u \in [0;2]} \\
u = \begin{cases}
2 & \lambda(t) > 0 \\
0 & \lambda (t) < 0 \\
[0; 2] & \lambda (t) = 0 
\end{cases}
\end{align*}
\begin{align*}
\dfrac{- \partial H}{\partial y} = -3 - \lambda(t) = \lambda^{'} (t) \\
\lambda = C_1 \cdot e ^{-t} - 3 \\
\lambda (2) = C_1 \cdot e ^{-4} - 3 = 0 \\
\dfrac{C_1}{e^4} = 3 \Rightarrow C_1 = 3 \cdot e ^ 4 \\ 
\lambda = 3 \cdot e ^{4 - t}  - 3 \\ 
\dfrac{\partial H}{\partial \lambda} = y + u = y ^{'} 
\end{align*}
\begin{align*}
\lambda  = 3 \cdot e ^{4 - t} - 3 = 0 \\
e^{4 - t} = 1 \\
\ln 1 = 4 - t \Rightarrow t = 4 \\
u^* = \begin{cases}
2 & t \in [0; 4) \\
[0; 2] & t = 4 
\end{cases} \\
\text{Если t} \in [0; 4] \\
y^{'} - y = 2 \\
y = C_4 \cdot e^t - 2 \\
y(0) = C_2 - 2 = 5 \Rightarrow C_2 = 7 \\
y = 7 \cdot e ^ t -2  \\
\end{align*}

А дальше давайте поразмышляем. На самом деле, чтобы $ y $ сохранялся непрерывным, ничего кроме 2 там получиться не может. Если в самой точке непонятно, мы смотрим в окрестности, а всегда ведет как двойка и т.д. Т.е. мы хотим, чтобы эти два уравнения совпали. они совпадают только при $ u_1 = 2 $. 

\begin{align*}
\text{Ответ} 
\begin{cases} 
u = 2 \\
y = 7 \cdot e^t - 2
\end{cases}
\end{align*}
\end{enumerate}
\end{otherlanguage*}
\end{document}