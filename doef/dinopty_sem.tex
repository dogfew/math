\documentclass{article}
\usepackage{amsmath}
\usepackage{epsfig}  
\usepackage[T2A,T1]{fontenc}
\usepackage[utf8]{inputenc}
\usepackage[russian,english]{babel}

\begin{document}
\title{\foreignlanguage{russian}{Динамическая оптимизация в экономике и финансах}}
\maketitle

\begin{otherlanguage*}{russian}
\section{\foreignlanguage{russian}{Вариацонное исчисление}}
Матанализ ; Варисч
переменная $x$ ; $y(t)$ - траекториия

функция $f(x)$ ; функционал $V[y(t)]$ 

производная $f'(x)$ ; Вариация $\delta V[y(t)]$ - функционал 

$\Delta f(x) $- приращение ; $ \Delta V[y(t)] $

\begin{enumerate}

\item Задача Коши

Условия:
\begin{align}
y^{''}(t) = \frac{1}{2} t \\
y(0) = 0 \\
y(1) = 1 
\end{align}
Решение:
\begin{align}
y^{'} (t) = \frac{t^2}{4} + C_1 \\
y(t) = \frac{t^3}{12} + C_1 t + C_2 \\
\begin{cases}
y(0) = \frac{0^3}{12} + C_1 \cdot 0 + C_2 = 0 \\
y(1) = \frac{1^3}{12} + C_1 \cdot 1 + C-2 = 1 
\end{cases} \Rightarrow
\begin{cases}
C_2 = 0 \\
C_1 = \frac{11}{12}
\end{cases}
\end{align}
Ответ: $y(t) = \frac{t^3}{12} + \frac{11}{12} t + 0 $

\item Уравнение с разделяющимися переменными

Условия:
\begin{align}
y^{'} - y = 2 t - 3 
\end{align}

Идея: все что с $y$ в левую часть. всё что с $t$ - в правую часть

Факты:
\begin{align}
y^{'} (t) = \frac{dy(t)}{dt}
\end{align}

Решение: 

$\frac{dy}{dt} = y + 2t - 3$

Введем обозначение:
\begin{align}
\frac{dy}{dt}= y^{'} = z \\
z = y + 2 t - 3 \\
\frac{dz}{dt} = \frac{d(y + 2 t - 3)}{dt} = z + 2 \\
\frac{1}{z + 2} dz = dt \\
\int \frac{1}{z + 2 } dz = \int dt \\
\ln |z + 2 | = C_1 + t \\
\ln | z + 2 | = \ln C_2 + \ln e^t \\
z + 2 = C_2 \cdot e^t \\
y + 2t - 3 + 2 = C_2 \cdot e ^ t 
\end{align}

Ответ: $y(t) = C_2 \cdot e^t - 2 t + 1 $

\item Однородные линейные уравнения с некратными вещественными корнями

Условия:
$ y^ {'''} - y^{'} = 0 $

Решение:

1. Характеристическое уравнение: $ \lambda ^ 3 - \lambda^1 = 0 $. То есть лямбды, где степень лямбды соответствует порядку производной. 

2.Корни этого характеристического уравнения: $ \lambda_1 = 0; \lambda_2 = 1; \lambda_3 = - 1$. 

3. Ответ: $y(t) = C_1 \cdot e^{0t} + C_2 \cdot e^{1t} + C_3 \cdot e^{-1t} $. 

4. В общем виде: $y(t) = C_1 \cdot e ^ {\lambda_1 t} + \cdots + C_n \cdot e ^ {\lambda_n t} $О
\item Однородные линейные уравнения с кратными вещественными корнями

Условия: 
$y^{'''} - 3 y^{''} + 3 y ^{'} - y = 0 $

1. Характеристическое уравнение: $\lambda^3 - 3 \lambda ^2 + 3 \lambda - 1 = 0 = (\lambda - 1)^3 $

2. Корни этого уравнения: $ \lambda_1 = \lambda_2 = \lambda_3 = 1$

3. Ответ: $y(t) = C_1 \cdot e ^{1 \cdot t} + t \cdot C_2 \cdot e^{1\cdot t} + t ^ 2 \cdot C_2 \cdot e^{1 \cdot t} $.  

4. В общем виде: $ y(t) = C_1 e ^ {\lambda t} + t \cdot C_2 \cdot e^{\lambda t} + t^2 \cdot C_3 \cdot e^{\lambda t} + \cdots + t^{n-1} \cdot C_n \cdot e^{\lambda t} $. Мнемоническое правило: кратность добавляет умножение на t. 

\item Однородные линейные уравнения с некратными комплексными корнями

Условия: 

$ y ^{''} + a^2 y = 0$ 

1. Характеристическое уравнение: $ \lambda^ 2 + a ^ 2 = 0 $ 

2. Корни этого уравнения: $ \lambda_{1,2} = \pm a \cdot i$. Комплексное число задаётся как $ \lambda = \alpha + \beta \cdot i $ 

3. Ответ: $ y(t) = C_1 \cdot \cos(at) + C_2 \cdot \sin(at) $.  

4. В общем виде $ y(t) = C_1 \cdot e ^ {\alpha t} \cdot \cos (\beta t) + C_2 \cdot e^{\alpha t} \cdot \sin(\beta t) + \cdots  C_n \cdot e ^ {\alpha_n t} \cdot \cos (\beta_n t) + C_{n+1} \cdot e^{\alpha_n t} \cdot \sin(\beta_n t)$

Мнемоническое правило: для каждой пары комплексных чисел пишется пара из cos и sin. 
 
\item Однородные линейные уравнения с кратными комплексными корнями

1.Условия: $ \cdots $

2. Корни этого уравнения: $ \lambda_{1, 2} = \pm a \cdot i; \lambda_{3,4} = \pm a \cdot i$

3. Ответ: $ y(t) = C_1 \cos (at) + C_2 \sin (at) + t \cdot \Big( C_3 \cos (at) + C_4 \cdot \sin (at) \Big)$

\item Однородные линейные уравнения со специальной правой частью 

1. Условия: $y^{''} + y = t ^ 2 + t$

2. Решение однородного уравнения: $ y^{''} + y = 0 $

2.1 Характеристическое уравнение: $ \lambda ^ 2 + 1 = 0 $ 

2.2 Корни этого х.у.: $ \lambda_{1, 2} = \pm i $ 

2.3 Ответ: $ y(t) = C_1 \cos t + C_2 \sin t $ 

3. Подбор частного решения. Введём обозначение: $ t^2 + t = f(t) $. Если мы можем представить правую часть в общем виде как $ f(t) = Pm(t) \cdot e^{\lambda t}$. Тогда $ y_2 (t) = Q_m (t) \cdot e ^{\lambda t} \cdot t ^k $

3.1 Попробуем представить правую часть в желаемом виде: $t^2 + t = (t^2 + t) \cdot e ^ {0t} $

3.2 $ y_2(t) = (A t ^ 2 + Bt + C) \cdot e ^{0t} $, поскольку корень х.у. не совпал с 0, то $t^k$ игнорируется. $ k $ - количество корней, которые соответствуют числу над $e$.  

3.3 $ y ^ {'}_2 (t) = 2 At + t$

3.4 $ y^{''}_2 (t) = 2 A  $

3.5 Подставим частное решение в исходное уравнение. 
\begin{align}
2 A + A t ^ 2 + Bt + C = t ^ 2 + t \\
\begin{cases}
2 A + C = 0 & 2 = - 2 \\
A = 1 & B = 1 
\end{cases} \Rightarrow y_2 (t) = t ^ 2 + t - 2 
\end{align}
4. Ответ : $ y(t) = y_{homo} (t) + y_{part} (t) = C_1 \cos t + C_2 \sin t + t ^ 2 + t - 2 $

\item Однородные линейные уравнения со специальной правой частью 

\begin{enumerate}
\item Условие: $y^{(4)} + 2 y ^{''} + y = \sin t $ 

\item Найти y однородное 

Характеритическое уравнение: $ \lambda ^ 4 + 2 \lambda ^ 2 + 1 = (\lambda ^ 2 + 1) ^ 2 = 0 $
Его корни: $\lambda_{1, 2} = \pm i$, $\lambda_{3, 4} = \pm i$ 

Решение однородного уравнения: $ C_1 \cdot \cos t + C_2 \cdot \sin t + t \big( C_3 \cos t + C_4 \sin t \big)$  

\item Как находится частное решение в этом случае

Представим правую часть и т.д.

$ f(t) = \big( P_m(t) \cos \beta t + Q_n(t) \sin \beta t \big) \cdot e ^{\alpha t} $

$ y_2(t) = \big( S_m (t) \cos \beta t + R_n (t) \sin \beta t \cdot e^{\alpha t} + t^k \big)$

\item Найти y частное

$ \sin t = ( 0 \cdot \cos \beta t + 1 \cdot \sin \beta t) \cdot e ^{0 t}$

$ \lambda = 0 \pm 1 \cdot i $

$ y_2 (t) = (A \cos t + B \sin t) \cdot e ^ {0 t} \cdot t ^ 2 $

\item Посчитать ответ как сумму однородного решения + частного. 

Мораль: считайте сами. 

Ответ: $ y(t) = y_{homo} (t) + y_{part} (t) = $
\end{enumerate}
\end{enumerate}
\section{\foreignlanguage{russian}{Вариация функционала}}
Обозначение: $ \delta V [y(t)] $ 

Способы подсчёта:
\begin{enumerate}
\item Способ

\begin{equation}
\delta V [y(t)] = \frac{\partial }{\partial \alpha} V[ y(t) + \alpha \cdot \delta y (t)]|_{\alpha = 0}
\end{equation} 

\begin{equation}
V [y(t)] = \int_a^b y - y^{'^{2}} dt 
\end{equation}
\begin{align}
y(t) \rightarrow y(t) + \alpha \delta y (t) \\
\delta V [ y(t)] = \frac{\partial}{\partial \alpha} \int_a^b y(t) + \alpha \delta y (t) - (y^{'} (t) + \alpha \delta y^{'} (t) ) ^ 2 dt |_{\alpha = 0} = \\
= \int_a^b \delta y (t) - 2 ( y^{'} (t) + \alpha \delta y^{'} (t) ) \cdot \delta y^{'} (t) dt |_{\alpha = 0} =  \\
= \int_a^b \delta y(t) - 2 y^{'}(t) \delta y^{'} (t) dt  
\end{align}

\item Способ 

\begin{equation}
\Delta V [y(t) ] = V [ y(t) + \delta y (t) ] - V [ y(t) ]
\end{equation}

\begin{align}
y(t) \rightarrow y(t) + \delta y(t) \\
\Delta V[y(t)] = \int_a^b y + \delta y - (y^{'} + \delta y ^{'}) ^ 2 dt - \int_a^b y - y^{'^{2}} dt = \\
\int_a^b y + \delta y - (y^{'} + \delta y ^{'}) ^ 2- y + y^{'^{2}} dt = \\
= \int_a^b \delta y - 2 y^{'} \delta y - \delta y^{'^2} dt 
\end{align}

На кончиках пальцев: вариация это когда мы даём линейное приращение и что-то у нас меняется. Нас интересует линейная часть. Вариация линейна по $ \delta y $ и по $ \delta y^{'}$. Фильтруем только линейные штуки. 

Итоговый результат: 

\begin{equation}
\delta V[y(t)] = \int_a^b \delta y - 2 y^{'} \delta y^{'} dt
\end{equation}
\item Расстояние между 2-мя функциями 

Расстояние между функциями - максимум между всеми порядками производных этих функций

\begin{equation}
\rho_n (y_1 (t) , y_2(t)) = \max \{ \max_{0 \le k \le n} | y_1^{(k)}(t) - y_2^{(k)} (t) | \} 
\end{equation}
Пример: 
\begin{align}
y_1(t) = e^t  \\
y_2(t) = t ^ 2 \\
t \in [0, 1]  
\end{align}

Порядок 0: 
\begin{equation}
\max_{0 \le t \le 1} | e^ t - t ^ 2 |
\end{equation}
\begin{align}
e^t - 2t = 0 
\end{align}
На $[0,1 ] $ они не пересекутся. 

\begin{align}
\delta y (0 ) = e^0 - 0 ^ 2 = 1 \\
\delta y (1) = e - 1 = 1.71 \\ 
\rho_0 = \max_{k = 0} \{  1.71 \} = 1.71
\end{align}

Порядок 1: 

\begin{align}
\max_{0 \le t \le 1} |e^t - 2t| \\
e^t - 2 = 0 \\
t^* = \ln 2 = 0.69 \\
\begin{cases}
\delta y(0) = e^ 0 - 2 \cdot 0 = 1 \\
\delta y(1) = e - 2 = 0.71 \\
\delta y(\ln 2) = e^{ln 2} - 2 \ln 2 = 0.69 \\
\end{cases}
\rho_1 = \max_{0 \le k \le 1} \{ 1.71;1 \} = 1.71 
\end{align}

Порядок 2: 

\begin{align}
\max_{0 \le t \le 1} | e ^ t - 2 | \\
e^t \ne 0  \\
\delta y(0) = |e^0 - 2| = 1 \\ 
\delta y(1) = |e^1 - 2| = 0.71 \\
\rho_2 = \max \{ 1.71; 1; 1 \} = 1.71
\end{align}

Порядок 3 и более: 

\begin{align}
\max_{0 \le t \le 1} | e ^ t | \\
\delta y(1) = e^ 1 = e \\
\rho_{n \ge 3}  = \max \{ 1.71; 1; 1; e \} = e
\end{align}

Ответ: 
\begin{equation}
\begin{cases}
\rho_0 = 1.71 \\
\rho_1 = 1.71 \\
\rho_2 = 1.71 \\
\rho_{n \ge 3} = 2.71  
\end{cases}
\end{equation}

Вас 300, а мы не психотерапевты и не психологи. Мы всю эмпатию проявить не сможем, мы где-то сойдём с ума. 

ДЗ Задачки: 
\begin{enumerate}

\item Найти вариацию функционала двумя способами 

\begin{equation}
V [ y(t) ] = y^ 2 (0) + \int_0^1 (ty + y^{'^2} ) dt  \\
\end{equation}
\begin{align}
y(0) + \alpha \delta y (0) \\
y(0) + \delta y (0) 
\end{align}

\item Найти расстояние между функциями 

\begin{align}
y_1 (t) = t ^ 2 - 2 t + 1 \\
y_2(t) = -t ^ 2 + 4 \\
t \in [0;2] 
\end{align}
\end{enumerate}
\end{enumerate}
\section{\foreignlanguage{russian}{Уравнение Эйлера}}
\begin{enumerate}
\item .. 
\begin{align}
\begin{cases}
V [y(t)] \cdot \int_0^T F(t, y(t), y^{'} (t) ) dt \\
y(0) = y_0 \\
y(T) = y_T  
\end{cases} \\
F_y - \frac{d}{dt} F_{y^{'}} = 0 
\end{align}
Как найти допустимую экстремаль 
\begin{align}
\begin{cases}
\int_0^1 t^2 + y^{'^{2}} dt \\
y(0) =0 \\
y(1) = 2 \\
\end{cases} \\
F_y - \frac{d}{dt} F_{y^{'}} = 0 \\
F_y = 0 \,\,\, F_{y^{'}} = 2 y^{'} \\
0 = - \frac{d}{dt} (2 y ^{'}) = 0 \Rightarrow y^{'} = 0 \\
y^{'} = C_1 \\
y = t C_1 + C_2 \\ 
y(0) = C_2 = 0; \,\,\, y(1) = C_1 + 0 - 2 \Rightarrow \begin{cases}
C_2 = 0 \\
C_1 = 2 
\end{cases} \Rightarrow y(t) = 2 t \\
\end{align}
\item Случай 1 

\begin{equation}
F(y^{'}, t) \Rightarrow F_y - \frac{d}{dt} F_{y^{'}} = 0 \Rightarrow F_{y^{'}} = C_1 
\end{equation}
Пример: 
\begin{align}
\int_0^1 (t \cdot y ^{'} + y ^{'^{2}} dt = y(0) = y(1) = 1 \\
F_{y^{'}} = C_1 \\
t + 2y^{'} = C_1  \\
y^{'} = \frac{C_1}{2} - \frac{t}{2} \Rightarrow y(t) = \frac{C_1}{2} t - \frac{t^2}{4} + C_2 \\
\begin{cases}
y(0) = C_2 = 1 \\
y(1) = \frac{C_1}{2} - \frac{1}{4} + 1 = 1
\end{cases} \Rightarrow C-1 = \frac{1}{2} \\
y(t) = \frac{t}{4} - \frac{t^2}{4} + 1
\end{align}

\item Случай 2 

\begin{equation}
F(y, y^{'}) 
\end{equation}

Первая история
\begin{align}
y^{'} F_{y^{'}} - F = C (*) 
\end{align}
Вторая история: 
\begin{equation}
F_y - \frac{d}{dt} F_{y^{'}} = 0
\end{equation}
Пример с неберущимся интегралом
\begin{align}
\begin{cases}
\int_0^1 y^2 + y^{'^{2}} dt  \\
y(0) = y(1) = 0 
\end{cases}
y^{'} (2y^{'}) - y ^ 2 - y^{'^{2}} = C \\
y^{'^{2}} - y ^ 2 = C 
\end{align}
В задаче стоит пойти по уравнению Эйлера.
\begin{align}
2y - \frac{d}{dt} 2 y^{'} = 0 \\
y - y^{''} = 0 \\
\end{align}
Это типовое дифференциальное уравнение: однородное ду. 
\begin{align}
\lambda_{1, 2} = \pm 1 \\
y(t) = C_1 \cdot e ^t + C_2 e ^{-t} \\
C_1 = -C_2 = 0 \\
y(t) = 0 
\end{align}

\item Случай 3

\begin{equation}
F(y^{'}) 
\end{equation}

Пример: 
\begin{align}
\int_0^1 y^{'^{5}} + y^{'^{4}} + 8y^{'^{2}} dt \\
y(0) = y(1) = 5 
\end{align}
В этой ситуации можно сразу выписывать решение уравнения Эйлера: 
\begin{align}
y^{'} = C_1 
\end{align}
Почему? Потому что: с
\begin{align}
F_y = \frac{d}{dt} F_{y^{'}} = 0 \\
\frac{d}{dt} F_{y^{'}} = 0 \\
F_{y^{'}} = C_0 \\
5 \cdot y^{'^{4}} + 4 y^{'^{3}} + 16 y ^{'} = C_0 
\end{align}
Проверка, соответствует ли экстремаль, которую мы выведем, нашим ограничениям
\begin{align}
y(t) = C_1 t + C_2 \\
\begin{cases}
y(0) = C_2 = 5 \\
y(1) = C_1 + 5 = 5 \Rightarrow C_1 = 0 
\end{cases} \Rightarrow y(t) = 5 
\end{align}

\item Случай 4 


\begin{equation}
F(y, t) \Rightarrow F_y - \frac{d}{dt} F_{y^{'}} = 0 \Rightarrow F_y = 0 
\end{equation}
Пример. Нам надо подобрать константы, где ограничение выполнено. В примере определенная траектория противоречит ограничению 
\begin{align}
\int_0^1 t ^3 + \sin t + y ^ 2 + 2 y dt \\
y(0) = y(1) = 1 \\
2 y + 2 = 0 \\
y(t) = - 1 
\end{align}

Ответ: нет решений / нет допустимых экстремалей

\item Обобщение уравнения Эйлера (1) Случай нескольких функций

\begin{equation}
F \big( t, y(t), y^{'} (t), z(t) , z^{'} (t) \big)
\end{equation}
Пример. Условия. 
\begin{align}
\begin{cases}
F_y - \frac{d}{dt} F_{y^{'}} = 0 \\
F_z - \frac{d}{dt} F_{z^{'}} = 0 
\end{cases} \\
\int_1^2 y^{'^{2}} + z ^ 2 + z ^{'^{2}} dt \\
y(1) = 1\,\,\, y(2) = 2 \,\,\, z(1) = 0 \,\,\, z(2) = 1 
\end{align}
Пример. Решение
\begin{align}
\begin{cases}
2y^{'} = C_1 \\
2z - \frac{d}{dt} 2 z^{'} = 0 
\end{cases} \Leftrightarrow \begin{cases}
y(t) = C_1 t + C_2 \\
z - z^{''} = 0 
\end{cases} \Rightarrow 
\begin{cases}
y(t) = t \\
z(t) = C_3 \cdot e^t + C_2 \cdot e^{-t} 
\end{cases}
\end{align}
При решении этих уравнений и подстановках получаем экстремали: 
\begin{equation}
\begin{cases}
C_3 = \frac{1}{e^2 - 1} & C_4 = - \frac{e^2}{e^2 - 1} \\
y(t) = t & \, \\
z(t) = \frac{1}{e^2 - 1} e^t - \frac{e^2}{e^2 - 1} \cdot e ^ {-t} & \,
\end{cases}
\end{equation}
\item Обобщение уравнения Эйлера (2) Случай производных высоких порядков 

\begin{align}
F(t, y, y^{'}, y^{''}, \ldots, y^{(n)} ) \\
F_y - \frac{d}{dt} F_{y^{'}} + \frac{d^2}{dt^2} \cdot F_{y^{'''}} - \frac{d^3}{dt^3} + \ldots (-1) ^ n \frac{d^n}{dt^n} F_{y^{(n)}} = 0 
\end{align}
Пример: 
\begin{align}
\int_0^1 360 t^2 y - y^{'''^2} dt \\
\begin{cases}
y(0) = 0 & y(1) = 0 \\ 
y^{'} (0) = 1 & y^{'} (1) = 2.5 
\end{cases}
\end{align}
\begin{align}
F_y - \frac{d}{dt} F_{y^{'}} + \frac{d^2}{dt^2} F_{y^{''}} = 0 \\
F_y = 360 t ^ 2 \\
F_{y^{'}} = 0 \,\,\,\,\, F_{y^{''}} = - 2 y^{''} \\
360 t ^ 2 + \frac{d^2}{dt^2} (-2 y^{''}) = 0 \\
360 t ^2 - 2 y^{(4)} = 0 \\
y^{(4)} = 180 t ^ 2 \\
y^{'''} = 60 t ^3 + C_1 \\
y^{''} = 15 t ^ 4 + C_1 t + C_2 \\
y^{'} = 3t^5 + \frac{C_1}{2} t^2 + C_2 t + C_3 \\
y = \frac{1}{2} \cdot t ^ 6 + \frac{C_1}{6} t^3 + \frac{C_2}{2} t^2 + C_3 t + C_4 
\end{align}
Мы не будем сейчас этим заниматься, а выпишем сразу же константы. 
\begin{align}
\begin{cases}
C_1 = \frac{3}{2} & C_2 = - 3\\
C_3 = 1 & C_4 = 0 
\end{cases}
\end{align}

\item Для тренировок 

(1) Найти допустимую экстремаль
\begin{align}
\int_0^2 7(y^{'^{3}}) dt \\
y(0) = 9\\
y(2) = 11
\end{align}

(2) Найти допустимую экстремаль 
\begin{align}
\int_0^1 \Big( y + y^{'} y + y^{'} + \frac{1}{2} \cdot y^{'^{2}} \Big) dt \\
y(0) = 2 \\
y(1) = 5  
\end{align}
История актуальна две недели с момента того, как мы её увидели. Как бы фидбек актуален тогда, когда он актуален 
\end{enumerate}
\end{otherlanguage*}
\end{document}