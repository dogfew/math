\documentclass{article}
\usepackage{amsmath}
\usepackage{tikz}
\usepackage{pgfplots}
\usepackage{epsfig}  
\usepackage[T2A,T1]{fontenc}
\usepackage[utf8]{inputenc}
\usepackage[russian,english]{babel}

\begin{document}
\title{\foreignlanguage{russian}{Макроэкономика 2}}
\maketitle
\begin{otherlanguage*}{russian}
\section{\foreignlanguage{russian}{Межвременной выбор}}
\begin{equation}
C_t = \bar{C} + \operatorname{mpc} \cdot Yd_t
\end{equation}
блин так поспать хочется

Почему потребительские расходы в долгосрочном периоде сглажены? Почему? 
Чето про динамическую задачу оптимального выбора потребителя. Рациональные ожидания. Согласны? Узнали?

Обозначения:

$C_t$ - потребление агрегированного блага в периоде t

$S_t$ - сбережения из трудового дохода в периоде t

$Y_t$ - трудовой доход в периоде t

$Yd_t$ - располагаемый трудовой доход в периоде t

$A_t$ - богатство в периоде t (на начало периода t) 

$r_t$ - реальная рыночная ставка процента в периоде t

$u(C_t)$ - мгновенная полезность домашнего хозяйства от потребления агрегированного блага в периоде t

$u(C_1)$ где $C_1$ - это потребление барана.
\begin{enumerate}
\item Целевая функция полезности домашнего хозяйства, которое живёт $T$ лет:

$ \sum_{i=1}^{T-1} \frac{u(C_t)}{(1 + \rho)^t} \rightarrow \max $

где: $\frac{1}{(1+\rho)^t} = \beta$  - дисконт-фактор

Свойства мгновенной функции полезности:

$u^{'}(C) > 0; u^{''}(C) < 0$

Согласны?
\item Динамическое бюджетное ограничение

$A_{t+1} = (A_t + S_t) (1 + r) $

\item Межвременное бюджетное ограничение

\begin{align}
A_{t+1} = (1 + r) (A_t + S_t)  \\
A_1 = (1 + r) (A_0 + S_0) \rightarrow A_0 = \frac{A_1}{1+r} - S_0 \\
A_2 = (1 + r) (A_1 + S_1) \rightarrow A_1 = \frac{A_2}{1 + r} - S_1 \\
A_3 = (1 + r) (A_2 + S_2) \rightarrow A_2 = \frac{A_3}{1 + r} - S_2
\end{align}
Мораль: 
\begin{equation}
A_t = - \sum_{\tau = 0}^{T-t-1} \frac{S_{t+r}}{(1 + r)^{\tau}} + \frac{A_T}{(1+r)^{T-t}}
\end{equation}

Согласны?
\item Решение задачи домашнего хозяйства
\begin{align}
\sum_{t=1}^{T-1} \frac{u(C_t)}{(1 + \rho)^2} \rightarrow \max_{C_t \ge 0} \,\,\,\, \forall t \in [0; T -1]\\
A_{t+1} = (A_t + Y^d_t - C_t) (1 + r)  \\
\Lambda = \sum_{t=0}^{T-1} \frac{u(C_t)}{(1 + \rho)^t} + \sum_{t=0}^{T-1} \lambda_t \cdot \Big( A_{t+1} - (A_t + Y^d_t - C_t) (1 + r) \Big)
\end{align}
Согласны?
\begin{align}
\frac{\partial \Lambda}{\partial C_t} = \frac{u^{'}(C_t)}{(1 + \rho)^t} + \lambda_t (-1) (1 + r) =  0 \\
\frac{\partial \Lambda}{\partial C_{t+1}} = \frac{u^{'}(C_{t+1})}{(1 + \rho)^{t+1}} + \lambda_{t+1} (-1) (1+r) = 0
\end{align}
Интересный факт: 
\begin{equation}
\frac{\frac{u^{'}(C_t)}{(1 + \rho)^t}}
{\frac{u^{'}(C_{t+1})}{(1 + \rho)^{t+1}}} = \frac{\lambda_t}{\lambda_{t+1}}
\end{equation}
\item Условие отсутствия игр Понци (условие трансверсальности) 
\begin{equation}
\frac{A_t}{(1+r)^{T - t}} = 0
\end{equation}
Предполагается, что $A_T = 0 $ (индивид не оставляет ни наследства, ни долгов после оконч). Тогда: 
\begin{equation}
A_t = - \sum_{\tau = 0}^{T-t-1} \frac{S_{t + \tau}}{(1 + r)^{\tau)}}
\end{equation}
Согласны?
\item Конечный вид межвременного бюджетного ограничения
\begin{equation}
\sum_{t=0}^{T-1} \frac{Y^d_t}{(1 + r)^t} + A_0 = \sum_{t=0}^{T-1} \frac{C_t}{(1+r)^t}
\end{equation}
\item Интересные факты про МБО 
\begin{align}
A_t > 0 \Rightarrow \sum_{\tau = 0}^{T-t-1} \frac{S_{t + \tau}}{(1 + r)^{\tau}} < 0 \\
A_t < 0 \Rightarrow \sum_{\tau = 0}^{T-t-1} \frac{S_{t + \tau}}{(1 + r)^{\tau}} > 0
\end{align}
Т.е. при накоплении богатства в некоторый момент времени t, в будущем можно делать отрицательные сбережения. 

В случае наличия задолженности на момент t, в будущем необходимо делать положительные сбережения. 	

\item Ещё один вариант решения задачи домашнего хозяйства

\begin{equation}
L = \sum_{t = 0}^{T-1} \frac{u(C_t)}{(1 + \rho)^t}  + \lambda \Big( A_0 + \sum_{t=0}^{T-1} \frac{Y^d_t}{(1 + r)^t} - \sum_{t=0}^{T-1} \frac{C_t}{(1 + r)^t}\Big)
\end{equation}
Решение:
\begin{align}
\frac{\partial L}{\partial C_t}: \frac{u^{'} (C_t)}{(1 + \rho) ^ t } - \lambda \cdot \frac{1}{(1 + r)^t} = 0 \\
\frac{\partial L}{\partial C_{t+1}} : \frac{u^{'}(C_{t+1})}{(1 + \rho)^{t+1}} - \lambda \cdot \frac{1}{(1 + r) ^ {t + 1}} = 0 \\
\frac{u^{'}(C_t)}{u^{'}(C_{t+1})} = \frac{(1 + r)}{(1 + \rho)} 
\end{align}
Ещё вариант: 
\begin{align}
\frac{\partial :}{\partial A_{t+1}}: \lambda_t + \lambda_{t+1} (-1) (1 + r) = 0 \Rightarrow \frac{u^{'}(C_t)}{u^{'}(C_{t+1})} = \frac{1 + r }{1 + \rho}
\end{align}
\item Уравнение Эйлера (Правило Рамсея-Кейнса) 

В любом случае, оптимальный выбор домашнего хозяйства описывается уравнением: 
\begin{equation}
\frac{u^{'}(C_t)}{u^{'}(C_{t+1})} = \frac{(1 + r)}{(1 + \rho)} 
\end{equation}
\begin{equation}
\frac{u^{'}(C_t)}{u^{'}(C_{t+1})(1 + \rho)^{-1}} = \frac{1}{(1 + r)^{-1}}
\end{equation}
$(1 + r) ^{-1} $ есть относительная цена потребления в периоде $ t + 1 $ 

Микроэкономическая интерпретация: предельная норма замещения между потреблением в периодах $ t$ и $ t + 1$ должна быть равна отношению цен 

\item Эффект межвременного замещения

Как меняется выбор домашнего хозяйства, если экзогенно меняется ставка процента 

\begin{enumerate}

\item Эффект замещения:
\begin{equation}
r \uparrow \Rightarrow \frac{u^{'}(C_t)}{u^{'}(C_{t+1})} \uparrow \Rightarrow C_t \downarrow, S_t \uparrow, C_{t+1 } \uparrow
\end{equation}

Пояснение: чем выше процентный доход от сбережений в будущем, тем дороже обходится нам текущее потребление (растут альтернативные издержки) 

\item Эффект дохода

Предположим, что $S_t > 0 $, тогда: 
\begin{equation}
r \uparrow \Rightarrow (1 + r) S_t \uparrow \Rightarrow C_t \uparrow, C_{t+1} \uparrow, S_t \downarrow
\end{equation}

Пояснение: рост ставки процента приводит к увеличению капитального дохода от сбережений в периоде $ t + 1 $ что приводит к росту совокупного дохода в течении всей жизни 

\item Свойства функции сбережений

Какой эффект будет доминировать: эффект дохода или эффект замещения? Это зависит от величины внутривременной эластичности замещения (короче, от функции полезности): 
\begin{equation}
\sigma = - \frac{u{'}(c)}{u^{''}(c)c} = \lim_{\tau \rightarrow t} \sigma (c_t, c_{\tau}) 
\end{equation}
\begin{equation}
\sigma(c_t, c_{\tau}) = - \frac{d \ln (c_{\tau} / c_t)}{d \ln MRS (c_t, c_{\tau})} 
\end{equation}
\begin{equation}
\frac{\partial S_t}{\partial r}  = \begin{cases}
> 0 & \operatorname{if} \sigma > 1 \\
 = 0 & \operatorname{if} \sigma = 1 \\
< 0 & \operatorname{if} \sigma < 1
\end{cases}
\end{equation}

\item Динамика потребительских расходов

Если $ r > \rho $, потребление растёт в  течение жизни

Если $ r < \rho $, потребление падает в течение жизни

Если $ r = \rho $, потреблениее постоянно в течение жизни  

\end{enumerate}
\item Межвременное бюджетное ограничение в случае бесконечного временного горизонта 

\begin{align}
A_{t+1} = (A_t + S_t) (1 + r) \\
A_t = - \sum_{\tau = 0 } ^{T - t -1 } \frac{S_{t + \tau}}{(1 + r)^{\tau}} + \frac{A_T}{(1 + r)^{T - t}}\\
\lim_{T \rightarrow \infty} \frac{A_T}{(1 + r)^{T - t}} = 0 \\
\sum_{t = 0}^{\infty} \frac{C_t}{(1 + r)^t} = A_0 + \sum_{t = 0 }^{\infty} \frac{Y^d_t}{(1 + r)^t} + \lim_{T \rightarrow \infty} \frac{A_t}{(1 + r)^T} \rightarrow 0
\end{align}
Короче, кажется, оно выглядит как-то так:
\begin{equation}
\sum_{\tau = 0}^{\infty} \frac{C_{t + \tau}}{(1 + r)^{\tau}} = A_t + \sum_{\tau = 0}^{\infty} \frac{Y^d_{t + \tau}}{(1 + r)^{\tau}} 
\end{equation}
\item Условия отсутствия игр Понци в бесконечном временном периоде 

\begin{equation}
\lim_{T \rightarrow \infty} \frac{A_T}{(1 + r)^{T - t}} = 0 
\end{equation}

\item Ещё обозначения:

$  G_t $ - государственные закупки в периоде $ t$ 

$ T_t$ - чистые налоги в периоде $t$.

$ T_t = T x_t - Tr_t $

$ b_t $ - государственный долг на начало периода $t$
 
$d_t = G_t - T_t$ - первичный дефицит государственного бюджета в периоде $ t $ 

$ r $ - рыночная ставка процента 

\item Динамическое бюджетное ограничение правительства: 

\begin{align}
d_t + b_t = \frac{b_{t+1}}{1 +r} \\
b_{t+1} = (b_t + d_t) (1 + r) = (b_t + G_t - T_t) (1 + r) \\
\cdots \\ 
b_t = - \sum_{\tau = 0}^{T - t - 1} \frac{d_{t + \tau}}{(1 + r) ^{\tau}} + \frac{b_T}{(1 + r)^{T -t}}
\end{align}

\item Условие отсутствия игр Понци для правительства

\begin{equation}
\lim_{T \rightarrow \infty} \frac{b_T}{(1 + r)^{T - t}} = 0 
\end{equation}
Приведенная стоимость государственного долга стремится к нулю на бесконечном промежутке времени;

Государственный долг не должен расти слишком быстро: темп роста не должен превышать ставку процента 

\item Межвременное бюджетное ограничение для правительства

В каждом периоде сумма приведенной стоимости будущих расходов правительства и накопленного госдолга должна равняться сумме приведенной стоимости налоговых сборов
\begin{equation}
b_t > 0 \Rightarrow \sum_{\tau = 0}^{\infty} \frac{d_{t+\tau}}{(1 + r)^{\tau}} < 0 
\end{equation}
Накопленный госдолг на сегодняшний день может быть обеспечен будущими излишками бюджета
\item Консолидированное бюджетное ограничение для правительства и домохозяйств 

\begin{align}
\sum_{\tau = 0}^{\infty} \frac{C_{t + \tau}}{(1 + r)^{\tau}} = \sum_{\tau = 0}^{\infty} \frac{(Y_{t + \tau} - G_{t + \tau})}{(1 + r)^{\tau}} 
\end{align}
\end{enumerate}
\section{\foreignlanguage{russian}{Рикардианская эквивалентность}}
\begin{enumerate}
\item Понятие

Мысль, вложенная в это понятие, такая: способ финансирования государственных закупок не влияет на величину потребительских расходов

\item Абстрактный пример

Пусть в периоде $ t $ чистые налоги сокращаются на $ \Delta T$. Фискальная политика более не устойчива, так как равенство 
\begin{equation}
\sum_{\tau = 0}^{\infty} \frac{G_{t + \tau}}{(1 + r)^{\tau}} + b_t = \sum_{\tau = 0}^{\infty} \frac{T_{t + \tau}}{(1 + r)^{\tau}}
\end{equation}

Способы вернуться на устойчивую траекторию :
\begin{enumerate}
\item Сократить $G_t$ на $\Delta T_t$ 

\item Увеличить $ T_{t + s} $ на $ \Delta T (1 + r)^s $ в периоде $ t + s $

\item Государственный долг даёт возможность перераспределить налоги  
\end{enumerate}
\item Конкретный примеер
\begin{align}
t = 1, 2 \\
Y_1 = 1000 \\ 
Y_2 = 1100 \\
r = 0.1 \\
\end{align}

Для домохозяйства: 
\begin{equation}
C_1 + \frac{C_2}{1 + r}  = (Y_1 - T_1) + \frac{(Y_2 - T_2)}{1 + r}
\end{equation}

Для правительства: 

\begin{equation}
G_1 + \frac{G_2}{1 + r}  = T_1 + \frac{T_2}{1 + r} 
\end{equation}

Первый пример:
\begin{align}
T_1 = 200 \\
T_2 = 0
\end{align}
Государство: $ 200 + \frac{0}{1.1} = 200 $

Д-ства: $ 1000 - 200 + \frac{1100}{1.1} = 1800 $  
\end{enumerate}
Но если $ T_2 = 220, T_1 = 0$, то ничего не меняется (в частности, оптимальный потребительский выбор)

У д-хва $ 1000 + \frac{1100 -200}{1.1} $

Что-то про муультипликатор
\begin{align}
C = \bar{C} + mpc Y_d \\ 
C^* = C(Y - T, r=\operatorname{const}) \\
Y = C + G = C(Y - T) + G \\
dY = C^{'} dY + dG \\ 
dY ( 1 - C^{'}_{Y_d}) = dG  \\
\frac{dY}{dG} = \frac{1}{1 - C^{'}_{Y_d}} = \frac{1}{1 - mpc} > 1 \\
\begin{cases}
C^* = C(Y_t - G_t, Y_{t+1} - G_{t+1}, \ldots, r) \\
Y = C + G 
\end{cases}\\
Y = G + C (Y_t - G_t, Y_{t+1} - G_{t+1}, \ldots, r)  \\
dy = dG + C^{'}_{Y_d} dY - C^{'}_{Y_d} dG \\
(1 - C^{'}_{Y_d}) dY = dG ( 1 - C^{'}_{Y_d}) \\
\frac{dY}{dG} = 1
\end{align}
В результате построение фискальной политики правительства в экономике с "рациональными и вперед смотрящими агентами" будет отличаться от политики в экономике, где агенты такими не являются

\section{\foreignlanguage{russian}{Модель Рамсея}}
\begin{enumerate}
\item Производственный сектор

Рост населения: $ L_{t + 1} = L_t (1 + n) $ 

Предложение труда экзогенно

Функция выпуска: $ Y_t = F(K_t, L_t) $  

Производственная функця в интенсивной форме: 

$ y_t = f(k_t)$, $ y_t = Y_t / L_t$, $ k_t = K_t / L_t$

Агрегированное тождество в двухсекторной экономике: $ Y_t = C_t + I_t $ 

Инвестиции: $I_t = (K_{t+1} - K_t) + \delta K_t $ 

Накопление капитала: $ K_{t + 1} = I_t + K_t ( 1 - \delta) = Y_t - C_t + K_t (1 - \delta) $ 

На душу населения:

\begin{align}
K_{t + 1} = Y_t - C_t + K_t (1 - \delta) \\
\frac{K_{t+1}}{L_t} = \frac{Y_t}{L_t} - \frac{C_t}{L_t} + \frac{K_t}{L_t} (1 - \delta), \\
L_t = \frac{L_{t+1}}{1 + n} \\
(1 + n ) k_{t+1} = f(k_t) - c_t + k_t (1 - \delta) 
\end{align}
Рынки факторов производства являются совершенно конкурентными, следовательно:

\begin{align}
w_t = f(k_t) - k f^{'} (k_t) \\
r_t = f^{'} (k_t) - \delta 
\end{align}

\item Задача социального планировщика:

Максимизация функции общественного благосостояния:
\begin{equation}
\sum_{t = 0}^{\infty} \frac{u(c_t)}{(1 + \rho)^t} \rightarrow \max_{c_t} \\
\end{equation}
\begin{equation}
(1 + n) k_{t+1} = f (k_t) - c_t + k_t (1 - \delta) 
\end{equation}
\item Эндогенные переменные в модели 

$ c, y, k, r, w $ 

\item Экзогенные переменные в модели

$\rho, n, \delta $ 

\item Решение задачи

\begin{align}
\sum_{t = 0}^{\infty} \frac{u(C_t)}{(1 + \rho)^t} \rightarrow \max_{C_t} \\
(1 + n) k_{t+1} = f(k_t) - c_t + k_t (1 - \delta) \\
\Lambda = \sum_{t = 0}^\infty \frac{u(c_t)}{(1 + \rho)^t} + \sum_{t = 0}^{\infty} \lambda_t \Big( f(k_t) - c_t + k_t (1 - \delta) - (1+ n) k_{t + 1} \Big) \\
\frac{\partial \Lambda}{\partial c_t} = 0 \Rightarrow \frac{u^{'}(c_t)}{(1 + \rho)^t} - \lambda_t = 0 \forall t \\ 
\frac{\partial \Lambda}{\partial k_{t+ 1}} = 0 \Rightarrow - \lambda_t (1 + n) + \lambda_{t + 1} (f^{'}(k_{t+1}) + (1 - \delta)) = 0 \forall t \\
\frac{\lambda_t}{\lambda_{t+1}} = \frac{1 + f^{'}(k_{t+1}) - \delta}{1 + n}
\end{align}
\end{enumerate}
\end{otherlanguage*}
\end{document}
