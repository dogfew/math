\documentclass{article}
\usepackage{amsmath}
\usepackage{tikz}
\usepackage{pgfplots}
\usepackage{epsfig}  
\usepackage[T2A,T1]{fontenc}
\usepackage[utf8]{inputenc}
\RequirePackage{amsfonts}
\usepackage[russian,english]{babel}

\begin{document}
\title{\foreignlanguage{russian}{Дзюба Конса}}
\maketitle
\begin{otherlanguage*}{russian}
\section*{Задача. Устойчивость фиксальной политики}
\begin{itemize}
\item Условия 
\begin{align*}
G(t); T(t) \\
\dot D(t) = r(t) \cdot D(t) + G(t) - T(t) \\ 
d(t) = \dfrac{D(t) }{Y(t)}; g_Y > 0 \\ 
r^{'}(\ldots) > 0 \\ 
r^{''} > 0  \\
a = \dfrac{\dot D(t)}{Y(t)} \,\,\, \text{операционный} 
\end{align*}
\item Решение
\begin{align*}
\dot d (t) = \Big( \dfrac{D(t)}{Y(t)}\Big) = \dfrac{\dot D(t) \cdot Y(t) - D(t) \cdot \dot Y(t) }{Y(t)^2}  = \dfrac{\dot D(t)}{Y(t)} - g_y \cdot d(t) = \\
= \dfrac{r(t) \cdot D(t) + G(t) - T(t) }{Y(t)} - g_y \cdot d(t)= r(t) \cdot d(t) + g(t) - \tau(t) - g_Y \cdot d(t) = a - g_Y \cdot d(t)  \\
\dot d(t) = 0 \Rightarrow \begin{cases} 
d^* = \dfrac{a}{g_y} \\ 
\end{cases} 
\end{align*}
\item Небольшая модификация
\begin{align*}
& \text{Первичный дефицит } \dfrac{G(t) - T(t) }{Y(t)} = a \\
& \dot d(t) = r(t) \cdot d(t) + \dfrac{G(t) - T(t) }{Y(t)} - g_Y \cdot d(t) = \\
& \dot d(t) = d(t) \cdot \Big( r(t)  (d(t)) - g_Y \Big) + a \\
& \text{F.O.C.: } \dot d(t)^{'}_{d(t)} = (u \cdot v)^{'} = r(t) - g_Y + d(t) \cdot r^{'} (t)  \rightarrow \text{Рассматриваемая штука, знак зависит от a. } \\
& \text{S.O.C. : } = \dfrac{\partial ^ 2 \dot  d(t)}{\partial d(t) ^ 2}=  r^{'} (t) + r^{'} (t) + d(t) \cdot r^{''} (t)  > 0 \\
& \dot d(t) = r(t) \cdot d(t) +  a - g_Y \cdot d(t) 
\end{align*}
\subsubsection*{Задача}
\begin{align*}
\text{Б.О.: } \sum_{t=0}^\infty \dfrac{T_t}{(1 + r)^t} = \sum_{t=0}^\infty \dfrac{G_t}{(1 + r)^t} + D_0 
\end{align*}
Издержки искажения: издержки, которые растут в связи с тем, что мы выбрали неоптимальную налоговую ставку. 
\begin{align*}
\sum_{t= 0 } ^\infty  \dfrac{Y_t f(\dfrac{T_t}{Y_t})}{(1 + r)^t} \rightarrow \min_{T_t} \\
Y_t = \text{const} = 7000 \\ 
r^* = 0.04 \\
G_t = 500; \,\,\, G_{t+1} = 250 \\
D_0 = 800 \\
800 + \dfrac{500}{(1 + 0.04)^0} + 250 \cdot \sum_{t=1}^\infty \dfrac{1}{(1 + 0.04)^t} = \sum_{t=0}^\infty \dfrac{T_t}{(1 + 0.04)^t} = \\
\text{Сумма бескконечно убывающей геометрической прогрессии: } b_1 = \dfrac{1}{1.04} \\
q_1 = \dfrac{1}{1.04} \Rightarrow S_n  = \dfrac{1}{0.04} = 25 \\
\begin{cases}
1300 + \dfrac{250 \cdot 1}{0.04} = \sum_{t=0}^\infty \dfrac{T_t}{(1.04)^t} \\
\sum_{t=0}^\infty \dfrac{Y_t \cdot \dfrac{T_t^2}{Y_t^2}}{(1 + r)^t} 
\end{cases}  \\
\mathcal{L} = \sum_{t=0}^ \infty \dfrac{(\dfrac{T_t}{Y_t})^2 \cdot Y_t}{(1 + r)^t} + \lambda \Big(\sum_{t=0}^\infty \dfrac{T_t}{(1 + r)^t} - \dfrac{G_t}{(1 + r)^t} - D_0\Big) \\
\dfrac{\partial \mathcal{L}}{\partial T_t} = \dfrac{2 \cdot T_t}{Y_t} \cdot \dfrac{1}{(1 + r)^t} - \dfrac{\lambda_t}{(1 + r)^t} = 0 \Rightarrow \lambda_t = \dfrac{2 T_t}{Y_t} \Rightarrow \\
\Rightarrow \dfrac{T_t}{Y_t}  = \dfrac{\lambda}{2} = \tau (\text{постоянна и не меняется во времени}) \Rightarrow T_t = \tau \cdot Y_t  \\
7550 = Y_t \cdot \tau \sum_{t=0}^\infty \dfrac{1}{(1 + r)^t} \\
7550 = \tau \cdot 7000 \cdot 25 \\
\tau = \dfrac{7550}{7000 \cdot 25}
\end{align*}
\item Держите в голове что отложенная мобилизация может быть сглаживание может быть первый семинар может быть сегодняшняя третья задача может быть. Единственное что вы знаете у вас точно не будет IS LM. 
\end{itemize}

\end{otherlanguage*} 
\end{document}