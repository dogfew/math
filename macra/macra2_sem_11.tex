\documentclass{article}
\usepackage{amsmath}
\usepackage{tikz}
\usepackage{pgfplots}
\usepackage{epsfig}  
\usepackage[T2A,T1]{fontenc}
\usepackage[utf8]{inputenc}
\RequirePackage{amsfonts}
\usepackage[russian,english]{babel}

\begin{document}
\title{\foreignlanguage{russian}{Михайлов, семинар 4 М2}}
\maketitle
\begin{otherlanguage*}{russian}
\subsubsection*{Модель долгового кризиса}
\begin{itemize}
\item Условия

Рассмотрим правительство, которое в момент времени $t_0$ имеет долг, равный $ D =100 $. Текущее сальдо
государственного бюджета нулевое, поэтому правительство прибегает к рефинансированию долга.
Долг имеет валовую доходность $ R $ (то есть сумма к погашению составит$ R \cdot D $) и он должен быть выплачен
полностью в следующем периоде. Если полученных в следующем периоде чистых налогов $ T $  недостаточно
для выплаты долга, правительство объявляет полный дефолт. $ T \sim \mathcal{U}[100, 300] $ 
\item Решение

\begin{align*}
CDF (t) = \begin{cases}
0 & T \le T_{\min} \\
\dfrac{T - T_{\min}}{T_{\max} - T_{\min}}   & T \in (100, 300) \\
1 & T \ge T_{\text{max}}  
\end{cases} 
\end{align*}

EE 1 (а-ля NAC) (условие отсутствия арбитража). 
$ R^e \rightarrow $ ожидаемая доходность 
$ R^d \rightarrow $ сумма долга которую мы должны выплатить
$ \pi \rightarrow $ вероятность дефолта 

\begin{align*}
R^f = 1 \rightarrow	 \text{альтернатива 1 } \\
R^e = (1 - \pi) \cdot R - \pi \cdot 0 = (1 - \pi) \cdot R \rightarrow \text{альтернатива 2 } \\
\text{N.A.C.: } R^f = R^e \rightarrow R^f = R - R \pi \rightarrow D^{\text{risk}} (\text{доходность рисковых активов}) \uparrow \Rightarrow \\  \Rightarrow P \uparrow P^{\text{risk}} \Rightarrow R^{e} \downarrow \\
R \cdot \pi = R - R^f \\
\pi  = 1 - \dfrac{R^f}{R}
\end{align*}
\begin{align*}
T \le RD \rightarrow  \text{дефолт} \\
T > RD \rightarrow \text{нет дефолта} \\
\pi = P (T \le RD) = \begin{cases}
0 & RD \le T_{\min} \\
\dfrac{RD - T_{\min}}{T_{\max} - T_{\min}} & RD \in (T_{\min}, T_{\max} ) \\
1 & RD \ge T_{\max} 
\end{cases} = \\
= 
\begin{cases}
0 & R \le \dfrac{T_{\min}}{D} \\
\dfrac{RD - T_{\min}}{T_{\max} - T_{\min	}} & R \in (\dfrac{T_{\min}}{D}, \dfrac{T_{\max}}{D}) \\
1 & R \ge  \dfrac{T_{\max}}{D}
\end{cases}  \\
EF1: \pi = 1 - \dfrac{R^f}{R} \rightarrow \pi = 1 - \dfrac{1}{R} \\
EF2:\pi =  
\begin{cases} 0 & R \le T_{\min} / D \\
\dfrac{RD - T_{\min}}{T_{\max} - T_{\min}} & R \in (\dfrac{T_{\min}}{D}, \dfrac{T_{\max}}{D} ) \\
1 & R \ge \dfrac{T_{\max}}{D}
\end{cases} \Rightarrow \text{При подстановке: } \\
\pi = \begin{cases}
0 & R \le 1 \\
\dfrac{100 R - 100}{200} & R \in (1, 3) \\
1 & R \ge 3 
\end{cases} \\
\text{Равновесие}: = \begin{cases}
\text{Equilibrium Equation } 1 \\
\text{Equilibrium Equation } 2 
\end{cases} \\
\text{Найдём равеновесия} \\
\begin{cases}
SS_1: R^f = R = 1 \Rightarrow \pi = 0 \\
SS_2: \dfrac{1}{2} R - \dfrac{1}{2} = 1 - \dfrac{1}{R} \Rightarrow R = [2, 1] \\
SS_3: R \rightarrow + \infty, \pi \rightarrow 1 		
\end{cases}
\end{align*}
\end{itemize}
\subsubsection*{Базовая модель динамической несогласованности}\
\begin{itemize}
\item Задача
\begin{align*}
\pi - \pi^e = \dfrac{1}{\beta} \cdot ( u_n - u) \,\,\, , \beta > 0 \\
L_{soc} = \dfrac{1}{2} (u - u^*)^2 + \dfrac{1}{2} \alpha (\pi - \pi^* ) ^  2 \,\,\, , \alpha > 0 \\
u^*, \pi ^* \rightarrow \text{оптимальные уровни безработицы и инфляции} 
\end{align*}
\item Политика правил (Policy rules) 
\begin{align*}
\pi \rightarrow \text{выбрано будет назначено. Люди верят политику} \Rightarrow \pi = \pi^ e \Rightarrow \\
\Rightarrow u = u_n  
\end{align*}
\begin{align*}
L_{soc} = \dfrac{1}{2} (u_n - u^* )^ 2 + \dfrac{1}{2} \alpha (\pi - \pi ^* ) ^ 2 \rightarrow \min_{\pi} \\
L_{soc}^{'} (\pi) = \alpha (\pi - \pi^*) = 0 \Rightarrow \pi = \pi^* \\
L^{''}_{soc} = \alpha > 0 \Rightarrow \min 
\end{align*}
commitment
\begin{align*}
\pi = \pi^* \\
\pi = \pi^e \\
u = u_n \\
\text{commitment} = \dfrac{1}{2} ( u_n - u^* ) ^ 2 
\end{align*}
\item Discretion
\begin{align*}
\pi - \pi^e = \dfrac{1}{\beta} \cdot ( u_n - u) \,\,\, , \beta > 0 \\
L_{soc} = \dfrac{1}{2} (u - u^*)^2 + \dfrac{1}{2} \alpha (\pi - \pi^* ) ^  2 \,\,\, , \alpha > 0 \\
u^*, \pi ^* \rightarrow \text{оптимальные уровни безработицы и инфляции}  \\
\beta ( \pi - \pi^e ) = u_n - u \\
u = u_n - \beta ( \pi - \pi^ e) \\
 L_{\text{soc}} = \dfrac{1}{2} (u_n - \beta (\pi - \pi^e) - u^* ) ^ 2 + \dfrac{1}{2} \alpha (\pi - \pi^e) ^ 2 \rightarrow \min \\
L_{soc}^{'} = (u_n - \beta (\pi - \pi^e ) - u^* ) (- \beta) + \alpha ( \pi - \pi ^* ) = 0 \\
L_{soc}^{''} = \beta ^ 2 + \alpha > 0  \\
- \beta (u_n - u^*) + \beta ^ 2 \pi - \beta ^ 2 \pi^e + \alpha \pi - \alpha \pi^e = 0 \\
( \alpha + \beta^ 2) \pi = \alpha \pi^* + \beta ^ 2 \cdot \pi^e + \beta (u_n - u^*) \\
\pi (\pi^e) = \dfrac{\alpha}{\alpha + \beta^2} \cdot \pi^* + \dfrac{\beta^2}{\alpha + \beta^2} \pi^e + \dfrac{\beta}{\alpha + \beta^2} \cdot (u_n - u^* ) \rightarrow \text{функция реакции, BR} \\
\end{align*}
\item 
\begin{align*}
\mathbb{E} (\pi) = \pi^e = \dfrac{\alpha}{\alpha + \beta^2} \pi^* + \dfrac{\beta^2 }{\alpha + \beta^2} \pi^e + \dfrac{\beta}{\alpha + \beta^2} (u_n - u^*) \\
\mathbb{E} (\mathbb{E} (\pi)) = \mathbb{E} (\pi^e) = \pi^e \\
\dfrac{\alpha}{\alpha + \beta^2} \pi^e = \dfrac{\alpha }{\alpha + \beta^2} \pi^* + \dfrac{\beta}{\alpha + \beta^2} (u_n - u^*) \\
\pi^e  = \pi^* + \dfrac{\beta}{\alpha} (u_n - u^* ) 
\end{align*}
\end{itemize}
\end{otherlanguage*} 
\end{document}