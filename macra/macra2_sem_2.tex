\documentclass{article}
\usepackage{amsmath}
\usepackage{tikz}
\usepackage{pgfplots}
\usepackage{epsfig}  
\usepackage[T2A,T1]{fontenc}
\usepackage[utf8]{inputenc}
\usepackage[russian,english]{babel}

\begin{document}
\title{\foreignlanguage{russian}{Макра-2 Семинар 2}}
\maketitle
\begin{otherlanguage*}{russian}

\begin{enumerate}
\item Задача 1. “Рикардианская эквивалентность (Ricardian Equivalence)”
\begin{enumerate}
\item Условия
\begin{align}
Y_1 - C_1 = S_1 + A_1 \\
\,\,\,\,(S_1 + A_1) (1 + r) = (S_1 + A_1) + S_1 \cdot r \\
S_1 \cdot r = (S_1 + A_1) \cdot r \cdot \tau + \big( (1 - r) \cdot (S_1+ A_1) \cdot r \big)
\end{align}
\begin{equation}
S_1 + S_1 \cdot r (1 - \tau) = S_1 ( 1 + r (1 - \tau)) 
\end{equation}

\item a) межвременное бюджетное ограничение потребителя, связывающее $C_1$ , $C_2$, $Y_1$ , $Y_2$ , r и $\tau$

\begin{align}
\begin{cases}
C_1 + \frac{C_2}{1 + r (1 - \tau)} = A_1 + Y_1 + \frac{Y_2}{1 + r ( 1 - \tau) } & C_1 < A_1 + Y_1 \\
C_1 + \frac{C_2}{1 + r} = A_1 + Y_1 + \frac{Y_2}{1 + r} & \operatorname{else} 
\end{cases}
\end{align}

\item б) $\check{C_1} < Y_1 + A_1 $. Какой налог получит правительство?


$ T_2 = \check{S_1} \cdot r \cdot \tau  = (Y_1 - \check{C_1} + A_1) \cdot r \cdot \tau $ 

\item c) Аккордный налог 

Выводим $ T_1 $ из идеи, что взятие $ T_1 $ или $ T_2 $ должно быть безразлично  
\begin{align}
T_2 = (Y_1 + A_1 - \check{C_1}) \cdot r \cdot \tau \\
T_1 = \frac{T_2}{1 + r} = \frac{(Y_1 + A_1 - \check{C_1}) \cdot r \cdot \tau }{1 + r }
\end{align}
Изменится ли приведенная стоимость реальных располагаемых доходов потребителя? 

\begin{align}
\check{C_1} + \frac{\check{C_2}}{1 + r (1 - \tau)} = A_1 + Y_1 + \frac{Y_2}{1 + r (1 - \tau)} \\
\check{C_2} = (A_1 + Y_1 - \check{C_1}) (1 + r (1 - \tau)) +  Y_2 \\
C_1 + \frac{C_2}{1 + r} = A_1 + Y_1 + \frac{Y_2}{1 + r} - T_1 \\
T_1 = \frac{(Y_1 + A_1 - \check{C_1}) \cdot r \cdot \tau}{1 + r} \\
C_2 = \Big( A_1 + Y_1 - \frac{Y_1 + A_1 - \check{C_1} \cdot r \cdot \tau }{1 _ r} - C_1 \Big) (1 + r) + Y_2
\end{align}
Проверим. Пусть $C_1$ при аккордном налоге равняется $\check{C_1}$

\begin{align}
C_2 = \Big( A_1 + Y - \check{C_1} \Big) (1 + r) - (Y_1 + A_1 - \check{C_1}) \cdot r \cdot \tau + Y_2 \\
C_2 = (A_1 + Y_1 - \check{C_1}) (1 + r (1 - \tau) ) + Y_2 
\end{align}

Почему мы для простоянных величин используем обозначения оценок??? 

\item d) Как отразится на выборе оптимального потребления переход к новой системе налогообложения?

(график 1) 

\pgfmathdeclarefunction{f_1}{1}{%
  \pgfmathparse{3 + -  1 * #1}%
}
\pgfmathdeclarefunction{u}{2}{%
  \pgfmathparse{#2 + 0.2 / #1}%
}
\begin{tikzpicture}
\begin{axis}[
 no markers, 
  domain=0:3, 
  samples=100,
  ymin=0,
  axis lines*=left, 
  xlabel=$C_1$,
  ylabel=$C_2$,
  every axis y label/.style={at=(current axis.above origin),anchor=south},
  every axis x label/.style={at=(current axis.right of origin),anchor=west},
  height=5cm, 
  width=8cm,
  xtick=\empty, 
  ytick=\empty,
  enlargelimits=false, 
  clip=false, 
  axis on top,
  grid = major,
]
\addplot[] coordinates {
(0, 2)
(0.5, 1.5)
(1, 0)};
 \addplot[only marks] coordinates {
(0, 3)
(1.5, 0)};
\addplot[mark = none] coordinates {
( 0.5, 0 )
( 0, 3 )};

\node[below] at (axis cs:0.6, 2.3)  {$ (1 + Y_1) (1 + r (1 - \tau)) + Y_2$};
\node[below] at (axis cs:1.11,  0)  {$ A_1 + Y_1 + \frac{Y_2}{1+r}$}; 
\node[below] at (axis cs:0.5, 0)  {$ A_1 + Y_1$}; 
\end{axis}
\end{tikzpicture}



Note: 

Слуцкий: это когда вам доступен старый набор

Хикс: Касание со старой кривой безразличия   


\begin{align} 
\Delta C_1^{IE} = \Delta C_1^{SE} + \Delta C_1^{TE} \\
\Delta C_2^{IE} = \Delta C_2^{SE} + \Delta C_2^{IE} 
\end{align} 

Мораль: потребителю не стало хуже. 
\end{enumerate}

\item Задача 2. “Различные способы финансирования государственного долга

Условия..
\begin{enumerate}
\item  Финансирование долга за счёт профицита госбюджета
По условию $ \beta = \frac{1}{1 + r}$  
\begin{align}
\beta = \frac{1}{1 + \rho} \\
\begin{cases}
U = \ln C_1 + \beta \ln C_2 \rightarrow \max \\
C_1 + \frac{C_2}{1 + r} = A_1 + Y_1 + \frac{Y_2}{1 + r}
\end{cases} \\
MRS_{1,2} = \frac{\frac{\partial U}{\partial C_1}}{\frac{\partial u}{\partial C_2}} = \frac{P_1}{P_2} \Rightarrow \frac{1 / C_1}{\beta / C_2} = \frac{1}{1 / (1 + r)} \\
\frac{C_2}{C_1} = (1 + r) \cdot \frac{1}{1 + r} = 1 \\ 
\end{align}
Выразим $C_1^*$
\begin{align}
C_2 = C_2 \\ 
C_1 + \frac{C_1}{1 + r} = A_1 + Y_1 + \frac{Y_2}{1 + r } \\
C_1 (\frac{2 + r }{1 + r} ) = Y_1 + \frac{Y_2}{1 + r} \\
C_1^* = \frac{Y_1 (1 + r) + Y_2}{2 + r} \\
S_1^* = Y_1 - C_1^* = \frac{Y_1 (2 + r) - Y_1 ( 1 + r) - Y_2}{2 + r} \\
S_1^* = \frac{Y_1 - Y_2}{2 + r }
\end{align}

Лирическое отступление 

$ S_1^* = \frac{Y_1 - Y_2}{2 + r }$ - эту сумму займёт государство в $t_1$ 

за 1 год накапают проценты. 

$ S_1 (1 + r) = \frac{1 + r }{2 + r} (Y_1 - Y_2) \rightarrow BS$ (Budget Surplus) 

\item Финансирование долга за счёт налогообложения (Debt financing through tax system)

\begin{align}
\begin{cases} 
U = \ln C_1 + \beta \ln C_2 \rightarrow \max_{C_1, C_2 \ge 0 } \\ 
\begin{cases}
C_1 + \frac{C_2}{1 + r(1 - \tau)} = Y_1 + \frac{Y_2}{1 + r (1 - \tau)} & A_1 + Y \ge C_1 \\
C_1 + \frac{C_2}{1 + r} = Y_1 + \frac{Y_2}{1 + r} & A_1 + Y_1 < C_1 
\end{cases}
\end{cases}
\cdots \\
\frac{C_2}{C_1} = \beta (1 + r (1 - \tau)) \\ 
C_2 = \beta (1 + r (1 - \tau )) C_1 
\end{align}
Подставляем $ C_2 $ в БО и решаем всё это
\begin{align}
C_1 + \frac{\beta (1 + r (1 - \tau) ) \cdot C_1}{1 + r (1 - \tau)} = Y_1 + \frac{Y_2}{1 + r (1 - \tau)} \\
C_1 (1 + \beta) = Y_1  + \frac{Y_2}{1 + r  (1 - \tau)} \\
C_1^* = \frac{Y_1 + \frac{Y_2}{1 + r (1 - \tau)}}{1 + \beta} \\
C^*_2 = \frac{\beta (1 + r (1 - \tau))}{1 + \beta} \Big( Y_1 + \frac{Y_2}{1 + r (1 - \tau)}\Big)
\end{align} 
\end{enumerate}
\end{enumerate}
\end{otherlanguage*} 
\end{document}