\documentclass{article}
\usepackage{amsmath}
\usepackage{tikz}
\usepackage{pgfplots}
\usepackage{epsfig}  
\usepackage[T2A,T1]{fontenc}
\usepackage[utf8]{inputenc}
\usepackage[russian,english]{babel}

\begin{document}
\title{\foreignlanguage{russian}{Модель Рамсея}}
\maketitle
\begin{otherlanguage*}{russian}
\begin{enumerate}
\item Равновесие в децентрализованной экономике
\begin{align*}
C_t + S_t = w_t + (r_t + \delta) k_t \\
K_{t+1} = (1 - \delta) K_t + \frac{I_t}{L_t} \,\,\,\, L_{t+1} = (1 + n) L_t \\
(1 + n) k_{t+1} = (1 - \delta) k_r + i_t \\ 
i_t = (1 + n) k_{t+1} - (1 - \delta) k_t \\ 
C_t + S_t = C_t + (1 + n) k_{t+1} - (1 - \delta) k_t = w_t + (r + \delta) k_t \\
(1 + n) k_{t+1} = w_t + (r_t + \delta) k_t + (1 - \delta) k_t - C_t  \\
\begin{cases}
Y_t = w_t + (r_t + \delta) k_t 
\end{cases}
\end{align*}
Выписали бюджетное ограничение, далее -- Лагранж
\begin{align*}
\Lambda = \sum U + \sum \lambda_t \Big( w_t + (r_t + \delta) k_t + (1 - \delta) k_t - C_t - (1+n) k_{t+1} \Big) \rightarrow \max_{C_t, C_{t+1}, \lambda} \\
\frac{u^{'}(C_t)}{u^{'}(C_{t+1})} = \frac{(1 + r_{t+1})}{(1+n) (1 + \rho)} \\
U = \frac{C_t^{1 - \theta} - 1 }{1 - \theta} \\
U^{'}_{C_t} = C_t^{-\theta} \\
\frac{\dot{C(t)}}{C(t)} = \frac{1}{\theta} \Big( r^{'}_{h_{t+1}} - n -\rho - \delta \Big) \\
\pi = P \cdot Y - W_t \cdot L_t - (r_t + \delta) K_t \rightarrow \max_{L_t, K_t} \\
\ldots \\
w_t = f(k_t) - f^{'}_h \cdot k_t \\ 
r_t = f^{'}_h - \delta \rightarrow (r_t + \delta) = f^{'}_h \rightarrow \max_{C_t, C_{t+1}, \lambda_t} \\
(1 + n) h_{t+1} = f(h_t) - f^{'}_h \cdot h_t + f^{'}_h \cdot h_t + (1 - \delta) h_t - C_t \\
(1 + n) = k_{t+1} = f(h_t) + (1 - \delta) h_t - C_t \\
\ldots \\
k(t) = f(h(t)) - C(t) - (n + \delta) k_t 
\end{align*}
\begin{align*}
\begin{cases}
\frac{\dot{C(t)}}{C(t)} = \sigma \Big( f(h_{t+1})^{'} - \delta - n - \rho \Big) = 0 \\
\dot{h(t)} = f(h_t) - C_t - (n + \delta) h_t = 0 
\end{cases}
\end{align*}
Локусы (точки, которые удовлетворяют словию) 
\begin{align}
f(h_{t+1})^{'} = n + \rho + \delta \\
C_t = f(h_t) - (n + \delta) h_t \\
L_{t+1} = (1 + n) L_t 
\end{align}
(график) 
\item Фискальная политика в децентрализованной экономике

В задаче подразумевается, что нет темпов прироста населения. Следовательно $ n = 0 $. 

В итоге $ C_t + S_t = w_t + (r_t + \delta) k_t - \tau_t $, где $\tau_t$ - акордный налог на душу населения. Уравнение Эйлера, записанное ранее, не меняется, потому что $\tau_t$ -- константа, на взятие производной не влияет 

Тогда Локусы поменяются
\begin{align*}
\begin{cases}
\frac{\dot{C(t)}}{C(t)} = \sigma \Big( f(h_{t+1})^{'} - \delta - \rho \Big) = 0 \\
\dot{h(t)} = f(h_t) - C_t - ( \delta) h_t - \tau (t) = 0 
\end{cases} \\
f(h_{t+1})^{'} = \rho + \delta \\
C_t = f(h_t) - (\delta) h_t - \tau(t) \\
L_{t+1} = (1) L_t 
\end{align*}
Картинка. Там парабола, она сдвигается вниз 

Вводится налог на зарплату это НДФЛ. Это будет чуть посложнее. 
\begin{align*}
w_{net} = (1 - \tau_L ) \cdot w_{gross} \\
x_t = w_t \cdot \tau_L 
\end{align*}
Там везде $ w_t \cdot (1 - \tau_L) $ 

\begin{align}
C_t + S_t = w_t (1 - \tau_L) + (r_t + \delta) k_t + x_t \\
x_t = w_t \cdot \tau_L 
\end{align}
Это все пихается в Эйлера. Стоит заметить, что при взятии производной это константы, Эйлер не меняется. 

Только когда мы думаем как Social Planner и думаем о всей экономике, мы можем селать замену как $ x_t = w_t \cdot \tau_L$. И все сократится короче, ресурсное ограничение не изменится. Картинка не изменится. 

В модели Рамсея предполагается, что труд гомогенный. Д/х не решают, сколько им работать, они предлагают весь труд, который есть. Это предпосылка модели, из которой следует, что налог ни на что не влияет. 

Ладно, не будем вас задерживать, последний пункт решим следующий раз.
\end{enumerate}
\end{otherlanguage*} 
\end{document}
