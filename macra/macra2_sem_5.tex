\documentclass{article}
\usepackage{amsmath}
\usepackage{tikz}
\usepackage{pgfplots}
\usepackage{epsfig}  
\usepackage[T2A,T1]{fontenc}
\usepackage[utf8]{inputenc}
\usepackage[russian,english]{babel}

\begin{document}
\title{\foreignlanguage{russian}{Модель Рамсея}}
\maketitle
\begin{otherlanguage*}{russian}
\begin{enumerate}
\item Есть 4 периода. 3 периода работает, 1 на пенсии, но пенсию не получает
\begin{align*}
\begin{pmatrix}
\, & y_t & c_t & s_t & a_t \\
t = 1 & 30 & 45 & -15 & 0  \\
t = 2 & 60 & 45 & 15 &-15 \\
t= 3 & 90 & 45 & 45 & 0  \\
t = 4 & 0 & 45 & -45 & 45  \\
\end{pmatrix}
\end{align*}
\begin{align*}
c_i = \sum_{i=1}^n y_i / n = \bar{	y} \\
C_1 + \dfrac{C_2}{(1+r)} + \dfrac{C_3}{(1+r)^2} + \dfrac{C_4}{(1 + r)^3} = 
y_1 + \dfrac{y_2}{(1+r)} + \dfrac{y_3}{(1+r)^2} + \dfrac{y_4}{(1 + r)^3} + a_1\\ 
\end{align*}
Подставляем
\begin{align*}
r = 0; a_1 = 0 \\
C_1 + C_2 + C_3 + C_4 = y_1 + y_2 + y _3 \Rightarrow c^* = 45  \\
r = 0 \\
\rho > 0 \\
a_t = a_{t-1} + s_{t-1}
\end{align*}
Дальше эта штука перерисовывается. 

\begin{enumerate}

\item Ограничение ликвидности $ \Rightarrow $ нельзя брать кредиты 
 
$ C_1 = 30 $ 

$ C_2 + C_3 + C_4  = 150 \Rightarrow C^* = 50 $ 

\item Если есть начальное богатство, то просто меняется таблица а дльше решается аналогично 
\end{enumerate}

\item Сумма бесконечно убывающей геометрической прогрессии

$ \alpha > 0 $ 

$ b_1 + b_1 \cdot q ^{-\alpha} + \cdots $ 

$ S_{geom} = \dfrac{b_1}{1 - q} $
PIH предполагает что $ r = \rho $

\begin{align*}
PIH: \\
\sum_{\tau = 0} ^ \infty \dfrac{C_{t+\tau}}{(1 +r)^\tau} = A_t + \sum_{\tau = 0}^\infty \dfrac{Y_{1+t}^d}{(1+r)^\tau} \\
r = \rho > 0 \rightarrow \dfrac{u^{'}(C_t)}{u^{'}(C_{t+1})} \\
C \cdot \sum_{\tau=0}^\infty \dfrac{1}{(1+r)^\tau} = A_t + \sum_{\tau = 0}^\infty \dfrac{Y_{1+t}^d}{(1+r)^\tau}  \\
S_{\operatorname{geom}} = \dfrac{1}{1 - 1 /(1+r)} = \dfrac{1+r}{r} \\
\dfrac{1+r}{r} \cdot C = A_t + \sum_{t = 0}^\infty \dfrac{Y^d_{t+\tau}}{(1+r)^\tau} \\
C = \dfrac{r}{1 + r} \cdot \Big( A_t + \sum_{\tau=0}^\infty \dfrac{Y_{t+\tau}^d}{(1+r)^\tau}\Big)
\end{align*}

\begin{align*}
Y_{t+\tau} = Y_t - T_t \\
C = \dfrac{r}{1+r} [ A_t + (Y_t - T_t) + \dfrac{Y_{t+1} - T_{\tau +1}}{(1+r)} + \cdots ] \\
Y^d = Y - T \\
\Delta C = \dfrac{1}{1 + r} \cdot \Big( \Delta A_t + ((\Delta Y_t - \Delta T_t) + \dfrac{\Delta Y_{t+1} - \Delta T_{t+1}}{1 +r} + \cdots \Big) \\
\end{align*}

\begin{enumerate}
\item $ \Delta T < 0 $ 

\begin{align*}
\Delta C = \dfrac{r}{1 + r} (-1) \cdot \dfrac{\Delta T_{t+1}}{1 + r} = -\dfrac{\Delta T}{(1+r)^2} \cdot r > 0
\end{align*}

\item 

\begin{align*}
\Delta C = \dfrac{r}{1 + r} \cdot \Big( - \dfrac{\Delta T}{1 + r} - \dfrac{\Delta T}{1+r}\Big) =  \dfrac{r}{(1+r)^2} \cdot \Delta T [ 1 + \dfrac{1}{1 + r} + \cdots ] \\
\Delta C = \dfrac{-\Delta T}{1 + r }
\end{align*}

\item 

\begin{align*}
\Delta C = \dfrac{r}{1+r} \dfrac{1}{(1+r)^2} \Delta Y_{t+2} = \dfrac{r}{(1+r)^3} \Delta Y_{t+2} 
\end{align*}
\end{enumerate}
\end{enumerate}
\end{otherlanguage*} 
\end{document}
