\documentclass{article}
\usepackage{amsmath}
\usepackage{tikz}
\usepackage{pgfplots}
\usepackage{epsfig}  
\usepackage[T2A,T1]{fontenc}
\usepackage[utf8]{inputenc}
\RequirePackage{amsfonts}
\usepackage[russian,english]{babel}

\begin{document}
\title{\foreignlanguage{russian}{Михайлов, семинар 9}}
\maketitle
\begin{otherlanguage*}{russian}
\section*{Предпосылки задачи}
\begin{align*}
M_t \rightarrow \text{military goods} \\ 
N_t \rightarrow \text{non-military goods} \\ 
i = \{1, \ldots, K \} \\
\alpha_i \rightarrow \text{i-ый индивид} \\
\alpha_i \in \{ 0; 1\} \\
V_i = \alpha_i \cdot \sqrt{M_t} + (1 - \alpha) \cdot \sqrt{N_t} \\
\alpha_i = 0 \rightarrow V_i = \sqrt{N_t} \rightarrow \text{пацифист} \\ 
\alpha_i = 1 \rightarrow V_i = \sqrt{M_t} \rightarrow \text{милитарист} \\ 
W \rightarrow \text{welfare} \rightarrow \text{благосостояние} \\
D \rightarrow \text{debt} 
\end{align*}
Ещё важные нюансы: 
\begin{itemize}
\item Избирательная система бинарная, варианты только 0 и 1 
\item Выборы происходят по принципу абсолютного большинства 
\item выбор соответствует $ \alpha_t^{\text{med}}$
\item $ V_i^{\text{voter}} = V_i^{\text{political}} $
\end{itemize}
\subsubsection*{Задача 2. }
\begin{align*}
\mathcal{\pi} \rightarrow \text{Вероятность  переизбраиния милитариста в } t = 2 \\ 
\end{align*}
Подход решения к этой задаче: backward induction, решение с конца. 

Пусть мы оказались в периоде $t = 2$

$D$ пришёл из $t = 1$

В периоде $ t = 1 $
\begin{align*}
\begin{cases} 
\alpha^{\text{med}}_{t=2} = 1 \\
V_{i, t=2} = \alpha_i \cdot \sqrt{M_2} + (1 - \alpha_i) \cdot \sqrt{N_2} \rightarrow \max_{N_2, M_2} \\
\text{s.t. } M_2 + N_2 = W - D \ge 0 \\
M_2 = W - D; \,\, N_2 = 0 
\end{cases}
\begin{cases}
\alpha^{\text{med}}_{t=2} = 0 \\
V_{i, t=2} = \alpha_i \cdot \sqrt{M_2} + (1 - \alpha_i) \cdot \sqrt{N_2} \rightarrow \max_{N_2, M_2} \\
\text{s.t. } M_2 + N_2 = W - D \ge 0 \\
N_2 = W - D; \,\, M_2 = 0 
\end{cases} 
\end{align*}

\begin{align*}
&\begin{cases}
\alpha_{t=1}^{\text{med}} = 1 \rightarrow \text{милитарист пришёл к власти} \\
V_{i, t=1} = \mathbb{E} \Big( \sum_{t=1}^2 \alpha_i \cdot \sqrt{M_t} + (1 - \alpha_i) \cdot \sqrt{N_t} \Big) = \alpha_i \cdot \sqrt{M_1} + (1 - \alpha_i) \cdot \sqrt{N_1} + \\ +  \pi \cdot \Big( \alpha_i \cdot \sqrt{M_2} + (1 - \alpha_i) \cdot \sqrt{N_2} \Big)  +  (1 - \pi)  \cdot \Big( \alpha_i \cdot \sqrt{M_2} + (1 - \alpha_i) \cdot \sqrt{N_2}  \Big) \\
\text{s.t. } \begin{cases} 
M_1 + N_1 \le W + D \\
M_2 + N_2 \le W - D 
\end{cases}  \Rightarrow - W \le D \le W \\
\text{Подстановка. Зануляем лишнее}.\\
V_{i, t=1} = \sqrt{M_1} + \sqrt{N_1} + \pi \cdot \sqrt{M_2} = \sqrt{W + D} + \pi \cdot \sqrt{W - D} \\
V = \sqrt{W + D} + \pi \sqrt{W - D} \rightarrow \max_{D} \\
\frac{\partial V}{\partial D} = \dfrac{1}{2 \sqrt{W + D}} - \dfrac{\pi}{2 \sqrt{W - D}} = 0 \\ 
D^* = \dfrac{(1 - \pi)^2}{(1 + \pi^2)} \cdot W \\
M^*_1 = W  + \dfrac{(1 - \pi^2)}{(1 + \pi^2)} \cdot W 
\end{cases} \\
&\begin{cases}
\alpha_{t=1}^{\text{med}} = 0 \rightarrow \text{пацифист пришёл к власти} \\
V = \sqrt{N_1} + (1 - \pi) \cdot \sqrt{N_2} \rightarrow \max_{D} \\ 
\dfrac{\partial W}{\partial D}  = \dfrac{1}{2 \sqrt{W + D}} - \dfrac{(1 - \pi)}{2 \sqrt{W - D}} = 0  \\
D^*_1 = \dfrac{(1 - (1 - \pi)^2)}{(1 + (1 - \pi)^2)} \cdot W 
\end{cases}
\end{align*}
\subsubsection*{Задача 3}
\begin{align*}
& L \rightarrow \text{labour} \\ 
& K \rightarrow \text{capitalist} \\
& U_L^{\text{outside}} = U_K^{\text{outside}} = 0 \\
& F(X \le x) \\ 
& \Pi \sim \mathcal{U} [A; B] \\
& W = \text{const} = 300 \\ 
& T = \text{const} = 150 \\
& x \in (0; T) \rightarrow \text{величина налога, которую платят капиталисты} \\
& U_L  = W - (T - x) \rightarrow \text{располагаемый доход рабочих} \\ 
\end{align*}
\begin{align*}
CDF: F( \Pi \le \pi) = \begin{cases} 0 & \pi \le A \\ \dfrac{\pi - A}{B - A} & \pi \in (A; B) \\ 1 & \pi \ge B \end{cases} \\
P (\Pi \ge \pi) = 1 - P( \Pi \le \pi) 
\end{align*}
\begin{enumerate}
\item Условие проведения реформы: $ \Pi - x \ge U^{\text{outside}}_K = 0 $ 

\begin{align*}
\Pi \ge x \\
P(\Pi \ge \pi = x) = 1 - P ( \Pi \le \pi = x) = 1 - \begin{cases} 0 & x \le A \\ \dfrac{x - A}{B - A} & x \in (A; B) \\ 1 & x \ge B \end{cases} = \\
= \begin{cases} 1 &  x \le A \\ \dfrac{x - A}{B - A} & x \in (A; B) \\ 0 & x \ge B \end{cases} 
\end{align*}
\item Записать выигрыш рабочего как функцию от предлагаемого капиталистом платежа 

\begin{align*}
V_L = P ( \Pi \ge X) \cdot (W - T + X) + (1 - P( \pi \ge X)) \cdot U^{\text{outside}}_L = ( U^{\text{outside}}_L = 0 ) = \\
= (W - T + X) \cdot \begin{cases} 
1 & x \le A \\ 
\dfrac{B - X}{B - A} & X \in (A, B) \\
0 & X \ge B 
\end{cases} = \begin{cases} W - T + X & X \le A \\ \dfrac{(B - X)(W - T + X) }{B - A} & X \in (A, B) \\ 0 & X \ge B \end{cases} = \\
= \begin{cases} W - T + X & X \le A \\ \dfrac{(B - X) (W - T + X)}{B - A} & X \in (A, B) \\ 0 & X > B \end{cases} \\
\dfrac{\partial V_L}{\partial X} = \begin{cases} 
1 & X \le A \\ 
\frac{- B - T + W + 2 X}{A - B} = 0  & X \in (A, B) \\
0 & X \ge B 
\end{cases} \\
\dfrac{\partial ^ 2 V_L}{\partial X^2} = \Big(\dfrac{B - (W - T) - 2 X}{B - A} \Big) = \dfrac{2}{A - B} < 0 \rightarrow 
\text{concave} \\
\Rightarrow X^* = \dfrac{B - (W - T) }{2} \le A \\
\end{align*}
\begin{align*}
X^* = \begin{cases} A & 2 A \ge B - (W - T) \\ \dfrac{B - (W - T)}{2}& \text{else} \end{cases} 
\end{align*}
Дальше подставить, не знаю зачем
\begin{align*}
X^* = \begin{cases} 100 & 200 \ge T \\ \dfrac{T}{2} & \text{else} \end{cases}  \\ \text{Если } T = 150 \text{, смотрим на } F(\ldots) \text{ выше } \rightarrow P^* = 1 \\
\text{Если } T = 300 \Rightarrow X^* = 150  \rightarrow P^* = 0.75
\end{align*}
\end{enumerate}
\end{otherlanguage*} 
\end{document}