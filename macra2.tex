\documentclass{article}
\usepackage{amsmath}
\usepackage{epsfig}  
\usepackage[T2A,T1]{fontenc}
\usepackage[utf8]{inputenc}
\usepackage[russian,english]{babel}

\begin{document}
\title{\foreignlanguage{russian}{Макроэкономика 2}}
\maketitle
\begin{otherlanguage*}{russian}
\section{\foreignlanguage{russian}{Межвременной выбор}}
\begin{equation}
C_t = \bar{C} + \operatorname{mpc} \cdot Yd_t
\end{equation}
блин так поспать хочется

Почему потребительские расходы в долгосрочном периоде сглажены? Почему? 
Чето про динамическую задачу оптимального выбора потребителя. Рациональные ожидания. Согласны? Узнали?

Обозначения:

$C_t$ - потребление агрегированного блага в периоде t

$S_t$ - сбережения из трудового дохода в периоде t

$Y_t$ - трудовой доход в периоде t

$Yd_t$ - располагаемый трудовой доход в периоде t

$A_t$ - богатство в периоде t (на начало периода t) 

$r_t$ - реальная рыночная ставка процента в периоде t

$u(C_t)$ - мгновенная полезность домашнего хозяйства от потребления агрегированного блага в периоде t

$u(C_1)$ где $C_1$ - это потребление барана.
\begin{enumerate}
\item Целевая функция полезности домашнего хозяйства, которое живёт $T$ лет:

$ \sum_{i=1}^{T-1} \frac{u(C_t)}{(1 + \rho)^t} \rightarrow \max $

где: $\frac{1}{(1+\rho)^t} = \beta$  - дисконт-фактор

Свойства мгновенной функции полезности:

$u^{'}(C) > 0; u^{''}(C) < 0$

Согласны?
\item Динамическое бюджетное ограничение

$A_{t+1} = (A_t + S_t) (1 + r) $

\item Межвременное бюджетное ограничение

\begin{align}
A_{t+1} = (1 + r) (A_t + S_t)  \\
A_1 = (1 + r) (A_0 + S_0) \rightarrow A_0 = \frac{A_1}{1+r} - S_0 \\
A_2 = (1 + r) (A_1 + S_1) \rightarrow A_1 = \frac{A_2}{1 + r} - S_1 \\
A_3 = (1 + r) (A_2 + S_2) \rightarrow A_2 = \frac{A_3}{1 + r} - S_2
\end{align}
Мораль: 
\begin{equation}
A_t = - \sum_{\tau = 0}^{T-t-1} \frac{S_{t+r}}{(1 + r)^{\tau}} + \frac{A_T}{(1+r)^{T-t}}
\end{equation}

Согласны?
\item Решение задачи домашнего хозяйства
\begin{align}
\sum_{t=1}^{T-1} \frac{u(C_t)}{(1 + \rho)^2} \rightarrow \max_{C_t \ge 0} \,\,\,\, \forall t \in [0; T -1]\\
A_{t+1} = (A_t + Y^d_t - C_t) (1 + r)  \\
\Lambda = \sum_{t=0}^{T-1} \frac{u(C_t)}{(1 + \rho)^t} + \sum_{t=0}^{T-1} \lambda_t \cdot \Big( A_{t+1} - (A_t + Y^d_t - C_t) (1 + r) \Big)
\end{align}
Согласны?
\begin{align}
\frac{\partial \Lambda}{\partial C_t} = \frac{u^{'}(C_t)}{(1 + \rho)^t} + \lambda_t (-1) (1 + r) =  0 \\
\frac{\partial \Lambda}{\partial C_{t+1}} = \frac{u^{'}(C_{t+1})}{(1 + \rho)^{t+1}} + \lambda_{t+1} (-1) (1+r) = 0
\end{align}
Интересный факт: 
\begin{equation}
\frac{\frac{u^{'}(C_t)}{(1 + \rho)^t}}
{\frac{u^{'}(C_{t+1})}{(1 + \rho)^{t+1}}} = \frac{\lambda_t}{\lambda_{t+1}}
\end{equation}
\item Условие отсутствия игр Понци (условие трансверсальности) 
\begin{equation}
\frac{A_t}{(1+r)^{T - t}} = 0
\end{equation}
Предполагается, что $A_T = 0 $ (индивид не оставляет ни наследства, ни долгов после оконч). Тогда: 
\begin{equation}
A_t = - \sum_{\tau = 0}^{T-t-1} \frac{S_{t + \tau}}{(1 + r)^{\tau)}}
\end{equation}
Согласны?
\item Конечный вид межвременного бюджетного ограничения
\begin{equation}
\sum_{t=0}^{T-1} \frac{Y^d_t}{(1 + r)^t} + A_0 = \sum_{t=0}^{T-1} \frac{C_t}{(1+r)^t}
\end{equation}
\item Интересные факты про МБО 
\begin{align}
A_t > 0 \Rightarrow \sum_{\tau = 0}^{T-t-1} \frac{S_{t + \tau}}{(1 + r)^{\tau}} < 0 \\
A_t < 0 \Rightarrow \sum_{\tau = 0}^{T-t-1} \frac{S_{t + \tau}}{(1 + r)^{\tau}} > 0
\end{align}
Т.е. при накоплении богатства в некоторый момент времени t, в будущем можно делать отрицательные сбережения. 

В случае наличия задолженности на момент t, в будущем необходимо делать положительные сбережения. 
\end{enumerate}
\end{otherlanguage*}
\end{document}
