\documentclass{article}
\usepackage{amsmath}
\usepackage{tikz}
\usepackage{pgfplots}
\usepackage{epsfig}  
\usepackage[T2A,T1]{fontenc}
\usepackage[utf8]{inputenc}
\usepackage[russian,english]{babel}

\begin{document}
\title{\foreignlanguage{russian}{Макра-2 Семинар 1}}
\maketitle
\begin{otherlanguage*}{russian}
\begin{enumerate}
\item Комплексная задача на модель межвременного выбора 
\begin{equation}
U(C_t) = u(C_1) + \frac{1}{1 + \rho} \cdot u(C_2)
\end{equation}

где $ \rho $ - это норма субъективного межвременного дисконтирования, $ u $ - функция мгновенной полезности, обладающая стандартными свойствами: $ u^{'} > 0$, $ u^{''} < 0 $
\begin{enumerate}

\item Динамическое бюджетное ограничение

\begin{enumerate}
\item Пусть $ A_1 = 0$. Надо написать динамические бюджетные ограничения для первого и второго периода. 
\begin{align}
S_t = Y^d_t - C_t \\
A_{t+1} = (A_t + S_t) ( 1 + r) \Rightarrow A_2 = S_1 (1 + r)\\
A_{t+2} = (A_{t+1} + S_{t+1}) (1 +r) \Rightarrow A_3 = (A_2 + S_2) (1 + r) ` 
\end{align}  

\item Пусть $ A_1 > 0 $. Тогда динамические бюджетные ограничения выглядят иначе:
\begin{align}
A_2 = (S_1 + A_1) (1 + r) \\
A_3 = (S_2 + A_2) (1 + r) 
\end{align}

\item Как связаны сбережения в периоде $ t $ и начальное богатство потребителя в периоде $ t + 1 $? 
\begin{align}
\frac{A_2}{1 + r} = S_1 +A_1 \rightarrow S_1 = - A_1 + \frac{A_2}{1 + r}\\
\frac{A_3}{1 + r} = S_2 + A_2 \Rightarrow S_2 = - A_2 + \frac{A_3}{1 + r}
\end{align}

\item Каковы будут $ A_3, C_1, C_2 $, если ограничения на $ A_3 $ отсутствуют. 

$ C_1 = + \infty $ 

$ C_2 = + \infty $ 

$ A_3 = - \infty $ 
 
\item Из динамического бюджетного ограничения первого периода выразите $ A_2 $ и подставьте результ в динамическое бюджетное ограничение второго периода 

\begin{align}
A_2 = (A_1 + S_1) (1 +r) \\
A_3 = (A_2 + S_2) (1 + r) \\
A_3 = \Big( (A_1 + S_1) (1 + r) + S_2 \Big) (1 + r) \\
A_3 = A_1 (1 + r)^2 + S_1 (1 + r)^ 2 + S_2 ( 1+ r) \\
A_3 = A_1 (1 + r)^2 + (Y_1 - C_1) (1 + r)^ 2 + (Y_2 - C_2) ( 1+ r) 
\end{align}

\item Выразить $A_3$, если оно $ \ge 0 $ и нарисовать 

С точки зрения первого периода:
\begin{align}
A_1 + Y_1 - C_1 + \frac{Y_2}{1 + r} - \frac{C-2}{1 +r } \ge 0 \\
(A_1 + Y_1) + \frac{Y_2}{1 + r} \ge C_1 + \frac{C_2}{1 + r} 
\end{align}
С точки зрения второго периода:
\begin{align}
(1 + r)(A_1 + Y_1) + Y_2 \ge C_1 ( 1 + r) + C_2 
\end{align}


\begin{equation}
C_2 = (1 + r) (A_1 + Y_1) + Y_2 - (1 + r) C_1
\end{equation}


\pgfmathdeclarefunction{f_1}{1}{%
  \pgfmathparse{3 + -  1 * #1}%
}
\pgfmathdeclarefunction{u}{2}{%
  \pgfmathparse{#2 + 0.2 / #1}%
}

\begin{tikzpicture}
\begin{axis}[
  no markers, 
  domain=0:3, 
  samples=100,
  ymin=0,
  axis lines*=left, 
  xlabel=$C_1$,
  ylabel=$C_2$,
  every axis y label/.style={at=(current axis.above origin),anchor=south},
  every axis x label/.style={at=(current axis.right of origin),anchor=west},
  height=5cm, 
  width=12cm,
  xtick=\empty, 
  ytick=\empty,
  enlargelimits=false, 
  clip=false, 
  axis on top,
  grid = major,
  ]

 \addplot [very thick,cyan!20!black] {f_1(x)};    

 \addplot[only marks] coordinates {
 (0, 4) 
 (4, 0)
( 3, 0 )
( 0, 3 )};

 \node[below] at (axis cs:0.7, 3.5)  {$ (1 + r) (A_1 + Y_1) + Y_2 $}; 
  \node[below] at (axis cs:3, 1)  {$ 1 + r$}; 

\end{axis}



\end{tikzpicture}

\end{enumerate}

\item Различные виды ограничения ликвидности
\begin{enumerate}
\item Мягкое ограничение ликвидности

Предположено, что $r^{cr} > r^d$. Надо записать, как будет выглядеть БО. 
\begin{equation}
C_2 = \begin{cases}
(1 + r^d ) (A_1 + Y_1) - (1 + r^d ) C_1, & C_1 \le A_1 + Y_1\\ 
(1 + r^{cr} ) ( A_1 + Y_1) + Y_2 - (1 + r^{cr}) C_1 & C_1 > A_1 + Y_1  
\end{cases} 
\end{equation}
\begin{tikzpicture}
\begin{axis}[
 no markers, 
  domain=0:3, 
  samples=100,
  ymin=0,
  axis lines*=left, 
  xlabel=$C_1$,
  ylabel=$C_2$,
  every axis y label/.style={at=(current axis.above origin),anchor=south},
  every axis x label/.style={at=(current axis.right of origin),anchor=west},
  height=5cm, 
  width=8cm,
  xtick=\empty, 
  ytick=\empty,
  enlargelimits=false, 
  clip=false, 
  axis on top,
  grid = major,
]
\addplot[] coordinates {
(0, 2)
(0.5, 1.5)
(1, 0)};
 \addplot[only marks] coordinates {
(0, 2.5)
(1.5, 0)};
\node[below] at (axis cs:0.45, 2.3)  {$ (1 + r^d) (A_1 + Y_1) + Y_2$};
\node[below] at (axis cs:0.9, 0)  {$ A_1 + Y_1$}; 
\end{axis}
\end{tikzpicture}

\item Жесткое ограничение ликвидности 

Потребитель не имеет возможности брать деньги в кредит, но может осуществлять вклады.  
\begin{equation}
C_2 = \begin{cases} 
(1 + r^d) (A_1 + Y_1) + Y_2 - (1 + r^d) \\
\in [0; Y_2] 
\end{cases}
\end{equation}
\begin{tikzpicture}
\begin{axis}[
 no markers, 
  domain=0:3, 
  samples=100,
  ymin=0,
  axis lines*=left, 
  xlabel=$C_1$,
  ylabel=$C_2$,
  every axis y label/.style={at=(current axis.above origin),anchor=south},
  every axis x label/.style={at=(current axis.right of origin),anchor=west},
  height=5cm, 
  width=8cm,
  xtick=\empty, 
  ytick=\empty,
  enlargelimits=false, 
  clip=false, 
  axis on top,
  grid = major,
]
\addplot[] coordinates {
(0, 2)
(0.5, 1.5)
(0.5, 0)};
 \addplot[only marks] coordinates {
(0, 3)
(0.5, 1.5)
(1, 0)};
\node[below] at (axis cs:0.4, 1.6)  {$1 + r^d$};
\node[below] at (axis cs:0.5, 0)  {$ A_1 + Y_1$}; 
\end{axis}
\end{tikzpicture}
\end{enumerate}
\item Уравнение Эйлера

Задача потребителя
\begin{equation}
U = u(C_1) + \frac{1}{1 + \rho} u(c_2) \rightarrow \max_{C_1, C_2} 
\end{equation}
\begin{equation}
s.t. \,\,\, C_1 + \frac{C_2}{1 + r} = A_1 + Y_1 + \frac{Y_2}{1 + r}
\end{equation}
\begin{align}
\Lambda = u(C_1) + \frac{1}{1 + \rho} u(C_2) + \lambda \Big( A_1 + Y_1 + \frac{Y_2}{1 + r} - C_1 - \frac{C_2}{1 + r}\Big) \\ 
\Lambda \rightarrow \max_{C_1 \ge 0, C_2 \ge 0, \lambda > 0} \\
\Lambda^{'}_{C_1} = u^{'}_{C_1} - \lambda = 0 \rightarrow u^{'}_{C_1} = \lambda \\
\Lambda^{'}_{C_2} =\frac{1}{1 + \rho}  u^{'}_{C_2} - \frac{\lambda}{1 + r} = 0 \rightarrow \frac{u^{'}_{C_2}}{1 + \rho} = \frac{u^{'}_{C_1}}{1 + r} 
\end{align}
Уравнение Эйлера в итоге выглядит так: 
\begin{equation}
 \frac{u^{'}_{C_2}}{1 + \rho} = \frac{u^{'}_{C_1}}{1 + r} 
\end{equation}
\item Слайды 17-18 - ответы на 3.с
\item Функция полезности представлена функцией CRRA 

\begin{equation}
u(C_t)  = \frac{C^{1 - \theta}_t - 1}{1 - \theta} \rightarrow u^{'}_{C_t} = \frac{1}{1 - \theta} \cdot C^{1 - \theta}_\theta - 1 \frac{1}{1 - \theta})^{'}_{C_t} = \cdots = C_t^{- \theta} 
\end{equation}
\begin{align}
\frac{u^{'}(C_1)}{u^{'}(C_2) = \frac{1 + r}{1 + \rho}} \Rightarrow \frac{C_1^{-\theta}}{C_2^{-\theta}} = \frac{1 + r}{1 + \rho} \Rightarrow \Big( \frac{C_2}{C_1}\Big)^\theta = \frac{1 + r}{1 + \rho} \\
(\frac{C_2}{C_1} + 1 - 1)^\theta = \frac{1 + r}{1 + \rho} \Rightarrow \Big( \frac{C_2 - C_1}{C_1} + 1 \Big)^\theta = \frac{1 + r}{1 + \rho} \\
\ln \Big(\Big( \frac{C_2 - C_1}{C_1} + 1 \Big)^\theta \Big)= \ln (\frac{1 + r}{1 + \rho}) \\
\theta \ln (\frac{C_2 - C_1}{C_1} + 1) = \ln ( 1+ r) - \ln (1 + \rho) 
\end{align}
После разложения на ряд Тейлора получится 
\begin{equation}
\theta \cdot \frac{C_2 - C_1}{C_1} = r - \rho 
\end{equation}
$ \frac{C_2 - C_1}{C_1} $ можно назвать как темп прироста потребления. Ответ. Ответ на что? 
\item На сколько процентных пунктов изменится темп прироста потребления если ставка процента изменится на 1 пп 
\begin{equation}
g_c = \frac{1}{\theta} (r - \rho ) 
\end{equation}
Перепишем "в дельтах"
\begin{align}
\Delta g_c = \frac{1}{\theta} ( \Delta r - \Delta \rho) \\
\Delta g_c = \frac{1}{\theta} \Delta z = \frac{1}{\theta}
\end{align}
\end{enumerate}
\end{enumerate}
\end{otherlanguage*} 
\end{document}