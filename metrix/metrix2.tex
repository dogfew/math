\documentclass{article}
\usepackage{amsmath}
\usepackage{epsfig}  
\usepackage{tikz}
\usepackage{pgfplots}
\usepackage[T2A,T1]{fontenc}
\usepackage[utf8]{inputenc}
\usepackage[russian,english]{babel}

\begin{document}
\title{\foreignlanguage{russian}{Эконометрика-1 Лекция 2}}
\maketitle

\begin{otherlanguage*}{russian}
Линейная регрессионная модель для случая одной переменной
\begin{enumerate}
\item Линейная регрессионная модель 

\begin{equation}
Y_i = \beta_0 + \beta_1 X_i  + \varepsilon_i; \,\,\, i = 1, \ldots, n 
\end{equation}
$\beta_0, \beta_1$ - параметры которые необходимо оценить по выборке 

$(X_1, Y_1), \ldots, (X_n, Y_n). $

$ Y = \beta_0+ \beta_1 X $ - линия теоретиеской регрессии 

$\check{Y} = \check{\beta_0} + \check{\beta_1} \cdot X $ - линия выборочной регрессии

\begin{enumerate}
\item Остатки  регрессии

$\varepsilon_i,\,\,\,\, i = 1,\ldots, n$ - ошибки регрессии

$\check{Y}_i = \check{\beta}_0 + \check{\beta}_1 \cdot X_i$ - оценочное значение зависимой еременной

$ e_i = Y_i - \check{Y}_i, \,\,\,\, i = 1, \ldots, n $ - остатки регрессии

На картинке остатки регрессии - этр расстояние от линии до точки. 

\begin{tikzpicture}
\begin{axis}[
ylabel=Y,
xlabel=X]
\addplot[only marks] coordinates {
( 338.1, 266.45 )
( 169.1, 143.43 )
( 84.5, 64.80 )
( 120.5, 75.80 )
( 42.3, 34.19 )
( 21.1, 9.47 )};
\addlegendentry{Dots}
\addplot[mark = none] coordinates {( 350, 279 )
( 0, 0 )};
\addlegendentry{$\check{Y} = \check{\beta}_0 + \check{\beta}_1 X$}
\end{axis}
\end{tikzpicture}
\item Оценка параметров линейной регрессиии

Критерии оценки $ \sum_{i=1}^n g(e_i) \rightarrow \min $

\pgfmathdeclarefunction{square_x}{1}{%
  \pgfmathparse{#1 ^ 2 }%
}

\pgfmathdeclarefunction{abs_x}{1}{%
  \pgfmathparse{abs(#1)}%
}


Ну графики корчое такие 

\begin{tikzpicture}
\begin{axis}[
  no markers, 
  domain=-2:2, 
  samples=100,
  ymin=0,
  axis lines*=left, 
  xlabel=$x$,
  every axis y label/.style={at=(current axis.above origin),anchor=south},
  every axis x label/.style={at=(current axis.right of origin),anchor=west},
  height=5cm, 
  width=12cm,
  enlargelimits=false, 
  ]

 \addplot [very thick,cyan!50!black] {square_x(x)};
  \addplot [very thick,purple!50!black] {abs_x(x)};

\end{axis}
\end{tikzpicture}

\item Mean Squared Error

\begin{enumerate}
\item Явные аналитические формулы для оценок параметров
\item Не робастнонсть, то есть чувствительность к выбросам
\item МНК, OLS
\end{enumerate}
\begin{equation}
\sum_{i=1}^n e^2_i \rightarrow \min
\end{equation}

\item Mean Absolute Error

\begin{enumerate}
\item Оценки более робасты
\item Нет аналитической формулы для оценок, они являются решением некоторой здачи линейного программирования
\item Оценки медианной регрессии

\begin{equation}
\sum_{i=1}^n |e_i| \rightarrow \min 
\end{equation}
\end{enumerate}
\item Метод наименьших квадратов

RSS - residual sum of squares

$ RSS(\check{\beta_0}, \check{\beta_1}) = e_1^2 + \cdots + e^2_n \rightarrow \min$ 

Сумма остатков в квадрате, будьте здоровы
\begin{equation}
RSS = \cdots = \sum_{i=1}^n Y_i^2 + n \check{\beta_0}^2 + \check{\beta_1}^2 \sum_{i=1}^n X_i^2 - 2 \check{\beta_0} \sum_{i=1}^n Y_i - 2 \check{\beta_1} \cdot \sum_{i=1}^n X_i\cdot Y_i + 2 \check{\beta_0} \check{\beta_1} \sum_{i=1}^n X_i 
\end{equation}

окэй.

\begin{align}
\frac{\partial RSS}{\partial \check{\beta_0}} = 0 \Rightarrow 2 n \check{\beta_0} - 2 \sum_{i=1}^n Y_i + 2 \check{\beta_1} \sum_{i=1}^n X_i = 0 \\
n \check{\beta_0} + \check{\beta_1} \cdot \sum_{i=1}^n X_i = \sum_{i=1}^n Y_i \\
n \check{\beta_0} = \sum_{i=1}^n Y_i - \check{\beta_1} \cdot \sum_{i=1}^n X_i  \\
\cdots \cdots \\ 
\check{\beta_0} = \bar{Y} - \check{\beta_1} \cdot \bar{X} \\
\check{\beta_1} = \frac{\sum_{i=1}^n X_i Y_i - n \bar{X} \bar{Y}}{\sum_{i=1}^n X_i - n \bar{X} ^ 2}
\end{align}

Важный результат 
\begin{equation}
\bar{Y} = \check{\beta_0} + \check{\beta_1} \bar{X} 
\end{equation}

Ещё одна оценка $\beta_1$ 
\begin{equation}
\check{\beta_1}^{MRS} = \frac{\check{cov} (X, Y)}{\check{var}(X)}
\end{equation}
\end{enumerate}
\item Зачем это всё надо 

Не привязывайтесь пожалуйста к статистическим пакетам. Окэй? 
\end{enumerate}
\end{otherlanguage*}
\end{document}