\documentclass{article}
\usepackage{amsmath}
\usepackage{epsfig}  
\usepackage{tikz}
\usepackage{pgfplots}
\usepackage[T2A,T1]{fontenc}
\usepackage[utf8]{inputenc}
\usepackage[russian,english]{babel}

\begin{document}
\title{\foreignlanguage{russian}{Эконометрика-1 семинар}}
\maketitle

\begin{otherlanguage*}{russian}
Формула оценки 

0.2 дз + 0.2 семинары + 0.05 лекции + 0.2 кр + 0.35 экз


Случайные величины бывают дискретными и непрерывными. Дискретные, как правило, задаются таблицами, а непрерывные функцией плотности $f_x(x)$
\begin{align}
F_x(x) = \operatorname{P}(X \le x) \\
f_x(x) = \frac{\partial F_x(x)}{\partial x} \\
F_x(x) = \int_{-\infty}^x f_x(t) dt
\end{align}
Матожидание
\begin{equation}
\operatorname{E} (X) = \begin{cases} 
\sum_{i=1}^n x_i \cdot p_i \\
\int_{-\infty}^{+\infty} f_x(x) \cdot x dx 
\end{cases}
\end{equation}
Дисперсия
\begin{equation}
\operatorname{D} (X) = \begin{cases} 
\operatorname{E} \big( (x - \operatorname{E}(x))^2\big) \\
\operatorname{E}(X^2) - (E(X))^2
\end{cases}
\end{equation}
\begin{align}
\operatorname{cov}(X, Y) = E \big( (X - E(X)) (Y - E(Y)) \big) = E(XY) - E(X)(EY) \\
\operatorname{corr}(X, Y) = \frac{\operatorname{cov}(X,Y)}{\sqrt{D(X)D(Y)}}
\end{align}
Пример задачи на что-то дискретное
\begin{align}
\begin{pmatrix}
Y / X & 3 & 4 & 5 \\
2 & 0.2 & 0.2 & 0.1 \\
4 & 0.12 & 0.12 & 0.05 \\
6 & 0.08 & 0.08 & 0.05
\end{pmatrix} \\
\begin{pmatrix}
X & 3 & 4 & 5 \\
P & 0.4 & 0.4 & 0.2
\end{pmatrix} \\
\begin{vmatrix}
Y & 2 & 4 & 6 \\
P & 0.5 & 0.29 & 0.21
\end{vmatrix} \\
\begin{vmatrix}
Y | X = 4 & 2 & 4 & 6 \\
P & \frac{0.2}{0.4} & \frac{0.12}{0.4} & \frac{0.08}{0.4}
\end{vmatrix}
\end{align}
\begin{enumerate}
\item Виды распределений

Нормальное
\begin{equation}
X \sim N(\mu, \sigma^2) \rightarrow f(x) ={\frac {1}{\sigma {\sqrt {2\pi }}}}e^{-{\frac {1}{2}}\left({\frac {x-\mu }{\sigma }}\right)^{2}}
\end{equation}
Хи-квадрат 
\begin{equation}
\chi^2_k = Z_1^2 + \cdots + Z_k^2; \forall i: Z_i \sim N(0;1) 
\end{equation}
Стьюдент 
\begin{equation}
t_k = \frac{N(0;1)}{\sqrt{\chi^2_k / k}}
\end{equation}
Фишер
\begin{equation}
F_{m;n} = \frac{\chi^2_m / m}{\chi^2_n / n}
\end{equation}
\item Проверка гипотез

Дана выборка: $X_1, \cdots, X_n \sim N(\mu; \sigma^2)$

$H_0: \mu = \mu_0$

$H_1: \mu \ne \mu_0$ 

Подсчет $Z_{stat} = \frac{\bar{X} - \mu_0}{\sqrt{\sigma^2 / n}} \sim^{H_0} N(0;1)$.

Если не выполняется $|Z_{stat}| < Z_{\alpha}$, то $H_0$ отвергается

\pgfmathdeclarefunction{gauss}{3}{%
  \pgfmathparse{1/(#3*sqrt(2*pi))*exp(-((#1-#2)^2)/(2*#3^2))}%
}

\begin{tikzpicture}
\begin{axis}[
  no markers, 
  domain=0:6, 
  samples=100,
  ymin=0,
  axis lines*=left, 
  xlabel=$x$,
  every axis y label/.style={at=(current axis.above origin),anchor=south},
  every axis x label/.style={at=(current axis.right of origin),anchor=west},
  height=5cm, 
  width=12cm,
  xtick=\empty, 
  ytick=\empty,
  enlargelimits=false, 
  clip=false, 
  axis on top,
  grid = major,
  hide y axis
  ]

 \addplot [fill=purple!30, draw=none, domain=0:1] {gauss(x, 3, 1)} \closedcycle;
  \addplot [fill=purple!30, draw=none, domain=5:6] {gauss(x, 3, 1)} \closedcycle;
 \addplot [very thick,cyan!50!black] {gauss(x, 3, 1)};
\pgfmathsetmacro\valueA{gauss(1,3,1)}
\pgfmathsetmacro\valueB{gauss(2,3,1)}
\draw [gray] (axis cs:1,0) -- (axis cs:1,\valueA)
    (axis cs:5,0) -- (axis cs:5,\valueA);
\draw [gray] (axis cs:2,0) -- (axis cs:2,\valueB)
    (axis cs:4,0) -- (axis cs:4,\valueB);
    
\node[below] at (axis cs:1, 0)  {$- Z_{1 - \alpha / 2}$}; 
\node[below] at (axis cs:0.5, 0.1)  {$H_1$}; 
\node[below] at (axis cs:2, 0)  {$-Z_{stat}$}; 
\node[below] at (axis cs:3, 0.2)  {$H_0$}; 
\node[below] at (axis cs:4, 0)  {$+Z_{stat}$}; 
\node[below] at (axis cs:5, 0)  {$Z_{1 -\alpha / 2}$}; 
\node[below] at (axis cs:5.5, 0.1)  {$H_1$}; 
\end{axis}

\end{tikzpicture}

$\operatorname{P-value}$ - минимальный уровень значимости, при котором $H_0$ отвергается. 

\begin{tikzpicture}
\begin{axis}[
  no markers, 
  domain=0:6, 
  samples=100,
  ymin=0,
  axis lines*=left, 
  xlabel=$x$,
  every axis y label/.style={at=(current axis.above origin),anchor=south},
  every axis x label/.style={at=(current axis.right of origin),anchor=west},
  height=5cm, 
  width=12cm,
  xtick=\empty, 
  ytick=\empty,
  enlargelimits=false, 
  clip=false, 
  axis on top,
  grid = major,
  hide y axis
  ]

 \addplot [fill=blue!30, draw=none, domain=0:2] {gauss(x, 3, 1)} \closedcycle;
  \addplot [fill=blue!30, draw=none, domain=4:6] {gauss(x, 3, 1)} \closedcycle;
 \addplot [very thick,cyan!50!black] {gauss(x, 3, 1)};
\pgfmathsetmacro\valueA{gauss(1,3,1)}
\pgfmathsetmacro\valueB{gauss(2,3,1)}
\draw [gray] (axis cs:1,0) -- (axis cs:1,\valueA)
    (axis cs:5,0) -- (axis cs:5,\valueA);
\draw [gray] (axis cs:2,0) -- (axis cs:2,\valueB)
    (axis cs:4,0) -- (axis cs:4,\valueB);
    
\node[below] at (axis cs:1, 0)  {$- Z_{1 - \alpha / 2}$}; 
\node[below] at (axis cs:0.5, 0.1)  {$H_1$}; 
\node[below] at (axis cs:1.6, 0.1)  {$\frac{\operatorname{P-value}}{2}$}; 
\node[below] at (axis cs:2, 0)  {$-Z_{stat}$}; 
\node[below] at (axis cs:3, 0.2)  {$H_0$}; 
\node[below] at (axis cs:4, 0)  {$+Z_{stat}$}; 
\node[below] at (axis cs:5, 0)  {$Z_{1 -\alpha / 2}$}; 
\node[below] at (axis cs:5.5, 0.1)  {$H_1$}; 
\node[below] at (axis cs:4.4, 0.1)  {$\frac{\operatorname{P-value}}{2}$}; 
\end{axis}

\end{tikzpicture}

Мораль: $\operatorname{P-value} < \alpha \Rightarrow H_0$ отвергается. 

Задача 1

Оценки $\sim N(\mu;1) $

реализация выборки: $ [10,9, 8, 5, 3]$

$\begin{cases}
H_0: \mu = 8 \\
H_1: \mu \ne 8 
\end{cases}$

$Z_{stat} = \frac{7-8}{\sqrt{1 / 5}} = - 2.24$ 

$\operatorname{P-value} = (1 - \Phi (2.24)) \cdot 2 = 0.025$, где $\Phi(x) = \int_{-\infty}^x f_x(x) dx$

\item Свойства оценок 

$\theta$ - параметр

$\check{\theta}$ - оценка

Несмещенность: $\operatorname{E} (\check{\theta}) = \theta $

Состоятельность: $\check{\theta} \rightarrow^P_{n \rightarrow \infty} \theta $

Эффективность: $\min MSE $ в классе, где $MSE = E \big( (\check{\theta} - \theta) ^ 2 \big)$

Интересные факты:
$
\begin{cases}
\operatorname{E} (\check{\theta}) = \theta \\ \operatorname{D} (\check{\theta}) \rightarrow_{n \rightarrow \infty} 0
\end{cases} \Rightarrow$ состоятельность

\item Матрицы

Повторить следующее: 

\begin{enumerate}
\item Умножение матрицы на число

Все элементы матрицы умножаются на число
\item Сложение матриц

Матрицы должны быть одинакового размера. Тогда все элементы матриц складываются друг с другом
\item Умножение матриц

Если есть матрицы $A: \,\, l \cdot m$ и $B: m \cdot n$:
\begin{equation}
{\displaystyle A={\begin{bmatrix}a_{11}&a_{12}&\cdots &a_{1m}\\a_{21}&a_{22}&\cdots &a_{2m}\\\vdots &\vdots &\ddots &\vdots \\a_{l1}&a_{l2}&\cdots &a_{lm}\end{bmatrix}},\;\;\;B={\begin{bmatrix}b_{11}&b_{12}&\cdots &b_{1n}\\b_{21}&b_{22}&\cdots &b_{2n}\\\vdots &\vdots &\ddots &\vdots \\b_{m1}&b_{m2}&\cdots &b_{mn}\end{bmatrix}}.}
\end{equation}
То их произведение: 
\begin{equation}
{\displaystyle C={\begin{bmatrix}c_{11}&c_{12}&\cdots &c_{1n}\\c_{21}&c_{22}&\cdots &c_{2n}\\\vdots &\vdots &\ddots &\vdots \\c_{l1}&c_{l2}&\cdots &c_{ln}\end{bmatrix}},}
\end{equation}
где ${\displaystyle c_{ij}=\sum _{r=1}^{m}a_{ir}b_{rj}\;\;\;\left(i=1,2,\ldots l;\;j=1,2,\ldots n\right).}$
\item Минор
\begin{equation}
M^3_2 = \begin{vmatrix}\,\,1&4&\bullet \,\\\,\bullet &\bullet &\bullet \,\\-1&9&\bullet \,\\\end{vmatrix} = {\displaystyle {\begin{vmatrix}\,\,\,1&4\,\\-1&9\,\\\end{vmatrix}}=1\cdot 9-4\cdot (-1)=13}
\end{equation}
\item Определитель
\begin{equation}
\Delta =\sum _{j=1}^{n}(-1)^{1+j}a_{1j}{M}_{j}^{1}
\end{equation}
\item Присоединенная матрица
\begin{equation}
\mathbf {C} = 
\begin{bmatrix}
+ M^1_1 & - M^2_1 & + M^3_1 \\
- M^1_2 & + M^2_2 & - M^3_2 \\
+ M^1_3 & - M^2_3 & + M^3_3
\end{bmatrix}
\end{equation}
\begin{equation}
\operatorname{adj} A = C^T
\end{equation}
\item Обратная матрица
\begin{equation}
{\displaystyle A^{-1}={\frac {{\mbox{adj}}A}{|A|}},}
\end{equation}
\begin{equation}
A^{-1} \cdot A = \mathbf{E}
\end{equation}
где $\mathbf{E}$ - единичная матрица (из нулей, кроме главной диагонали, где единицы) 
\end{enumerate}

\end{enumerate}
\end{otherlanguage*}
\end{document}
