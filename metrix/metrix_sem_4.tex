\documentclass{article}
\usepackage{amsmath}
\usepackage{tikz}
\usepackage{pgfplots}
\usepackage{epsfig}  
\usepackage[T2A,T1]{fontenc}
\usepackage[utf8]{inputenc}
\usepackage[russian,english]{babel}

\begin{document}
\title{\foreignlanguage{russian}{Эконометрика, семинар 3}}
\maketitle
\begin{otherlanguage*}{russian}

\begin{align*}
y_i = \beta_0 + \beta_1 x_i  + \varepsilon_i \\ 
\check{y_i} = \check{\beta_0} + \check{\beta_1} X_i \\
\check{Var(\check{\beta_0})} = \dfrac{\sum X_i^2}{n \sum (X_i - \bar{X})^2} \cdot \sigma^2_{\varepsilon} \\ 
\check{Var(\check{\beta_1})} = \dfrac{1}{\sum (X_i - \bar{X})^2} \cdot \sigma^2_{\varepsilon} \\ 
\end{align*}

$ H_0: \beta = \beta^ * $ 

$ H_1: \beta \ne \beta^* $

$ t_{\operatorname stat} = \dfrac{\check{\beta} - \beta^*}{\check{s.e} (\check{\beta})} \sim t_{n-2} $

$ \operatorname{p-value} < \alpha \Rightarrow H_0 $ отвергаем

$ \operatorname{p-value} = 2 \cdot (1 - F_t (\operatorname{t-stat})) $

Через доверительынй интервал

$ \check{\beta} - t_{n-2, \alpha / 2} \cdot \check{s.e} (\check{\beta}) \le \beta \le \check{\beta} + t_{n-2, \alpha/2} \cdot \check{s.e} (\check{\beta}) $ 
\begin{enumerate}
\item 

$$ \widehat{cov} (\hat \beta_0, \hat \beta_1) = \dfrac{\sum X_i}{n \sum (X_i - \bar X )^2} \cdot \hat \sigma^2_{\varepsilon} $$

\item Рассмотрим модель 

\begin{align*}
\widehat{\text{Grade}}_i = 2.1 + 0.2 \text{Hours}_i \\
\widehat{cov} (\hat \beta_0, \hat \beta_1) = -0.06 \\
\widehat{var} (\hat \beta) = \begin{pmatrix}
0.8^2 & -0.06 \\
-0.06 & 1.2^ 2
\end{pmatrix} \\
\begin{cases}
H_0: \beta_0 + \beta_1 = 3 \\
H_1: \beta_0 + \beta_1 < 3 
\end{cases} \\
\text{t}_{\text{stat}} = \dfrac{\hat \beta_0 + \hat \beta_1 - 3}{\sqrt{\widehat{var} (\hat \beta_0 + \hat \beta_1)}} \\
\widehat{var} (\hat \beta_0 + \hat \beta_1) = \widehat{var} (\hat \beta_0) + \widehat{var} (\beta_1) + 2 \widehat{cov} (\hat \beta_0, \hat \beta_1) = 0.8 ^ 2 + 1.2 ^2 + 2 \cdot (-0.06) = 1.4^ 2 \Rightarrow \text{t}_{\text{stat}} = - 0.5 
\end{align*}
\begin{align*}
-\text{t}_{\text{crit}} = - \text{t}_{88 \cdot 0.05} = -1.66 \Rightarrow
\end{align*}
$H_0$ не отвергается 

\item Прогнозы 
\begin{enumerate}
\item Точечный

Подставляется $ \hat{\text{Grade}_j} = 2.13 $

\item Интервальный 

\begin{enumerate}
\item Для среднего значения

$ 2.13 \pm  t_{88, \dfrac{0.05}{2}} \cdot \sqrt{\widehat{var} (\widehat{Grade}_j}$
\begin{align*}
 \widehat{var} (\widehat{Grade}_j) = \widehat{var} (\hat \beta_0 + \hat \beta_1 \cdot 1/6) = \widehat{var} (\hat \beta_0) + \dfrac{1}{36} \cdot \widehat{\beta_1} + 2 \cdot \dfrac{1}{6} \cdot \widehat{cov} (\hat \beta_0, \hat \beta_1) = 0.8^2 \\
2.13 \pm 1.6 \rightarrow [0.53, 3.73]
\end{align*}
\item Для индивидуального значения
\begin{align*}
2.13 \pm t_{88, 0.05 / 2} \cdot \sqrt{\widehat{var} (\widehat{\text{Grade}_j} + \varepsilon_j)} = \widehat{var} (\widehat{Grade}_j) + \widehat{var} (\varepsilon_j) = 0.66 + \dfrac{176}{90-2} = 2.66 \\
2.13 \pm 2 \cdot 1.63 \Rightarrow [-1.13; 5.39] 
\end{align*}
\end{enumerate}
\end{enumerate}
\item Решение большой задачки 

У нас есть исследователи, которые полагают что влияние длины судьбы на ладони оказывает какое-то ввлияние на возраст вступления в брак  (зачем?) 

\begin{align*}
\begin{pmatrix}
cm & 4 & 6 & 8 \\
age & 20 & 30 & 25 
\end{pmatrix}
\end{align*}

\begin{enumerate}
\item $ \hat \beta_0$ $ \hat \beta_1 $ МНК 

\item $ TSS, ESS , RSS, R^2 $ 

\item $ \widehat{var} ( \hat \beta )  $ 

\item 


\item Решение  

\begin{align*}
\beta_0 = \bar{Y} - \hat \beta \bar{X} \\ 
X = \begin{pmatrix}
1 & 4 \\
1 & 6 \\
1 & 8 \\
\end{pmatrix} \\
\beta = (X^T \cdot X) ^ {-1} \cdot X^T \cdot y \\
\beta = \begin{pmatrix}
17.5 \\ 1.25
\end{pmatrix}
\end{align*}

\item 

\begin{align*}
TSS = \sum_{i=1}^\beta (Y_i - \bar{Y})^ 2 = (25 - 25)^ 2 + (30 - 25) ^ 2 + (25 - 20) ^ 2 = 50 \\
\hat{Y} = \begin{pmatrix}
22.5 \\ 25 \\ 27.5 
\end{pmatrix} \\
RSS = (y - X \cdot \beta)^T \cdot (y - X \cdot \beta) = 37.5 \\
\widehat{var} (\hat \beta_0) = \dfrac{\bar{X^2}}{\sum (x_i - \bar{Y})} \cdot \hat \sigma^2_{\varepsilon} = 181.2 \\ 
\widehat{var} (\hat \beta_1) = \dfrac{1}{\sum (x_i - \bar{x})} \cdot \hat \sigma^2_{\varepsilon} = 4.69 \\ 
\widehat{cov} (\beta_0, \beta_1) = \dfrac{\bar{X}}{\sum (X_i - \bar{X})^2} \cdot \hat \sigma_{\varepsilon}^ 2 = 28.125  
\end{align*}

\begin{align*}
H_0: \beta_1 = 0 \\ 
H_1: \beta_1 \ne 0 \\
t_{stat} = \dfrac{1.25}{\sqrt{4.69}} = 0.58 \\
t_{3-2, 0.025} = t_{1, 0.025} = 12.706 \\
\end{align*}

\item Интервальный для индивидуального 

\begin{align*}
30 \pm t_{1, 0.05 / 2} \cdot \sqrt{\widehat{Var} (\hat Age + \varepsilon)}
\end{align*}
\end{enumerate}

\item 

\begin{align*}
\hat Y_i = \hat \beta_i + \hat \beta_i \cdot X_i \\
\hat X_i = \hat \alpha_0 + \hat \alpha_1 \cdot Y_i 
\end{align*}

Доказать что $R^2 $ совпадают, $ \hat \beta_1 \cdot \hat \alpha_1 = R^2 $
\begin{align*}
R^2 = \dfrac{ESS}{TSS} = \widehat{corr}^2 (X; y) = \widehat{corr}^2 (y;X) 
\end{align*}
\end{enumerate}
\end{otherlanguage*} 
\end{document}
