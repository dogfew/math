\documentclass{article}
\usepackage{amsmath}
\usepackage{tikz}
\usepackage{pgfplots}
\usepackage{epsfig}  
\usepackage[T2A,T1]{fontenc}
\usepackage[utf8]{inputenc}
\usepackage[russian,english]{babel}

\begin{document}
\title{\foreignlanguage{russian}{Эконометрика, семинар 8}}
\maketitle
\begin{otherlanguage*}{russian}
\subsection*{Множественная регрессия и F-тесты}
\subsubsection*{Задача на множественную регрессию}
\begin{align*}
X^T \cdot X = \begin{pmatrix}
10 & 0 & 1 \\ 
0 & 20 & 2 \\
1 & 2 & 5
\end{pmatrix} \\
X^T \cdot y = \begin{pmatrix}
90 \\ 50 \\ -10 
\end{pmatrix} \\
y^T \cdot y = 1155 
\end{align*}
Напоминание: вид матрицы $ X $:
\begin{align*}
\begin{pmatrix}
1 & X_{11} & X_{21} \\
\ldots & \ldots & \ldots \\ 
1 & \ldots & X_{2n}
\end{pmatrix}
\end{align*} 
Тогда можно заметить, насчет этой $ X^T \times X $:
\begin{align*}
X^T \cdot X = \begin{pmatrix}
n & \sum X_{1i} & \sum X_{2i} \\
\ldots & \ldots & \ldots 
\end{pmatrix}
\end{align*}
\begin{align*}
X^T \times y = \begin{pmatrix}
\sum y_i \\
50 \\
-10 
\end{pmatrix} \\
y^T \times y = \sum y_i ^ 2 
\end{align*}
\begin{align*}
\hat \beta = (X^T \times X)^{-1} \times X^T \times y \\
(X^T \times X) ^{-1} = \dfrac{1}{det (X^T X)} \times A^T
\end{align*}
Дальше всё это благополучно считается на компьютере и получаются ответы для $ \hat \beta $ 
\begin{align*}
\hat \sigma^2_{\varepsilon} = \dfrac{RSS}{n-3} \\
RSS = (y - X \times \hat \beta)^T \times (y - X \times \hat \beta ) = Y^T - Y^T \times X \times \hat \beta - \hat \beta^T \times X^T \cdot y + \hat \beta^T \times X^T \times X \times \hat \beta 
\end{align*}
Правило транспонирования, которое следует помнить $ (A \times B)^T = B^T \times A^T  $

Дальше всё подставить и посчитать. 
\begin{align*}
\dfrac{RSS}{n-3} = \dfrac{100}{10-3} \approx 14.3 
\end{align*}
\begin{align*}
R^2 = \dfrac{ESS}{TSS} = 1 - \dfrac{RSS}{TSS} \\
TSS = \sum (y_i - \bar y)^ 2 = y^T \times y - 2 \bar y \cdot \sum y_i + \bar{y}^2 = \sum y_i^2 - n \cdot (\bar y ) ^ 2
\end{align*}
Дальше посчитать, вся необходимая информация имеется. 

Проверка значимости коэффициентов, всё это делается через т-статистику 

\begin{align*}
t_{\text{stat}} = \dfrac{\hat \beta - 0 }{\sigma_{\hat \beta} } \\
\widehat{\text{var} (\hat \beta)} = (X^T \times X)^{-1} \cdot \hat \sigma^2_{\varepsilon} = \ldots	= (X^T \times X ) ^{-1} \cdot \dfrac{RSS}{n-3} = \text{Дальше просто посчитать} 
\end{align*} 
Дальше найти критические значения статистики. 
\begin{align*}
t_{\text{crit}} = t_{n-3; \dfrac{0.05}{2}} 
\end{align*}
Их надо сравнить с расчётными значениями статистики. Они попадут левее и правее в хвосты. Итого модель такая: 
\begin{align*}
\hat y_i = 10 + 3 \cdot X_{1i} - 5 \cdot X_{2i} 
\end{align*}
Это надо как-то интерпретировать и т.д. 
\subsubsection*{F-тесты}
\begin{align*}
y_i = \beta_0 + \beta_1 \cdot X_{1i} + \ldots + \beta_{k} \cdot X_{ki} + \varepsilon_i 
\end{align*}
\begin{align*}
H_0: \begin{cases}
\beta = \ldots \\
\ldots \\
\beta_i = \ldots  
\end{cases}
H_1 : \begin{bmatrix}
\beta_1 \ne \ldots \\
\ldots \ldots \\
\beta_i \ne \ldots 
\end{bmatrix} \text{Хотя бы одно из условий не выполняется} \\
\text{Альтернативная форма записи:} \\
H_\alpha : \sum | \beta_i - \hat \beta | = 0 
\end{align*}

\subsubsection*{Алгоритм решения подобной задачи}
\begin{enumerate}
\item Оценить изначальную модель без ограничений $ \rightarrow RSS_{ur} $ 
\item Оценить модель с ограничениями из $ H_0 $. $ \rightarrow RSS_{R} (\text{restricted}) $ 	
\item $ F_{\text{расч.}} = \dfrac{(RSS_R - RSS_{ur}) / q}{RSS_{ur} / (n - k - 1)} $, где $ \begin{cases} q \rightarrow  \text{количество знаков '='} \\ k \rightarrow \text{количество переменных} \\ k+1 \rightarrow \text{количество переменных плюс константа} \end{cases} $
\item $
F_{\text{crit}} = F_{q; n-k-1; \alpha} 
$

\item Все что справа от $F_{\text{crit}} $ отвергается. 
\begin{align*}
TSS_{R, UR}  = \sum (Y_i - \bar{Y}) ^ 2 \\
F_{\text{расч}} = \dfrac{\Big( RSS_R / TSS - RSS_{UR} / TSS \Big) / q}{RSS_{UR} / (TSS \cdot (n - k - 1) )}  = \dfrac{\Big( (1 - R^2_R) - (1 - R^2_{ur})\Big) /q}{(1 - R^2_{ur} ) / (n - k - 1) } \\
\end{align*}
\subsubsection*{Подвид F-теста. Гипотеза об адекватности регрессии. (Или гипотеза о совместной значимости коэффициентов)  }
\begin{align*}
&y_i = \beta_0 + \beta_1 \cdot x_{1i} + \ldots + \beta_k \cdot x_{ki} + \varepsilon \\ 
&H_0 : \begin{cases}
\beta_1 = 0 \\
\beta_2 = 0 \\
\ldots \ldots \\ 
\beta_k = 0 
\end{cases} \Rightarrow \beta_1 = \beta_2 = \ldots = \beta_k = 0 \\
&H_1: \sum_{i=1}^k \beta_j^2 > 0 
\end{align*}
\begin{align*}
&F_{\text{стат}}  = \dfrac{(RSS_R - RSS_{UR}) / k}{RSS_{UR} / (n - k - 1) } \\
&\text{Restricted}: Y_i = \beta_0 + \varepsilon_i \Rightarrow \hat y_i = \hat \beta_0 \\
&\beta_0 = \bar{Y} \\
&RSS_{R} = \sum (Y_i - \hat \beta_0 ) ^ 2 = \sum (Y_i - \bar{Y} ) ^ 2 = TSS_{R, UR} \Rightarrow RSS_{R} = TSS  
\end{align*}
\begin{align*}
F_{\text{стат}} = \dfrac{(RSS - RSS_{UR}) / q }{RSS_{UR} / (n - k -1 ) } =  \dfrac{ESS_{UR}  / k }{RSS_{UR} / (n - k -1 ) } 
\end{align*}
\item Для парной регрессии: 
\begin{align*}
y_i = \beta_0 + \beta_1 \cdot X_i + \varepsilon \\ 
H_0: \beta_1 = 0 \rightarrow \begin{cases}
\text{F-тест} \\
\text{t-тест} \rightarrow \begin{cases} t_{cr} = t_{n-2; \alpha / 2} \\ (t_{n-2})^2 = F_{1; n-2} \end{cases}
\end{cases}
\end{align*}
\end{enumerate}
\end{otherlanguage*} 
\end{document}
