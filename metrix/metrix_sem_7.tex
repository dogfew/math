\documentclass{article}
\usepackage{amsmath}
\usepackage{tikz}
\usepackage{pgfplots}
\usepackage{epsfig}  
\usepackage[T2A,T1]{fontenc}
\usepackage[utf8]{inputenc}
\usepackage[russian,english]{babel}

\begin{document}
\title{\foreignlanguage{russian}{Эконометрика, семинар 10?}}
\maketitle
\begin{otherlanguage*}{russian}
\subsection*{Категориальные переменные}
\subsubsection*{Пример}
\begin{align*}
\text{сезон} = 
\begin{cases} 
1 & \text{зима} \\ 
2 & \text{весна} \\ 
3 & \text{лето} \\ 
4 & \text{осень} 
\end{cases} 
\end{align*}
Если категориальных переменных много, включаются dummy переменные на все категории в количестве $ n - 1 $. Это называется OneHotEncoding. Та категория, которую мы выкидываем, называется базовой категорией.  
\begin{align*}
\text{educ} = \begin{cases} 1 & \text{school} \\ 2 & \text{college} \\ 3 & \text{university} \\ 4 & \text{Phd} \end{cases} \\
\text{В этой штуке мы вводим дамми на максимальный уровень образования} \\ 
\begin{pmatrix} \\
 & Dschool & Dcollege & Duni & Dphd \\
\text{school} & 1 & 0 & 0 & 0 \\
\text{college} & 0 & 1 & 0 & 0 \\
\text{uni} & 0 & 0 & 1 & 0 \\
\text{Phd} & 0 & 0 & 0 & 1 \\
\end{pmatrix}
\end{align*}
То есть кодирование выглядит, как ни страннно, так. 
\begin{align*}
\widehat{\text{wage}} = 20  + 5 \cdot \text{D college} + 8 \cdot \text{D uni} - 3 \cdot \text{D phd} + 2 \cdot \text{expirience} - 0.04 \cdot \text{expirience} ^ 2  
\end{align*}
Найдём опыт, который даёт максимальную заработную плату: 
\begin{align*}
\dfrac{2}{2 \cdot -0.04} = 25 
\end{align*}
Давайте поменяем базовую категорию. Теперь базовая категориия это ученая степень. Найдём 
\begin{align*}
&\widehat{\text{wage(\text{school})}} = 20 + 2 \cdot \text{expirience} - 0.04 \cdot \text{expirience} ^ 2  \\
&\widehat{\text{wage(\text{college})}} = 25 + 2 \cdot \text{expirience} - 0.04 \cdot \text{expirience} ^ 2  \\
&\widehat{\text{wage(\text{university})}} = 28 + 2 \cdot \text{expirience} - 0.04 \cdot \text{expirience} ^ 2  \\
&\widehat{\text{wage(\text{Phd})}} = 17 + 2 \cdot \text{expirience} - 0.04 \cdot \text{expirience} ^ 2  \\
& \widehat{\text{wage}} = 17 + 3 \cdot \text{D school} + 8 \cdot \text{D college} + 11 \cdot \text{D uni}  + 2 \cdot \text{expirience} - 0.04 \cdot \text{expirience} ^ 2
\end{align*}
Перекодируем dummy, чтобы она была на все оконченные уровни образования. Это будет "премия" за повышение категории. 
\begin{align*}
\begin{pmatrix} \\
 & Dschool^* & Dcollege^* & Duni^* & Dphd^* \\
\text{school} & 1 & 0 & 0 & 0 \\
\text{college} & 1 & 1 & 0 & 0 \\
\text{uni} & 1 & 1 & 1 & 0 \\
\text{Phd} & 1 & 1 & 1 & 1 \\
\end{pmatrix}
\end{align*}
\begin{align*}
\widehat{\text{wage}} =  20 + 5 \cdot \text{D college}^* + 3 \cdot \text{D uni}^*  -11  \cdot \text{Phd}^* +  2 \cdot \text{expirience} - 0.04 \cdot \text{expirience} ^ 2
\end{align*}
\subsection*{Тест Чоу}
\begin{align*}
n \rightarrow \begin{cases} n_1 \\ n_2 \end{cases} \\
\text{F}_{\text{stat}} = \dfrac{\Big(RSS_0 - (RSS_1 + RSS_2) \Big)/ (k + 1)}{(RSS_1 + RSS_2) / \Big( n - 2 \cdot (k + 1) \Big)} \\
\text{Как у нас получаются такие степени свободы?} \\ 
df RSS = n - k - 1  \\
\text{числитель}: k + 1 = n - k - 1 - (n_1 - k - 1 + n_2 - k - 1 ) \\
\text{знаменатель}: n - 2 (k + 1) = n_1 - k - 1 + n_2 - k - 1 
\end{align*}
Если у нас 3 разбиения: 
\begin{align*}
\text{F}_{\text{stat}} = \dfrac{\Big( RSS_0 - (RSS_1 + RSS_2 + RSS_3) \Big) / (2k +2)}{\Big( RSS_1 + RSS_2 + RSS_3 \Big) / (n - 3 (k + 1))}
\end{align*}
\subsubsection*{Выбор функциональной формы модели}
\begin{itemize}
\item lin-lin 
\begin{align*}
y = X \cdot \beta  + \varepsilon \\
\dfrac{\partial y_i}{\partial x_k} = \beta_k 
\end{align*}
\item lin-log 
\begin{align*}
y = \ln X \cdot \beta + \varepsilon \\
\dfrac{\partial y_i}{\partial x_k} = \dfrac{1}{X_k} \cdot \beta_k \Rightarrow \Delta y =  \dfrac{\Delta x}{x} \cdot \beta  \\
\end{align*}
\item log lin 
\begin{align*}
\ln y = X \cdot \beta + \varepsilon \\ 
\dfrac{1}{y_i} \cdot \dfrac{\partial y_i}{\partial x_k} = \beta_k \Rightarrow \Delta y = \Delta X \cdot y \cdot \beta \\
\end{align*}
\item log-log 
\begin{align*}
\ln y_i = \ln X \cdot \beta \\
\dfrac{1}{y_i} \cdot \dfrac{\partial y_i}{\partial x_i} = \dfrac{1}{x_k} \cdot \beta_k \\
\beta_k = \dfrac{x_k}{y_i} \cdot \dfrac{\partial y_i}{\partial x_k} = \dfrac{\partial y_i / y_i}{\partial x_k / x_k } \rightarrow \text{что-то вроде эластичности игрека по икс} \\
\Delta y = \dfrac{\Delta x}{x} \cdot y \cdot \beta  
\end{align*}
\end{itemize}
\end{otherlanguage*} 
\end{document}
