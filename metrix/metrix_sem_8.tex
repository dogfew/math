\documentclass{article}
\usepackage{amsmath}
\usepackage{tikz}
\usepackage{pgfplots}
\usepackage{epsfig}  
\usepackage[T2A,T1]{fontenc}
\usepackage[utf8]{inputenc}
\usepackage[russian,english]{babel}

\begin{document}
\title{\foreignlanguage{russian}{Эконометрика, семинар 11?}}
\maketitle
\begin{otherlanguage*}{russian}
\subsection*{$R^2$ adjusted? }
\subsubsection*{Используется для сравнения моделей}
\begin{align*}
R^2_{adjusted} = 1 - \dfrac{RSS / (n - k - 1) }{TSS / (n - 1) } = 1 - \dfrac{RSS}{TSS} \cdot \dfrac{n-1}{n-k-1} \\
\text{добавляем регрессор } k \uparrow \Rightarrow \begin{cases} 
R^2 \uparrow & \text{почти всегда} \\
R^2 = \text{const} & \text{t-stat} = 0 \\
R^2_{\text{adj}} \uparrow & |\text{t-stat}| > 1 \\
R^2_{\text{adj}} \text{const} & |\text{t-stat}| =  1 \\
R^2_{\text{adj}} \downarrow & |\text{t-stat}| < 1 \\
\end{cases}
\end{align*}
\subsubsection*{Преобразование Бокса-Кокса}
Это не степени это типа какие-то преобразования
\begin{align*}
y^{(\lambda)} = \begin{cases} 
\dfrac{y^{\lambda} - 1 }{\lambda} & \lambda \ne 0 \\
\ln y & \lambda = 0 
\end{cases} \\
X_j^{(\theta_j)} = \begin{cases} 
\dfrac{X_j^{\theta_j} - 1}{\theta_j} & \theta_j \ne 0 \\
\ln X_j & \theta_j = 0 
\end{cases} \\
y^{(\lambda)} = \beta_0 + \langle \beta, X \rangle + \varepsilon \\
\varepsilon \sim N(0, \sigma^2) \\
\mathcal{L} = \prod_{i=1}^n  \varepsilon_i \rightarrow \max_{\theta, \lambda, \beta} \\
\end{align*}
\subsubsection*{Ошибка спецификации}
\begin{itemize}
\item Недоопределенность (выкинули важные переменные) 
\begin{align*}
\hat \beta \text{становятся смещенными и несостоятельными} \Rightarrow \text{мы получаем неверную интерпретацию} 
\end{align*}
\item Переопределенность (добавили лишние переменные) 
\begin{align*}
\hat \beta \rightarrow \text{неэффективные, но все равно состоятельные и несмещенные} \Rightarrow \text{Var} (\hat \beta) \uparrow \Rightarrow \\
\Rightarrow \text{t-stat} \downarrow \Rightarrow \text{мы чаще будем принимать гипотезу о незначимости} 
\end{align*}
\end{itemize}
\subsubsection*{Почему недоопределенность это страшно и т.д.}
\begin{align*}
& \text{Данные генерируются так: } y_i = \beta_0 + \langle \beta, X \rangle + \varepsilon_i \\
& \text{Оценили без } X_k \Rightarrow \varepsilon^* = \varepsilon + \beta_k \cdot X_k \\
& \hat \beta = (X^T  \times X) ^{-1} \times X^T \times y \\
& E ( \hat \beta ) = (X^T X) ^{-1} X^T \cdot E(X \beta + \varepsilon^*) = \\
& = (X^T X) ^ { -1} X^T X \beta + (X^T X ) ^{-1} X^T \cdot E ( \varepsilon^* ) 
\end{align*} 
Если раньше $ E(\varepsilon) = 0 $, то теперь не факт. 
Смещения может не быть, если 
\begin{enumerate}
\item $ \beta_k = 0 \Rightarrow \varepsilon^* = \varepsilon +( \beta_k X_k = 0) = \varepsilon $
\item если $ X_k $ никак не связан с другими переменными 
\subsubsection*{Ошибка спецификации, задача. }
\begin{align*}
\text{Настоящая модель: } \text{bonus} = 30 - 2 \cdot \text{missed} + 5 \cdot \text{talents} + \varepsilon \\
\text{cov} ( \text{missed}, \text{talents}) = 2 \\
\text{var} (\text{missed}) = 4 \\
\widehat{\text{var}} (\text{talents}) = 9 \\
\text{непонятно как оценивать талант, поэтому выкидываем его из модели} \\
\widehat{\text{bonus}} = \hat \beta_0 + \hat \beta_1 \cdot \text{missed} \\
\text{Найти смещение } \hat \beta_1 \\
\hat \beta_1 = \dfrac{\bar{XY} - \bar{X} \cdot \bar{Y} }{\bar{X^2} - (\bar{X}) ^ 2} = \dfrac{\text{cov} (b, m) }{\text{var}(m) } \\
\hat{\text{cov}} (b, m ) = \hat{\text{cov} } (30 - 2 \cdot M + 5 \cdot T + \varepsilon, M ) =  -2 \hat D (M) + 5 \widehat{\text{cov}} (M, T) \\
\text{cov} (B, M) = \dfrac{-2 \text{var} (M) + 5 \text{cov} (M, T)}{\text{var}(M)} = - 2 + 5 \cdot \dfrac{\text{cov} (M, T) }{\text{var}(M)} \\
\text{plim} \hat  \beta_1 = -2  + 5 \dfrac{\text{cov} (M, T) }{\text{var}(M)} = -2 + 5 \cdot \dfrac{1}{2} = 0.5 
\end{align*} 
Вспомнить конструкцию эту вот 
\begin{align*}
\lim_{n \rightarrow \infty} P(| \hat{\beta} - \text{plim} \,\, \hat \beta | > \varepsilon ) = 0 \\
\text{plim} \, \hat \beta \rightarrow \text{что-то типа матожидания} 
\end{align*}
\end{enumerate}
\subsubsection*{Оценка коэффициента в модели  пропущенными переменными для парной регресси в общем виде}
\begin{align*}
 \hat \beta_1 = \beta_1 + \beta_2 \cdot \dfrac{\text{cov} (X_1, X_2)}{\text{var}(X_1) }
\end{align*}
\subsubsection*{RESET тест Рамсея}
Он может сказать, что вы не включили полиномы и т.д. в модель, хотя их мб стоило бы включить 
\begin{align*}
\text{Оцениваем } y_i = \beta_0 + \langle \beta, X_i \rangle + \varepsilon_i \rightarrow \text{получаем } \hat{y_i} \\
\text{вставляем } \hat{y} \text{ в модель и заново оцениваем } \\
y_i = \beta_0 + \beta_1 \cdot X_{1, i} + \ldots + \beta_K + X_{k, i} + \gamma_1 \cdot \hat{y_i} ^2 + \gamma_2 \cdot \hat{y_i}^3 + \ldots + \gamma_n \cdot \hat{y_i}^n + \varepsilon^*_i \\
\text{F-test на совместную незначимость всех } \gamma: \\
H_0: \gamma_1 = \gamma_2 = \ldots = \gamma_m = 0 \Rightarrow \text{в изначальной модели нет пропущенных полиномов} 
\end{align*}
\end{otherlanguage*} 
\end{document}
