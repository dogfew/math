\documentclass{article}
\usepackage{amsmath}
\usepackage{tikz}
\usepackage{pgfplots}
\usepackage{epsfig}  
\usepackage[T2A,T1]{fontenc}
\usepackage[utf8]{inputenc}
\usepackage[russian,english]{babel}

\begin{document}
\title{\foreignlanguage{russian}{Эконометрика, семинар 12}}
\maketitle
\begin{otherlanguage*}{russian}
\subsection*{Мультиколлинеарность}
\begin{align*}
\text{мультиколлинеарность} \rightarrow 
\begin{cases}
\text{строгая, (теоретическая, идеальная) } \\
\text{нестрогая (квази-, практическая)} 
\end{cases}
\end{align*} 
\begin{itemize}
\item строгая 

Существует линейная зависимость между столбцами. Тогда $ \det (X^T \cdot X) = 0 $ и $ \hat \beta $ не посчитается. Нарушается условие теоремы Гауса Маркова. 

\item нестрогая 

существует какая-то тесная связь между признаками. Условия теоремы Гауса Маркова не нарушаются. 
$ \text{det} (X^T \cdot X) \rightarrow 0 $ 

$ R ^ 2 $ высокий, но почти все коэффициенты незначимы.  

ПРИЧИНЫ:
\begin{enumerate}
\item мало наблюдений, но много переменных
\item неправильная или неудачная спецификация модели 
\item использование близких по смыслу переменных 
\item неудачная выборка
\item просто не повезло
\end{enumerate}
ДИАГНОСТИКА
\begin{enumerate}
\item $ R ^ 2 $ высокий, почти все коэффициенты незначимы 
\item посчитать парные корреляции между признаками 
\item если корреляция между признаками больше 0.9, это уже плоховато
\item VIF (variance infation factor ) : $ VIG_j == \dfrac{1}{1 - R^2_j} $ 
\begin{align*}
\text{var} \beta_j = \dfrac{\hat \sigma^2_{\varepsilon}}{\sum (X_{ji} - \bar{X}_j)^2} \cdot \dfrac{1}{1 - R^2_j} = \dfrac{\hat \sigma^2_{\varepsilon}}{TSS_j} \cdot \dfrac{1}{1 - R^2_j} \\
TSS_j == (n - 1) \cdot \text{var} X_j
\end{align*}
\item VIF показывает, во сколько рах увеличение дисперсии $ B_j $ по сравнению с тем, когда иксы независимы. Эмпирически умные люди выявили что 
\begin{align*}
\begin{cases}
VIF > 10 \rightarrow \text{плохо} \\
VIF < 3 \rightarrow \text{хорошо}
\end{cases}
\end{align*}
\end{enumerate}
\end{itemize}
\subsubsection*{Пример}
Известно, что $ \text{corr} (x, z) = 0.5 $ 

Найти $ \min VIF_x$ в регрессии $ y_i = \beta_1 + \beta_2 \cdot x_i + \beta_3 \cdot z_i + \beta_4 \cdot w_i + u_i $ 

Чтобы найти $ VIF_x $ мы строим регрессию для $ X $ с помощью остальных переменных модели
\begin{align*}
VIF_x = \dfrac{1}{1 - R^2_x} \rightarrow R^2_x \text{ из регрессии } X_i \sim z_i + w_i \\
R^2_x = \widehat{\text{cov}} (X, \hat X) \\
\widehat{\text{cov}} (x, z) = R ^ 2  \text{ из модели } x_i \sim z_i \Rightarrow R^2 = 0.5 ^ 2 = 0.25 \\
\min \text{VIF}_x = \dfrac{1}{1 - 0.25} = \dfrac{4}{3}
\end{align*}
\subsubsection*{CN (conditional number, число обусловленности)}
\begin{align*}
CN(X^T X) = \sqrt{\dfrac{\lambda_{\max}}{\lambda_{\min}}} \\
\text{ Собственные значения, как их считать? Отсюда: } \\
\text{det} (X^T  X - \lambda \cdot \mathcal{I} ) = 0  \\
CI_{index} = \dfrac{\lambda_{\max}}{\lambda_{\min}} 
\end{align*}
\subsubsection*{Пример}
\begin{align*}
X^T X = \begin{pmatrix}
10 & 0 & 1 \\
0 & 20 & 2 \\
1 &  2 & 5
\end{pmatrix} \Rightarrow \\
\Rightarrow \text{det} \begin{pmatrix}
10 - \lambda & 0 & 1 \\
0 & 20 - \lambda & 2 \\
1 & 2 & 5 - \lambda 
\end{pmatrix} = \ldots = - \lambda ^ 3 + 35 \lambda ^ 2 - 345 \lambda + 940 \Rightarrow
\lambda  = \begin{cases}
\lambda_1 = 4.56 \\
\lambda_2 = 10.18 \\
\lambda_3 = 20.26
\end{cases}
\end{align*}
ЛЕЧЕНИЕ: 
\begin{enumerate}
\item Выкинуть лишние регрессоры (переменные) 
\item Изменить спецификацию модели (функциональные преобразования) 
\item Найти больше данных 
\item Снижение размерности (метод главных компонент) 
\item Регуляризация 
\item Забить
\end{enumerate}
\end{otherlanguage*} 
\end{document}
