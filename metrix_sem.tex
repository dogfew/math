\documentclass{article}
\usepackage{amsmath}
\usepackage{epsfig}  
\usepackage[T2A,T1]{fontenc}
\usepackage[utf8]{inputenc}
\usepackage[russian,english]{babel}

\begin{document}
\title{\foreignlanguage{russian}{Эконометрика-1 семинар}}
\maketitle

\begin{otherlanguage*}{russian}
Формула оценки 

0.2 дз + 0.2 семинары + 0.05 лекции + 0.2 кр + 0.35 экз

Кстати семерша приятная по первым впечатлениям

Она всё-таки базируется

Вообще, в МО мы тоже можем сказать, что произошло с данными и посмотреть, как какие факторы влияют.

Случайные величины бывают дискретными и непрерывными. Дискретные, как правило, задаются таблицами, а непрерывные функцией плотности $f_x(x)$
\begin{align}
F_x(x) = \operatorname{P}(X \le x) \\
f_x(x) = \frac{\partial F_x(x)}{\partial x} \\
F_x(x) = \int_{-\infty}^x f_x(t) dt
\end{align}
Матожидание
\begin{equation}
\operatorname{E} (X) = \begin{cases} 
\sum_{i=1}^n x_i \cdot p_i \\
\int_{-\infty}^{+\infty} f_x(x) \cdot x dx 
\end{cases}
\end{equation}
Дисперсия
\begin{equation}
\operatorname{D} (X) = \begin{cases} 
\operatorname{E} \big( (x - \operatorname{E}(x))^2\big) \\
\operatorname{E}(X^2) - (E(X))^2
\end{cases}
\end{equation}
\begin{align}
\operatorname{cov}(X, Y) = E \big( (X - E(X)) (Y - E(Y)) \big) = E(XY) - E(X)(EY) \\
\operatorname{corr}(X, Y) = \frac{\operatorname{cov}(X,Y)}{\sqrt{D(X)D(Y)}}
\end{align}
Пример задачи на что-то дискретное
\begin{align}
\begin{pmatrix}
Y / X & 3 & 4 & 5 \\
2 & 0.2 & 0.2 & 0.1 \\
4 & 0.12 & 0.12 & 0.05 \\
6 & 0.08 & 0.08 & 0.05
\end{pmatrix} \\
\begin{pmatrix}
X & 3 & 4 & 5 \\
P & 0.4 & 0.4 & 0.2
\end{pmatrix} \\
\begin{vmatrix}
Y & 2 & 4 & 6 \\
P & 0.5 & 0.29 & 0.21
\end{vmatrix} \\
\begin{vmatrix}
Y | X = 4 & 2 & 4 & 6 \\
P & \frac{0.2}{0.4} & \frac{0.12}{0.4} & \frac{0.08}{0.4}
\end{vmatrix}
\end{align}
\begin{enumerate}
\item Виды распределений
Нормальное
\begin{equation}
X \sim N(\mu, \sigma^2) \rightarrow f(x) ={\frac {1}{\sigma {\sqrt {2\pi }}}}e^{-{\frac {1}{2}}\left({\frac {x-\mu }{\sigma }}\right)^{2}}
\end{equation}
Хи-квадрат 
\begin{equation}
\chi^2_k = Z_1^2 + \cdots + Z_k^2; \forall i: Z_i \sim N(0;1) 
\end{equation}
Стьюдент 
\begin{equation}
t_k = \frac{N(0;1)}{\sqrt{\chi^2_k / k}}
\end{equation}
Фишер
\begin{equation}
F_{m;n} = \frac{\chi^2_m / m}{\chi^2_n / n}
\end{equation}
\item Проверка гипотез
Дана выборка: $X_1, \cdots, X_n \sim N(\mu; \sigma^2)$
$H_0: \mu = \mu_0 \\
H_1: \mu \ne \mu_0$ 
Подсчет $Z_{stat} < \frac{\bar{X} - \mu_0}{\sqrt{\sigma^2 / n}} \sim^{H_0} N(0;1)$.
Если $Z_{stat} < Z_{\alpha}$ , то $H_0$ отвергается
$\operatorname{P-value}$ - минимальный уровень значимости, прик котором $H_0$ отвергается.
Мораль: $\operatorname{P-value} < \alpha \Rightarrow H_0$ отвергается. 

Задача 1

Оценки $\sim N(\mu;1) $

реализация выборки: $ [10,9, 8, 5, 3]$

$\begin{cases}
H_0: \mu = 8 \\
H_1: \mu \ne 8 
\end{cases}$

$Z_{stat} = \frac{7-8}{\sqrt{1 / 5}} = - 2.24$ 

$\operatorname{P-value} = (1 - \Phi (2.24)) \cdot 2 = 0.025$, где $\Phi(x) = \int_{-\infty}^x f_x(x) dx$

\item Свойства оценок 

$\theta$ - параметр

$\check{\theta}$ - оценка

Несмещенность: $\operatorname{E} (\check{\theta}) = \theta $

Состоятельность: $\check{\theta} \rightarrow^P_{n \rightarrow \infty} \theta $

Эффективность: $\min MSE $ в классе, где $MSE = E \big( (\check{\theta} - \theta) ^ 2 \big)$

Интересные факты:
$
\begin{cases}
\operatorname{E} (\check{\theta}) = \theta \\ \operatorname{D} (\check{\theta}) \rightarrow_{n \rightarrow \infty} 0
\end{cases} \Rightarrow$ состоятельность

\item Матрицы


\end{enumerate}
\end{otherlanguage*}
\end{document}
