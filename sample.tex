\documentclass{article}
\usepackage{amsmath}
\usepackage{tikz}
\usepackage{pgfplots}
\usepackage{epsfig}  
\usepackage[T2A,T1]{fontenc}
\usepackage[utf8]{inputenc}
\usepackage[russian,english]{babel}

\begin{document}
\title{\foreignlanguage{russian}{Заголовок}}
\maketitle
\begin{otherlanguage*}{russian}
\section{\foreignlanguage{russian}{Секция}}
Пример графика в TeX (соединение точек) 
\begin{tikzpicture}
\begin{axis}[
ylabel=Data,
xlabel=time ]
\addplot[only marks] coordinates {
( 338.1, 266.45 )
( 169.1, 143.43 )
( 84.5, 64.80 )
( 42.3, 34.19 )
( 21.1, 9.47 )};
\addlegendentry{Dots}
\addplot[mark = none] coordinates {( 350, 279 )
( 0, 0 )};
\addlegendentry{Line}
\end{axis}
\end{tikzpicture}

Пример считывания соединений точек из файла foo.dat
3\begin{tikzpicture}
\begin{axis}[ no markers,
xmin=0, xmax=20 ]
\addplot table[x=x,y=t] {foo.dat};
\end{axis}
\end{tikzpicture}

\pgfmathdeclarefunction{gauss}{3}{%
  \pgfmathparse{1/(#3*sqrt(2*pi))*exp(-((#1-#2)^2)/(2*#3^2))}%
}3

\begin{tikzpicture}
\begin{axis}[
  no markers, 
  domain=0:6, 
  samples=100,
  ymin=0,
  axis lines*=left, 
  xlabel=$x$,
  every axis y label/.style={at=(current axis.above origin),anchor=south},
  every axis x label/.style={at=(current axis.right of origin),anchor=west},
  height=5cm, 
  width=12cm,
  xtick=\empty, 
  ytick=\empty,
  enlargelimits=false, 
  clip=false, 
  axis on top,
  grid = major,
  hide y axis
  ]

 \addplot [fill=purple!30, draw=none, domain=0:4] {gauss(x, 3, 1)} \closedcycle;
 \addplot [very thick,cyan!50!black] {gauss(x, 3, 1)};
\pgfmathsetmacro\valueA{gauss(1,3,1)}
\pgfmathsetmacro\valueB{gauss(2,3,1)}
\draw [gray] (axis cs:1,0) -- (axis cs:1,\valueA)
    (axis cs:5,0) -- (axis cs:5,\valueA);
\draw [gray] (axis cs:2,0) -- (axis cs:2,\valueB)
    (axis cs:4,0) -- (axis cs:4,\valueB);
    
\node[below] at (axis cs:1, 0)  {$\mu - 2\sigma$}; 
\node[below] at (axis cs:2, 0)  {$\mu - \sigma$}; 
\node[below] at (axis cs:3, 0)  {$\mu$}; 
\node[below] at (axis cs:4, 0)  {$\mu + \sigma $}; 
\node[below] at (axis cs:5, 0)  {$\mu + 2\sigma $}; 
\end{axis}



\end{tikzpicture}
\end{otherlanguage*}
\end{document}	